\section{Discussions}
\label{sec:discussions}

\subsection{Des restaurations de l'haplotype sauvage inattendues}
\label{subsec:discu-restaur}

\afterpage{%
  \null
  \vfill
  \begin{figure}[ht]
    \centering
    \includegraphics[width = \textwidth]{img/randomweak.png}
    \caption{\label{fig:simul-count} }
  \end{figure}
  \vfill
  \newpage
}
% TODO parler des erreurs de séquençage, qu'on y a répondu par des manips
% parallèles.
Certains marqueurs sont d'une qualité inférieure au reste des marqueurs sur la
lecture, qui est la marque de pics secondaires sur les électrophérogrammes.
% TODO figure pics secondaires
Des pics secondaires apparaissent quand la population d'amplicon séquencée n'est
pas homogène au site considéré. De façon surprenante, les deux pics présents à
un marqueur donné correspondent toujours soit à la base sauvage, soit à la base
synthétique introduite. Cette donnée peut être interprétée de deux façons. i)
S'il s'agit de la marque d'un signal biologique, la colonie dont la zone de
recombinaison a été séquencée montre une hétérogénéité au marqueur considéré.
Autrement dit, une part de la population séquencée a converti la base, l'autre
partie l'a conservée. ii) La présence de pics secondaires à une position de
marqueur peut aussi indiquer la présence de contaminations entre les puits des
plaques de séquençage, qui ont pu avoir lieu au cours du séquençage ou au cours
des PCRs. Si ces pics secondaires sont la marques d'une hétérogénéité dans la
colonie séquencée, les séquences affectées devraient être réparties
aléatoirement dans la plaque séquencée. Nous avons montré par simulation
qu'elles ne l'étaient pas (voir figure \ref{fig:simul-count}). Les séquences
montrant des pics secondaires sont plus souvent voisines avec une autre séquence
affectée que si elles étaient réparties aléatoirement.



% TODO discuter de ces cas de restaurations. comment elles ont pu arriver ?
% mécanismes, double crossing over ? Normalement ce qu'on voit chez la levure
% c'est un mécanisme global sur l'ensemble de la trace de conversion, pas
% localement. Lesecque et al excluent le BER comme possible mécanisme de bgc.

% TODO calculer la taille d'échantillon nécessaire pour observer un biais de
% longueur ou d'introduction d'un SNP.
Pour détecter une différence de l'ordre de \num{30} paires de bases entre
longueurs de régions converties correspondant à l'introduction d'un marqueur
supplémentaire, en supposant qu'elles sont normalement distribuées, avec l'écart
type observé dans nos résultats (\num{227}), avec une puissance de \num{0.8} et
un niveau de confiance de \num{0.05}, il faudrait une taille d'échantillon de
\num{717} (test \textrm{t}) par groupe.

Cette taille d'échantillon n'est pas envisageable en utilisant les techniques
actuelles : le séquençage par la technique de Sanger implique un coût prohibitif
pour une telle taille d'échantillon. Nous avons étudié différents procédés
permettant de séquencer les régions converties à haut débit. Les séquençage à
haut débit permettrait de réduire les coûts de séquençage. Il nécessite
néanmoins de prendre en compte les facteurs suivants. i) La taille des lectures
est de l'ordre de \(2 \times 300\) paires de bases par lecture, soit un total de
\(600\)pb, inférieure à la taille du locus considéré. ii) Lors de la
construction de la librairie de séquence, les amplicons sont mélangés les uns
aux autres. Il faut pouvoir les discriminer les uns des autres facilement, de
façon à individualiser les évènements de recombinaison. iii) La divergence très
faible des amplicons peut introduire des artéfacts lors de la capture des
clusters par la caméra de séquençage.

% TODO parler des problèmes de second pic

% TODO parler des dinucléotides, de radman et cie.
% parler ici de la manip contrôle

% TODO parler du fait que le biais est généralement faible. Donner l'exemple des
% levures pour qui le biais est calculé.

% TODO count_last_snp. Il y a une sorte de balance. Peut être que ça
% dépend du premier SNP donneur ? Lesecque et al ;). Selon laurent ça dépend
% surtout d'un effet distance : plus on est loin du premier marqueur, plus on a
% de chance d'avoir basculé après le marqueur précédent. Il y a donc un effet de
% la distribution du dernier marqueur.

% TODO parler du fait que la distribution n'était pas attendu. référence aux
% mexicains.
% TODO placer figure de la distribution du point de recombinaison chez
% Rhizobium.

% TODO expliquer le point froid de recombinaison par le taux de GC ?
% TODO placer figure taux de gc dans rapport.

\afterpage{\blankpage}