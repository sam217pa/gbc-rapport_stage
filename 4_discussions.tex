\blankpage
\null
\vfill

% contient le schéma de la base secondaire et la figure obtenue par la
% simulation.
\begin{center}

  \begin{figure}

    \centering
    \tikzset{legende fleche/.style={Gray, dotted, thick, opacity = 0.6}}
    \rmfamily
    \scriptsize

    \begin{tikzpicture}[scale = 0.5]
      \node[above] at (0, 0) {\includegraphics[scale = 0.5]{img/pic_second.png}};
      % \draw[opacity = 0.2, line width = 1pt, Gray] (-15,0) grid (15, 15);
      % \draw[opacity = 0.2, line width = 0.005pt, step = 0.5, Gray] (-15,0) grid (15, +15);

      \draw[legende fleche] [->] (-6, 0)        node[ above] {Score de qualité} -- ++(4.5, 0) -- ++(1, 1);
      \draw[legende fleche, Black] [->] (-6, 7) node[ above] {Base G} -- ++(4.5, 0) -- ++(1, -1);
      \draw[legende fleche, Red3] [->] (-6, 11) node[ above] {Base T} -- ++(4.5, 0) -- ++(1, -1);

    \end{tikzpicture}

    \caption[Marqueur montrant des traces de contaminations]{%
      \textbf{Exemple de marqueur montrant des traces de
        contaminations} \\ \rmfamily Cet électrophérogramme montre les bases autour du
      marqueur à la position 200. Dans une région de qualité moyenne élevée (bases en
      5' et en 3'), le marqueur présente une trace de contamination par une autre
      base. La base déterminée est la base T mais une base G est présente dans la
      population d'amplicon séquencée.
      \label{fig:pic-second}
    }
  \end{figure}

  \begin{figure}[h!]

    \centering

    \tikzset{legende fleche/.style={Gray, thick}}
    \begin{tikzpicture}[scale = 0.3]
      \scriptsize
      \foreach \x in {1,2,...,12} {
        \foreach \y in {1,2,...,8} {
        \draw (\x, \y) circle (0.3cm) ;
      }}
    \draw[fill = Red] (8, 5) circle (0.3cm);
    \draw[fill = Red] (7, 6) circle (0.3cm);
    \draw[fill = Red] (7, 4) circle (0.3cm);
    \draw[fill = Red] (6, 5) circle (0.3cm);

    \draw[Gray, densely dotted] (5.5, 3.5) rectangle (8.5, 6.5);
    \draw[Gray, densely dotted] (9.5, 0.6) rectangle (12.4, 3.5);
    \draw[Gray] (0.5, 0.5) rectangle (12.5, 8.5);

    \draw[legende fleche] [<-] (7, 5) -- ++(1, 0.5) -- ++(5, 0)
    node[right, Black] {\(\bar{n} = \frac{4}{8} = \frac{1}{2}\)}
    ;

    \draw[legende fleche] [<-] (11, 2) -- ++(1, 2.0) -- ++(1, 0)
    node[right, Black] {\(\bar{n} = \frac{0}{8} = 0\)}
    ;

    \node[right] at (12.5, 8) {\(\bar{X} = 4\)};
    \node[right] at (12.5, 1) {\(\bar{N} = \sum_{i = 1}^{96}\frac{\bar{n_i}}{96}\)};

    \draw [->] (18.5, 4.5) -- ++(5, 0) node[midway, above] {\(\times 10000\)}
    node[right] {
      \includegraphics[width = 0.5\textwidth]{img/randomweak.png}
    };
    \end{tikzpicture}


    \caption[Des contaminations dues au hasard ?]{%
      \label{fig:simul-count}\textbf{Des contaminations dues au hasard ? } \\
      \rmfamily Par plaque de 96 puits, nous avons déterminé \(X\) le nombre de
      puits dont la séquence montre des traces de pics secondaires (voir
      figure~\ref{fig:pic-second}) et mesuré \(\bar{n}\) la moyenne du nombre de
      puits voisins contaminés. \(\bar{N}\) est la moyenne des 96 \(\bar{n}\)
      obtenus par plaque. Nous avons simulé \num{1e4} plaques avec \(X\) puits
      contaminés répartis aléatoirement, mesuré \(\bar{N}\) et comparé la valeur
      expérimentale de \(\bar{N}\) (trait vertical rouge) avec la distribution
      des \num{1e4} \(\bar{N}\) (en gris) Seules \(78 / 10000\) plaques simulées
      montrent un \(\bar{N}\) supérieur à la valeur expérimentale : la
      répartition des séquences contaminées dans les plaques ne peut pas être
      attribuée au hasard.
      %
    }
    % TODO légender figure pics secondaire simul
  \end{figure}

\end{center}

\vfill
\thispagestyle{empty}
\addtocounter{page}{-1}
\clearpage
\newpage

\section{Discussions}
\label{sec:discussions}

Nous avons séquencé un grand nombre de clones d'\emph{A. baylyi} recombinants à
un locus choisi de façon à introduire la correction des mésappariemments, un
processus qui est biaisé vers l'introduction des bases C et G chez certains
eucaryotes.

\subsubsection{Des erreurs de séquençage sans conséquences ?}
\label{sub:discu-conta}

% TODO parler des erreurs de séquençage, qu'on y a répondu par des manips
% parallèles.
Certains marqueurs sont d'une qualité inférieure au reste des marqueurs sur la
lecture. C'est la marque de pics secondaires sur les électrophérogrammes (voir
figure~\ref{fig:pic-second}). Des pics secondaires apparaissent quand la
population d'amplicon séquencée n'est pas homogène au site considéré. De façon
surprenante, les deux pics présents à un marqueur donné correspondent toujours
soit à la base sauvage, soit à la base synthétique introduite. Cette donnée peut
être interprétée de deux façons.

1) S'il s'agit de la marque d'un signal biologique, la colonie dont la zone de
recombinaison a été séquencée montre une hétérogénéité au marqueur considéré.
Autrement dit, une part de la population séquencée a converti la base, l'autre
partie l'a conservée. Nous avons séquencé à nouveau 31 isolats issus d'un clone
séquencé en premier lieu qui montrait des pics secondaires. Aucun des
sous-clones ne montrent l'allèle correspondant au pic secondaire (voir
figure~\ref{fig:confirm-haplotype}).

2) S'il s'agit d'un erreur de séquençage, la population d'amplicon séquencée est
hétérogène à cause de contaminations entre les puits des plaques, qui ont pu
avoir lieu au cours du séquençage ou au cours des PCRs. Dans ce cas, la
répartition dans la plaque des puits dont la séquence montre des traces de
contamination ne devrait pas être déterminée que par le hasard. Nous avons
montré par simulation qu'elles ne l'étaient pas (voir
figure~\ref{fig:simul-count}). Les séquences montrant des pics secondaires sont
plus souvent voisines avec une autre séquence affectée que si elles étaient
réparties aléatoirement. En conséquence, nous avons filtré de façon très
stringente les sites de faible qualité.

\subsection{Une distribution des points de recombinaison surprenante}

\afterpage{
  \null
  \vfill

  \begin{figure}[htbp]
    \centering


    \includegraphics[width = \textwidth]{img/gc_content.pdf}

    \caption[Taux de GC]{Taux de GC \\
    TODO légender}
    \label{fig:gctaux}
  \end{figure}


  \vfill
  \thispagestyle{empty}
  \addtocounter{page}{-1}
  \clearpage
  \newpage

}
L'hypothèse \ac{gbgc} prédit que la région convertie devrait être plus longue
lorsque la conversion introduit des bases GC que lorsqu'elle introduit des bases
AT. Nous n'avons pas détecté de différences significatives. En fait, la
distribution du point de recombinaison est assez uniforme, contrairement
à ce qui est décrit par Yáñez-Cuna \emph{et al.}\cite{yanez-cuna_biased_2015}.
Chez un modèle de \emph{Rhizobium etli}, la distribution de la longueur des
régions converties est bimodale, avec un pic à respectivement 120 et 600 bp.
Comment peut-on expliquer ces variations ?

\subsubsection{Une influence du taux de GC local ?}

La figure~\ref{fig:distrircb} montre un point froid de recombinaison dans la
région entre 600 et 800 bp, environ 200 paires de bases après l'origine de
l'hétéroduplex. Nous avons supposé que le taux de GC local diminuait localement
l'efficacité de la recombinaison : un taux de GC plus faible diminue le nombre
de liaisons hydrogène entre les brins et pourrait donc diminuer les probabilités
d'appariemment avec un brin d'ADN exogène. En effet, lors de la recherche par la
protéine RecA d'une matrice homologue permettant de réparer la lésion, la
reconnaissance est pûrement basée sur l'appariemment Watson-Crick entre les
brins\cite{lee_base_2015}. Nous avons donc représenté le taux de GC moyen par
fenêtre de 50bp au long de la séquence sauvage (voir figure~\ref{fig:gctaux}b).

Le point froid de recombinaison (fig.\ref{fig:gctaux}a) semble bien correspondre
à un taux de GC local plus faible (fig.\ref{fig:gctaux}b). Un taux de GC plus
faible pourrait donc conduire au rejet du brin homologue, et diminuer les
fréquences de recombinaison localement. C'est une observation qui va à
l'encontre de la théorie du \ac{gbgc} : si un taux d'AT élevé implique une
faible fréquence de recombinaison, il diminue les probabilités d'introduire des
bases GC par conversion génique. À l'inverse, les zones riches en GC feraient
plus souvent l'objet de conversion, ce qui conduirait à l'effacement progressif
du pic local de GC : c'est le paradoxe des points chauds de
recombinaison\cite{coop_live_2007}.

Lieb \emph{et al.}\cite{lieb_recombination_1985} ont montré chez \emph{E. coli}
que les enzymes du \ac{vsp} (Very Short Patch repair) répairaient un
mésappariemment dans un sens spécifique en fonction des bases au voisinage
immédiat du mésappariemment.

TODO terminer analyses des dinucléotides.

TODO mal dit ``Si ce genre de mécanisme'' est à l'œuvre chez \emph{A. baylyi},
les marqueurs AT et les marqueurs GC de nos constructions ne sont pas
rigoureusement dans le même contexte nucléotidique (voir
figure~\ref{fig:construct}). Pour palier à cet éventuel biais, il faudrait
transformer à nouveau un clone ayant converti tous les marqueurs par la séquence
sauvage : chaque évènement de conversion serait alors rigoureusement dans le
même contexte.

% ------------------------------------------------------------------------------
\subsection{Des restaurations de l'haplotype sauvage inattendues}
\label{subsub:discu-restaur}
% ------------------------------------------------------------------------------

% DONE discuter de ces cas de restaurations. comment elles ont pu arriver ?
% mécanismes, double crossing over ? Normalement ce qu'on voit chez la levure
% c'est un mécanisme global sur l'ensemble de la trace de conversion, pas
% localement. Lesecque et al excluent le BER comme possible mécanisme de bgc.
Au sein des régions converties, 14 cas de restauration de l'allèle sauvage ont
été détectés. Quels mécanismes peuvent les expliquer ? Ce ne sont probablement
pas des mutations spontanées : bien que la recombinaison soit un processus
mutagène en soi\cite{rodgers_error-prone_2016,hicks_increased_2010}, les
restaurations sont retrouvées spécifiquement aux positions des marqueurs, et
correspondent précisément à l'allèle sauvage. Une mutation spontanée pourrait
introduire aléatoirement l'une des quatre bases. S'il s'agit de l'action de la
machinerie de correction des mésappariemments, ces cas sont inattendus :
pourquoi la machinerie de correction des mésappariemments restaurerait
spécifiquement ces marqueurs dans une région qui présente un mésappariemment
toutes les 30 paires de bases ?

Dans le contexte de la conversion génique biaisée vers GC, est-ce que ces cas de
restaurations pourraient expliquer un biais ? Chez les eucaryotes, la correction
des mésappariemments dans les cellules en mitose est effectuée en partie par la
voie du \ac{ber} (Base Excision Repair). Cette voie excise spécifiquement une
base mésappariée en détectant les malformations qu'elle occasionne dans la
structure de la double-hélice, puis la remplace par la base complémentaire à
celle du brin opposé\cite{krokan_base_2013}. Cette voie a été considérée comme
l'un des moteurs possibles du \ac{gbgc} chez les
mammifères\cite{duret_biased_2009}, mais elle a été rejetée chez la
levure\cite{lesecque_gc-biased_2013}. Chez cette dernière, le \ac{gbgc} est
associé aux régions converties les plus longues et dans lesquelles tous les
allèles sont convertis depuis le même haplotype donneur.

Est-ce que le \ac{ber}, s'il est actif chez les procaryotes, pourrait conduire à
de la conversion biaisée vers GC ? L'étude des régions converties chez des
mutants d'\emph{A. baylyi} délétés pour les fonctions clés de la correction des
mésappariemments ($\Delta$MutS, $\Delta$MutH, $\Delta$MutL) devrait permettre de
répondre à cette question.

Ces cas de restaurations de l'haplotype sauvage ajoutent une incertitude sur la
nature des premiers marqueurs conservés (en rouge sur la
figure~\ref{fig:convtract}). En effet, ces marqueurs peuvent être la résultante
de trois choses : 1) ils ne font pas partie de l'hétéroduplex, et ne sont pas
soumis à de la conversion, 2) ils font partie de l'hétéroduplex, mais la
conversion génique conserve l'allèle sauvage ou 3) ils ont été d'abord convertis
par l'allèle donneur, puis restaurés. Dans tous les cas, la base séquencée
correspond à l'allèle sauvage du marqueur.

\subsection{Quelle est la fréquence de conversion en faveur de GC ?}
\label{subsec:discu-freq}

% ------------------------------------------------------------------------------
\subsection{Comment augmenter la puissance du test ?}
\label{subsub:discu-puissance}
% ------------------------------------------------------------------------------

% DONE calculer la taille d'échantillon nécessaire pour observer un biais de
% longueur ou d'introduction d'un SNP. voir [schema/longueur_traces.R]
Pour détecter une différence de l'ordre de \num{30} paires de bases entre
longueurs de régions converties correspondant à l'introduction d'un marqueur
supplémentaire, en supposant qu'elles sont normalement distribuées, avec l'écart
type observé dans nos résultats (\num{227}), avec une puissance de \num{0.8} et
un niveau de confiance de \num{0.05}, il faudrait une taille d'échantillon de
\num{717} (test \textrm{t}) par groupe.

Cette taille d'échantillon n'est pas envisageable en utilisant les techniques
actuelles : le séquençage par la technique de Sanger implique un coût prohibitif
pour une telle taille d'échantillon. Nous avons étudié différents procédés
permettant de séquencer les régions converties à haut débit. Les séquençage à
haut débit permettrait de réduire les coûts de séquençage. Il nécessite
néanmoins de prendre en compte les facteurs suivants. 1) La taille des lectures
est de l'ordre de \(2 \times 300\) paires de bases par lecture, soit un total de
\(600\)pb, inférieure à la taille du locus considéré. 2) Lors de la
construction de la librairie de séquence, les amplicons sont mélangés les uns
aux autres. Il faut pouvoir les discriminer les uns des autres facilement, de
façon à individualiser les évènements de recombinaison. 3) La divergence très
faible des amplicons peut introduire des artéfacts lors de la capture des
clusters par la caméra de séquençage.


% TODO parler du fait que le biais est généralement faible. Donner l'exemple des
% levures pour qui le biais est calculé.

% TODO lassalle et al : H.pylori transforme beaucuop, et ne montre pas de
% corrélation.

\afterpage{\blankpage}
