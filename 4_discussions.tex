% \blankpage
\null
\vfill

% contient le schéma de la base secondaire et la figure obtenue par la
% simulation.

\begin{figure}[htbp]
  \centering


  \includegraphics[width = \textwidth]{img/gc_content.pdf}

  \caption[Taux de GC]{Taux de GC \\
    TODO légender}
  \label{fig:gctaux}
\end{figure}

\vfill
\thispagestyle{empty}
\addtocounter{page}{-1}
\clearpage
\newpage

\section{Discussions}
\label{sec:discussions}

Nous avons séquencé un grand nombre de clones d'\emph{A. baylyi} recombinants à
un locus choisi de façon à introduire la correction des mésappariemments, un
processus qui est biaisé vers l'introduction des bases C et G chez certains
eucaryotes.

\subsection{Distribution des points de recombinaison}

L'hypothèse \ac{gbgc} prédit que la région convertie devrait être plus longue
lorsque la conversion introduit des bases GC que lorsqu'elle introduit des bases
AT. Nous n'avons pas détecté de différences significatives entre les deux
constructions donneuses pCG/AT et pAT/CG. Cependant, quelle que soit la
construction donneuse, la distribution du point de recombinaison suit
globalement le même patron, avec une zone de plus faible intensité entre
\num{600} et \num{800} pb et une autre autour de \num{350} pb. En fait, la
distribution du point de recombinaison est assez différente de celle décrite par
Yáñez-Cuna \emph{et al.}\cite{yanez-cuna_biased_2015} chez un modèle de
\emph{Rhizobium etli}. Ils observent que les recombinants changent d'haplotype
pour la majorité dans les \num{200} premières paires de bases, et seuls
\SI{10}{\percent} des clones ont une région convertie de plus de \num{800}pb.
Les différences entre leurs résultats et les nôtres peuvent potentiellement
s'expliquer par les différences de modèles et de locus. Nous avons en effet
supposé que la composition en base du locus avait un impact sur l'efficacité
locale de la recombinaison.

\subsubsection{Une influence du taux de GC local ?}

\afterpage{\blankpage}

La figure~\ref{fig:gctaux}a montre un point froid de recombinaison dans la
région entre \num{600} et \num{800} bp, environ \num{200} paires de bases après
l'origine de l'hétéroduplex. La figure~\ref{fig:gctaux}b montre l'évolution du
taux de GC par fenêtre de \num{50}pb. Le point froid de recombinaison à
\num{600}pb semble correspondre avec un taux de GC plus faible localement. Nous
avons donc supposé que l'efficacité de la recombinaison, qui dépend pour partie
de la probabilité de rejet du brin exogène et de la probabilité d'extension de
l'hétéroduplex, pourrait être influencée par le taux de GC local. Un taux de GC
plus faible diminue le nombre de liaisons hydrogène entre les brins et pourrait
donc diminuer les probabilités d'appariemment avec un brin d'ADN exogène. En
effet, lors de la recherche par la protéine RecA d'une matrice homologue
permettant de réparer la lésion, la reconnaissance est pûrement basée sur
l'appariemment Watson-Crick entre les brins\cite{lee_base_2015}.

En fait, la zone pauvre en GC à \num{600}pb correspond au taux de GC moyen du
génome d'\emph{A. baylyi} de \SI{40.3}{\percent}, mais le locus étudié a un taux
de GC moyen de \SI{46.8}{\percent}. Or ce locus est connu au laboratoire pour
permettre de bonnes efficacités de transformations. Si on suppose qu'un taux de
GC plus élevé entraîne une efficacité de recombinaison plus élevée, le taux de
GC à ce locus supérieur à celui du génome pourrait ainsi expliquer ces
efficacités de transformations. Cette supposition pourrait également expliquer
les corrélations observées par Lassalle \emph{et
  al.}\cite{lassalle_gc-content_2015} entre taux de GC et taux de recombinaison.

\subsubsection{Une influence des bases adjacentes ?}

Lieb \emph{et al.}\cite{lieb_recombination_1985} ont montré chez \emph{E. coli}
que les enzymes du \ac{vsp} (Very Short Patch repair) répairaient un
mésappariemment dans un sens spécifique en fonction des bases au voisinage
immédiat du mésappariemment. Cette activité de réparation est spécifiquement
liée au produit du gène \emph{vsr}\cite{hennecke_vsr_1991,lieb_specific_1983}.
\emph{A. baylyi} possède également certaines des enzymes impliquées dans les
voies du \ac{vsp}\cite{kanehisa_kegg_2016}.

Si ce genre de mécanisme est à l'œuvre chez \emph{A. baylyi}, les marqueurs AT
et les marqueurs GC de nos constructions ne sont pas rigoureusement dans le même
contexte nucléotidique (voir figure~\ref{fig:construct}). Pour palier à cet
éventuel biais, il faudrait transformer à nouveau un clone ayant converti tous
les marqueurs par la séquence sauvage : chaque évènement de conversion serait
alors rigoureusement dans le même contexte.

% ------------------------------------------------------------------------------
\subsection{Restaurations de l'haplotype sauvage}
\label{subsub:discu-restaur}
% ------------------------------------------------------------------------------
\afterpage{\blankpage}

% DONE discuter de ces cas de restaurations. comment elles ont pu arriver ?
% mécanismes, double crossing over ? Normalement ce qu'on voit chez la levure
% c'est un mécanisme global sur l'ensemble de la trace de conversion, pas
% localement. Lesecque et al excluent le BER comme possible mécanisme de bgc.
Au sein des régions converties, 14 cas de restauration de l'allèle sauvage ont
été détectés. Quels mécanismes peuvent les expliquer ? Ce ne sont probablement
pas des mutations spontanées : bien que la recombinaison soit un processus
mutagène en soi\cite{rodgers_error-prone_2016,hicks_increased_2010}, les
restaurations sont retrouvées spécifiquement aux positions des marqueurs, et
correspondent précisément à l'allèle sauvage. Une mutation spontanée pourrait
introduire aléatoirement l'une des quatre bases. Chez les eucaryotes, des cas
complexes de conversion ont déjà été
décrits\cite{martini_genome-wide_2011,yeadon_recombination_2001}. Chez la
levure, ces cas complexes ont été interprétés comme des basculements multiples
entre le choix des matrices permettant la réparation des
cassures\cite{hoffmann_trans_2005}.

Dans le contexte de la conversion génique biaisée vers GC, est-ce que ces cas de
restaurations pourraient expliquer un biais ? Chez les eucaryotes, la correction
des mésappariemments dans les cellules en mitose est effectuée en partie par la
voie du \ac{ber} (Base Excision Repair). Cette voie excise spécifiquement une
base mésappariée en détectant les malformations qu'elle occasionne dans la
structure de la double-hélice, puis la remplace par la base complémentaire à
celle du brin opposé\cite{krokan_base_2013}. Cette voie a été considérée comme
l'un des moteurs possibles du \ac{gbgc} chez les
mammifères\cite{duret_biased_2009}, mais a été rejetée chez la
levure\cite{lesecque_gc-biased_2013}. Chez cette dernière, le \ac{gbgc} est
associé aux régions converties les plus longues et dans lesquelles tous les
allèles sont convertis depuis le même haplotype donneur.

Est-ce que le \ac{ber}, s'il est actif chez les procaryotes, pourrait conduire à
de la conversion biaisée vers GC ? L'étude des régions converties chez des
mutants d'\emph{A. baylyi} délétés pour les fonctions clés de la correction des
mésappariemments par le \ac{ber} devrait permettre de répondre à cette question.

Ces cas de restaurations de l'haplotype sauvage ajoutent une incertitude sur la
nature des premiers marqueurs conservés (en rouge sur la
figure~\ref{fig:convtract}). En effet, ces marqueurs peuvent être la résultante
de deux choses : 1) ils ne font pas partie de l'hétéroduplex, et ne sont pas
soumis à de la conversion ou 2) ils font partie de l'hétéroduplex, mais l'allèle
sauvage est restauré. Dans tous les cas, la base séquencée correspond à l'allèle
sauvage du marqueur. Ces cas peuvent perturber la mesure de la longueur de
région convertie.

% ------------------------------------------------------------------------------
\subsection{Mutations \emph{de novo} et recombinaison homologue}
\label{subsec:discu-neomut}
% ------------------------------------------------------------------------------
\afterpage{\blankpage}

Nous avons observé 20 cas de mutations spontanées, qui sont dans
\SI{70}{\percent} des cas localisés dans la région convertie, et introduisent
des bases G et C. Ils peuvent être induits par deux phénomènes.

1) Ces mutations spontanées peuvent traduire des mutations survenues
\emph{ex-vivo}, au cours de la construction et/ou de l'amplification des
plasmides donneurs chez \emph{E. coli} (voir figure~\ref{fig:construct}).

2) Elles peuvent également traduire des mutations survenues \emph{in-vivo}, au
cours de l'intégration de l'ADN dans la cellule ou de la réparation par
recombinaison homologue du brin lésé. En effet, la recombinaison homologue a été
démontrée à plusieurs reprises comme
mutagène\cite{hicks_increased_2010,arbeithuber_crossovers_2015,malkova_mutations_2012},
les polymérases impliquées tendent à introduire plus d'erreurs que les
polymérases classiques\cite{pomerantz_dna_2013}. Cependant, à notre
connaissance, il n'a jamais été démontré que la mutation favorisait
l'introduction des bases C et G au cours de la recombinaison homologue. Au
contraire, la mutation est universellement biaisée vers l'introduction des bases
A et T, chez les eucaryotes\cite{lynch_rate_2010} comme chez les
procaryotes\cite{hildebrand_evidence_2010,hershberg_evidence_2010}. Est-ce que
le biais de conversion génique vers GC pourrait être influencé par un processus
mutationnel spécifique à la recombinaison homologue ?


% ------------------------------------------------------------------------------
\subsection{Comment augmenter la puissance du test ?}
\label{subsec:discu-puissance}
% ------------------------------------------------------------------------------

% DONE calculer la taille d'échantillon nécessaire pour observer un biais de
% longueur ou d'introduction d'un SNP. voir [schema/longueur_traces.R]
Pour détecter une différence de l'ordre de \num{30} paires de bases entre
longueurs de régions converties correspondant à l'introduction d'un marqueur
supplémentaire, en supposant qu'elles sont normalement distribuées, avec l'écart
type observé dans nos résultats (\num{227}), avec une puissance de \num{0.8} et
un niveau de confiance de \num{0.05}, il faudrait une taille d'échantillon de
\num{717} (test \textrm{t}) par groupe.

Pour détecter une différence significative entre le nombre de marqueurs où
l'allèle sauvage respectivement AT ou GC est restauré, en supposant que l'effet
observé soit moyennement fort ($\omega = $\num{0.5}), avec un niveau de
confiance de \num{0.05} et une puissance de \num{0.8}, il faudrait observer
\num{31} marqueurs. En présumant que les cas de restauration se produisent de
façon indépendante entre chaque clones, étant donné que nous avons observé des
cas de restaurations \SI{4}{\percent} des recombinants, il faudrait séquencer
\num{775} clones au total. Si l'effet est plus faible ($\omega = $\num{0.3}), il
faudrait observer \num{87} marqueurs, donc séquencer \num{2175} clones.

Cette taille d'échantillon n'est pas envisageable en utilisant les techniques
actuelles : le séquençage par la technique de Sanger implique un coût prohibitif
pour une telle taille d'échantillon. Nous avons étudié différents procédés
permettant de séquencer les régions converties à haut débit. Les séquençage à
haut débit permettrait de réduire les coûts de séquençage. Il nécessite
néanmoins de prendre en compte les facteurs suivants. 1) La taille des lectures
est de l'ordre de \(2 \times 300\) paires de bases par lecture, soit un total de
\(600\)pb, inférieure à la taille du locus considéré. 2) Lors de la
construction de la librairie de séquence, les amplicons sont mélangés les uns
aux autres. Il faut pouvoir les discriminer les uns des autres facilement, de
façon à individualiser les évènements de recombinaison. 3) La divergence très
faible des amplicons peut introduire des artéfacts lors de la capture des
clusters par la caméra de séquençage.


\afterpage{\blankpage}
