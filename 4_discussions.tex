\section{Discussions}
\label{sec:discussions}

% TODO parler des erreurs de séquençage, qu'on y a répondu par des manips
% parallèles.

% TODO discuter de ces cas de restaurations. comment elles ont pu arriver ?
% mécanismes, double crossing over ? Normalement ce qu'on voit chez la levure
% c'est un mécanisme global sur l'ensemble de la trace de conversion, pas
% localement. Lesecque et al excluent le BER comme possible mécanisme de bgc.

% TODO calculer la taille d'échantillon nécessaire pour observer un biais de
% longueur ou d'introduction d'un SNP.

% TODO parler des problèmes de second pic

% TODO parler des dinucléotides, de radman et cie.

% TODO parler du fait que le biais est généralement faible. Donner l'exemple des
% levures pour qui le biais est calculé.

% TODO count_last_snp. Il y a une sorte de balance. Peut être que ça
% dépend du premier SNP donneur ? Lesecque et al ;). Selon laurent ça dépend
% surtout d'un effet distance : plus on est loin du premier marqueur, plus on a
% de chance d'avoir basculé après le marqueur précédent. Il y a donc un effet de
% la distribution du dernier marqueur.

% TODO parler du fait que la distribution n'était pas attendu. référence aux
% mexicains.
% TODO placer figure de la distribution du point de recombinaison chez
% Rhizobium.

% TODO expliquer le point froid de recombinaison par le taux de GC ?
% TODO placer figure taux de gc dans rapport.

\afterpage{\blankpage}