% \afterpage{%
  % \blankpage
  \null
  \vfill

  % contient le schéma de la base secondaire et la figure obtenue par la
  % simulation.

  \begin{figure}[htbp]
    \centering


    \includegraphics[width = \textwidth]{img/gc_content.pdf}

    \caption[Taux de GC]{Taux de GC \\
      TODO légender}
    \label{fig:gctaux}
  \end{figure}

  \vfill
  \thispagestyle{empty}
  \addtocounter{page}{-1}
  \clearpage
  \newpage
% }

\section{Discussions}
\label{sec:discussions}

Nous avons séquencé un grand nombre de clones d'\emph{A. baylyi} recombinants à
un locus choisi de façon à introduire la correction des mésappariemments et à
comparer la fréquence des évènements de conversion conduisant à l'introduction
de bases G ou C par rapport à ceux introduisant des bases A ou T.

\subsection{Distribution des points de recombinaison}

\afterpage{\blankpage}

Le changement entre l'haplotype donneur et l'haplotype receveur se produit entre
le dernier marqueur converti et le premier marqueur conservé. Cette information
permet d'estimer la distribution du point de recombinaison et la longueur
moyenne des régions converties. Les évènements de recombinaison induits par les
constructions donneuses sont des évènements de type non-crossover : ils ne
conduisent pas à l'échange des régions flanquantes\cite{chen_gene_2007}. La
longueur moyenne des régions converties de \num{393} \(\pm\) \num{228}~pb est
comprise entre \num{100} et \num{1000}~bp et correspond ainsi à celle observée
pour les évènements non-crossover chez l'Homme
\cite{williams_non-crossover_2015}. La distribution du point de recombinaison
est assez différente de celle décrite par Yáñez-Cuna \emph{et
  al.}\cite{yanez-cuna_biased_2015} chez un modèle de \emph{Rhizobium etli}. Ils
observent que les recombinants changent d'haplotype pour la majorité dans les
\num{200} premières paires de bases, et seuls \SI{10}{\percent} des clones ont
une région convertie de plus de \num{800} pb. Les différences entre leurs
résultats et les nôtres peuvent potentiellement s'expliquer par les différences
de modèles et de locus.

Quelle que soit la construction donneuse, la distribution du point de
recombinaison suit globalement le même patron, les marqueurs entre \num{600} et
\num{800} pb et autour de \num{350} pb sont moins souvent les derniers marqueurs
convertis que les autres. Une hypothèse pour expliquer cette distribution serait
que le taux de GC local conditionne la probabilité de rejet du brin d'ADN
exogène. Un taux de GC plus faible diminue le nombre de liaisons hydrogène entre
les brins et pourrait donc diminuer les probabilités d'appariemment avec un brin
d'ADN exogène. En effet, lors de la recherche par la protéine RecA d'une matrice
homologue permettant de réparer la lésion, la reconnaissance est pûrement basée
sur l'appariemment Watson-Crick entre les brins\cite{lee_base_2015}. La
figure~\ref{fig:gctaux}a montre un point froid de recombinaison dans la région
entre \num{600} et \num{800} bp, environ \num{200} paires de bases après
l'origine de l'hétéroduplex. La figure~\ref{fig:gctaux}b montre l'évolution du
taux de GC par fenêtre de \num{50}~pb. Le point froid de recombinaison à
\num{600}~pb semble correspondre avec un taux de GC plus faible localement. En
fait, la zone pauvre en GC à \num{600} pb correspond au taux de GC moyen du
génome d'\emph{A.~baylyi} de \SI{40.3}{\percent}, mais le locus étudié a un taux
de GC moyen de \SI{46.8}{\percent}. Or ce locus est connu au laboratoire pour
permettre de bonnes efficacités de transformations. Si on suppose qu'un taux de
GC plus élevé entraîne une efficacité de recombinaison plus élevée, le taux de
GC à ce locus supérieur à celui du génome pourrait ainsi expliquer ces
efficacités de transformations. Cette supposition pourrait également expliquer
les corrélations observées par Lassalle \emph{et
  al.}\cite{lassalle_gc-content_2015} entre taux de GC et taux de recombinaison.

\subsection{Des preuves de l'existence d'une conversion génique biaisée vers GC
  chez \textit{A.~baylyi}}

Nous avons montré que la probabilité pour une base A ou T d'être convertie en G
ou C est significativement supérieure à celle d'une base G ou C d'être convertie
en A ou T. Cette observation correspond précisément aux prédictions du modèle
\ac{gbgc}. Or nous n'avons pas détecté de différences significatives de
longueurs de régions converties entre les clones transformés par des donneurs
différents. Quel pourrait-alors être le mécanisme responsable de ce biais ?

% % ------------------------------------------------------------------------------
% \subsubsection{Les restaurations ponctuelles de l'haplotype sauvage sont-elles
%   responsables du biais ?}
% \label{subsub:discu-restaur}
% % ------------------------------------------------------------------------------
\afterpage{\blankpage}

% DONE discuter de ces cas de restaurations. comment elles ont pu arriver ?
% mécanismes, double crossing over ? Normalement ce qu'on voit chez la levure
% c'est un mécanisme global sur l'ensemble de la trace de conversion, pas
% localement. Lesecque et al excluent le BER comme possible mécanisme de bgc.
Au sein des régions converties, 14 cas de restauration de l'allèle sauvage ont
été détectés. Quels mécanismes peuvent les expliquer ? Ce ne sont probablement
pas des mutations spontanées : bien que la recombinaison soit un processus
mutagène en soi\cite{rodgers_error-prone_2016,hicks_increased_2010}, les
restaurations sont retrouvées spécifiquement aux positions des marqueurs, et
correspondent précisément à l'allèle sauvage. Une mutation spontanée pourrait
introduire aléatoirement l'une des quatre bases. Chez les eucaryotes, des cas
complexes de conversion ont déjà été
décrits\cite{martini_genome-wide_2011,yeadon_recombination_2001}. Chez la
levure notamment, ces cas complexes ont été interprétés comme des basculements multiples
entre le choix des matrices permettant la réparation des
cassures\cite{hoffmann_trans_2005}. Au cours de la recombinaison méiotique, les
mésappariemments sont corrigés préférentiellement en utilisant la chromatide
homologue pour matrice, mais ces événèments montrent que des alternances
multiples entre le choix de la chromatide homologue et la chromatide sœur
peuvent avoir lieu. Cependant, dans notre système, la seule matrice envisageable
est le brin exogène. Il se pourrait également qu'une cellule en cours de
réplication utilise le chromosome néo-synthétisé comme matrice alternative,
mais, à notre connaissance, cette possibilité n'a encore jamais été documentée.

L'hypothèse la plus probable est que ces cas de restauration de l'allèle sauvage
soient dus à l'action de la machinerie de correction des mésappariemments de
type \ac{ber} (Base Excision Repair). Cette voie excise spécifiquement une base
mésappariée en détectant les malformations qu'elle occasionne dans la structure
de la double-hélice, puis la remplace par la base complémentaire à celle du brin
opposé\cite{krokan_base_2013}. Chez les eucaryotes, la correction des
mésappariemments dans les cellules en \emph{mitose} est effectuée en partie par
la voie du \ac{ber}, qui est biaisée vers l'introduction des bases G et C chez
les mammifères\cite{brown_specific_1987,brown_different_1988}. Ce biais est
probablement dû à une adaptation pour compenser l'hypermutabilité des cytosines
méthylées, dont la déamination, spontanée en conditions physiologiques normales,
conduit à l'introduction d'une thymine. Des enzymes dédiées, les thymines
DNA-glycosylases, excisent spécifiquement ces bases. La voie du \ac{ber} a donc
été considérée comme l'un des moteurs possibles du \ac{gbgc} chez les
mammifères\cite{duret_biased_2009}, mais a été rejetée chez la
levure\cite{lesecque_gc-biased_2013}. Chez cette dernière, le \ac{gbgc} est
associé aux régions converties les plus longues et dans lesquelles tous les
allèles sont convertis depuis le même haplotype donneur.

Le BER existe également chez les procaryotes. Lieb \emph{et
  al.}\cite{lieb_recombination_1985} ont montré chez \emph{E. coli} que les
enzymes du \ac{vsp} (Very Short Patch repair) répairaient un mésappariemment
dans un sens spécifique en fonction des bases au voisinage immédiat du
mésappariemment. Ce dernier point a également été démontré pour les enzymes du
\ac{ber} chez les eucaryotes\cite{brown_different_1988}. Cette activité de
réparation est spécifiquement liée au produit du gène
\emph{vsr}\cite{hennecke_vsr_1991,lieb_specific_1983}. \emph{A.~baylyi} possède
également certaines des enzymes impliquées dans les voies du
\ac{vsp}\cite{kanehisa_kegg_2016}. Si ce genre de mécanisme est à l'œuvre chez
\emph{A.~baylyi}, les marqueurs AT et les marqueurs GC de nos constructions, qui
sont adjacents, ne se situent pas rigoureusement dans le même contexte
nucléotidique (voir figure~\ref{fig:construct}). Pour palier à cet éventuel
biais, il faudrait transformer à nouveau un clone ayant converti tous les
marqueurs par la séquence sauvage : chaque évènement de conversion serait alors
rigoureusement dans le même contexte.

Est-ce que le \ac{ber}, s'il est actif chez \emph{A.~baylyi}, pourrait
conduire à de la conversion biaisée vers GC ? Nous avons trouvé un excès de
restauration des bases G et C par rapport aux cas de restauration des bases A et
T, respectivement de 10 contre 4 cas. Cet écart n'est pas significatif. Si le
rapport réel entre le nombre de cas de restaurations de l'allèle GC et le nombre
de cas de restaurations de l'allèle AT est le même, il faudrait doubler la
taille d'échantillon pour que le test soit significatif au seuil de \num{5e-2}.
L'étude des régions converties chez des mutants d'\emph{A. baylyi} délétés pour
les fonctions clés de la correction des mésappariemments par le \ac{ber} devrait
permettre de répondre formellement à la question de la contribution du BER au
biais de conversion. Néanmoins, le résultat obtenu avec \num{6818} marqueurs est
robuste à l'exclusion des marqueurs restaurés (probabilité critique \(=\)
\num{3e-2}). Ces cas de restaurations de l'allèle sauvage au sein de la région
convertie ne sont donc probablement pas le principal moteur du biais mesuré,
bien qu'ils puissent y contribuer faiblement.

Le \ac{ber} pourrait cependant contribuer de façon plus subtile à diminuer la
probabilité de transmission des bases A et T. En effet, les premiers marqueurs
conservés peuvent être la résultante de deux choses : 1) ils ne font pas partie
de l'hétéroduplex, et ne sont pas soumis à de la conversion ou 2) ils font
partie de l'hétéroduplex, mais l'allèle sauvage est restauré. Il n'est donc pas
exclu que l'hétéroduplex s'étende au delà du dernier marqueur converti et que le
\ac{ber} restaure l'allèle sauvage précisément sur ces marqueurs. Cependant,
telle qu'est conçue l'approche expérimentale, il est impossible de déterminer
précisément le degré d'extension de l'hétéroduplex, et donc la contribution
relative du \ac{ber} au nombre de marqueurs AT ou GC convertis.

Le principal contributeur au biais mesuré semble donc être le degré d'extension
de la région convertie. Bien que les comparaisons de longueur de région
converties ne soient significatives, une analyse globale montre bien une
sur-transmission des bases G et C. Ce résultat conforte l'extension du modèle
\ac{gbgc} aux procaryotes amorcée par \textit{Lassalle et
  al.}\cite{lassalle_gc-content_2015}.

% % ------------------------------------------------------------------------------
\subsection{Conséquences évolutives de la conversion biaisée vers GC}
\label{subsec:discu-conséquences}
% ------------------------------------------------------------------------------

L'écart entre la probabilité pour une base AT d'être convertie en GC
(\SI{54}{\percent}) et la probabilité pour une base GC d'être convertie en AT
(\SI{51}{\percent}) est faible. Un faible biais peut néanmoins avoir des
conséquences importantes. Chez la levure, un mésappariemment GC/AT a
\SI{50.62}{\percent} de chances d'être converti en GC, contre
\SI{49.38}{\percent} d'être converti en
AT\cite{mancera_high-resolution_2008,duret_biased_2009}. Mais de façon similaire
à la sélection, l'impact du gBGC sur l'évolution des génomes dépend de son
intensité par rapport à l'action de la dérive génétique. L'impact du gBGC à un
locus donné dépend uniquement de la fréquence à laquelle le locus est impliqué
dans un évènement de recombinaison, de la force du biais de conversion et de la
taille de la
population\cite{nagylaki_evolution_1983,lassalle_gc-content_2015,duret_biased_2009,duret_impact_2008}.
Ceci implique que même lorsque le biais est faible et la structure de population
très clonale, dès lors que la taille de population est grande, l'allèle G ou C a
une probabilité plus forte de se fixer. L'une des observations que ce modèle
explique bien est le taux de GC très bas des bactéries endosymbiotiques, pour
lesquelles le taux de recombinaison est effectivement
nul\cite{moran_genomics_2008}.

La conversion génique biaisée vers GC est l'un des candidats pour expliquer la
mystérieuse force évolutive qui permet aux procaryotes de maintenir de tels taux
de GC alors même que la mutation est universellement biaisée vers
AT\cite{hildebrand_evidence_2010,hershberg_evidence_2010}.

\afterpage{\blankpage}
