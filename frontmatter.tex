\rmfamily
\pagenumbering{roman}
\includepdf{img/titlepage.pdf}
% \blankpage
% \clearpage
% \subsubsection*{Crédits photographiques}
% Photographie de l'image de couverture reproduite avec permissions depuis
% Wikimedia Commons, domaine public, accès libre.\\
% \url{https://commons.wikimedia.org/wiki/File:Dark_side_of_the_moon.jpg}

\subsection*{Remerciements}

% \afterpage{\blankpage}

En premier lieu, je tiens à remercier messieurs Yvan Moënne-Loccoz et Manolo
Gouy, responsables respectifs de mes deux terrains de stage, qui rendent nos
stages possibles à grands coups de planning surchargés et de réunions sans fin.

Je remercie également l'ensemble de l'équipe pédagogique du Master Écosciences
et Microbiologie, qui a réussi à faire de ce master un lieu où il fait bon
apprendre, malgré les coupes budgétaires et contraintes administratives en tout
genre.

Viennent ensuite celles et ceux qui rendent notre vie à la paillasse beaucoup
plus simple : Mme Céline Lavire, qui donne beaucoup de son temps pour animer
cette équipe éparpillée aux quatres coins de la Doua ; M. Xavier Nesme ; Audrey
Dubost, qui prend soin de notre santé au Mendel ; Corinne Sannaire, qui prend
soin de la santé de nos paillasses et nous évite bien des heures à remplir des
boîtes de cônes ; David Chapuliot, qui prend soin de laisser ses consommables
bien en vue et nous évite ainsi d'avoir à utiliser les nôtres ; Raphaël Masse,
qui prend soin de laisser nos consommables pas très très bien en vue et s'évite
ainsi d'avoir à utiliser les siens ; Aurélie, qui prend soin de notre sens moral
et nous incite à passer très vite devant les rayons boucherie ; Thibault, qui ne
prend pas du tout soin de notre sens olfactif ; Rosa, qui n'accorde pas plus
d'attention à nos oreilles que Thibault à nos narines ; Quentin, qui prend soin
de nous, petits étudiants ; Yoann, qui prend soin d'eux, vieux thésards ; Jordan
Senior, qui accorde un soin tout particulier à être le moins drôle possible ;
tous les thésards qu'on a pu cotôyer, et enfin les M2, qui ont rendu ce stage
très agréable à force de croissants, miels et thés en tout genre.

Viennent après ceux qui rendent ma vie à moi beaucoup plus simple : ma Lu, qui
me supporte depuis bientôt trois ans et \emph{a fortiori} depuis janvier ; ma
grand-mère, à laquelle je dois beaucoup ; ma mère, qui me supporte depuis vingt
trois ans et sans qui je ne serais pas grand-chose ; mon père, que j'espère
rendre fier de là haut, tu m'as appris à ``ne pas me satisfaire de ne pas
comprendre le monde'', pour citer Dawkins.

Viennent enfin ceux qui ont rendu ma vie beaucoup plus dure : Vincent, que je
remercierai dans un style qu'il comprendra certainement et qui, alors que tout
le prédisposait à être impliqué dans ce projet à 100\%, n'a montré qu'un dédain
souverain et un manque d'investissement à la limite de la bienséance, manifeste
flagrant du peu d'intérêt qu'il porte à la science\footnote{Tout ceci n'est
  qu'un clin d'œil à la rhétorique de Vincent ou l'art de l'antiphrase, et n'est
  bien entendu pas vrai. Pas d'alarmes inutiles ! } ; Franck, pour ses conseils
et ses coups de main à la paillasse, ses relectures avisées, ses critiques
détaillées, perfectionnistes mais toujours constructives, et surtout sa
persévérance à mes côtés lorsque tout le poussait à rester chez lui pour se
soigner ; et enfin Laurent. Richard Dawkins dans ``Il Était une Fois Nos
Ancêtres'' disait à propos de John Maynard Smith :

\begin{quote}
  Il [...] écoute ce que les jeunes chercheurs ont à dire, les inspire, ravive
  les enthousiasmes qui pourraient vaciller, et les renvoit à leur laboratoire
  ou aux incertitudes de leur spécialité, revigorés, et impatients d'essayer les
  nouvelles idées qu'il a généreusement partagées avec eux.
\end{quote}

Comme encadrant, je n'ai pas eu J.M. Smith, mais avec vous Laurent, c'était ma
foi tout comme ; j'ai pris un grand plaisir à venir chauffer mes neurones à
votre bois.

J'espère avoir été à la hauteur de l'intérêt que vous avez porté tous les trois
à mon stage !


\newpage

\setstretch{1.0}

\printacronyms[include-classes=abbrev,name=Abbréviations]
\listoffigures
\listoftables

\newpage
\null
\vfill

\tableofcontents
\addtocontents{toc}{\protect\vfill}
\vfill

\blankpage
\blankpage
\clearpage

\pagenumbering{arabic}
\setcounter{page}{1}

\sffamily

\setstretch{1.5}