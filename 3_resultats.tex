
\addfig{%
  \begin{center}
  \rmfamily
  % \fontspec{Gill Sans}
  \setstretch{1.0}

  %% TODO indiquer l'origine de la région convertie.
  %% TODO décaler les flèches bleues et rouge pour qu'elles correspondent mieux.

  \tikzset{trace text/.style = {align = left, below right, font = \scriptsize}}
  \tikzset{trace legend/.style = {Black, opacity = 1, align = center, text width = 4.5cm,
      above, font = \scriptsize}} %
  \tikzset{trace fleche/.style={Gray, dotted, thick, opacity = 0.6}}
  % \tikzset{trace text/.style={black, text width = 7cm, font = \scriptsize, above}}

  \begin{tikzpicture}

    %% Trace de conversion des donneurs strong

    \node[trace text] at (0, 0) {%
      \textcolor{white}{cuicui}\\ % un titre était présent ici. doublon avec le titre en bas. remplacé
         % par un blank space pour ne pas avoir à tout décaler.
      \includegraphics[width = 0.65\textwidth]{img/trace_ws.pdf}};

    \coordinate (complexe strong) at (8, -6.3);
    \coordinate (complexe weak) at (6.7, -6.3);
    \coordinate (badqual) at (1.4, -14);
    \coordinate (tout bleu) at (12, -13.5);

    % \draw[opacity = 0.2, line width = 0.01pt, Gray] (0,0) grid (15, -15);
    % \draw[opacity = 0.2, line width = 0.005pt, step = 0.5, Gray] (0,0) grid (15, -15);

    \draw[trace fleche, gs_rec_col] [<-] (10.2, -1.0) -- (12, -1.0)
    node[trace legend] {\textcolor{gs_rec_col}{Génotype du receveur}}
    node {\(\bullet\)}
    ;

    \draw[trace fleche, gs_don_col] [<-] (10.2, -14.35) -- (12, -14.35)
    node[trace legend] {\textcolor{gs_don_col}{Génotype du donneur}}
    node {\(\bullet\)}
    ;

    \draw[trace fleche] (11.5, -5) %
    node{\(\bullet\)} %
    node[trace legend, darkgray] {Restauration de l'haplotype sauvage GC.}
    -- ++(-1, 0) -- (complexe strong);

    \draw[trace fleche]  (11.5, -8) %
    node{\(\bullet\)} %
    node[trace legend, darkgray] {Restauration de l'haplotype sauvage AT.}
    -- ++(-1, 0) -- (complexe weak);

    %% flèche sens du séquençage
    \draw[trace fleche, opacity = 0.4] [->] (5, -0.3)
    node[left, font=\scriptsize, opacity = 0.5] {Sens du séquençage}
    % node{\(\bullet\)}
    -- ++(2, 0)
    ;

    % \draw[opacity = 0.2, line width = 0.01pt, Gray] (0,0) grid (15, -19);
    % \draw[opacity = 0.2, line width = 0.005pt, step = 0.5, Gray] (0, 0) grid (15, -19);
    % \draw[]

    \begin{scope}[shift={(-2, 11.0)}]

      \draw[gs_rec_col,    very thick] (5,    -10.0) -- (7, -10.0);
      \draw[ancre_don_col, very thick] (11.4, -10.0) -- ++(-0.5, 0);
      \draw[solid,         am1_col,           very thick] (7.0, -10.0) -- ++(0.2, 0);
      \draw[solid,         kanr_col,          very thick] (7.2, -10.0) -- ++(2.0, 0);
      \draw[solid,         genome_col,        very thick] (9.2, -10.0) -- ++(0.2, 0);
      \draw[solid,         ancre_don_col,     very thick] (9.4, -10.0) -- ++(1.5, 0);
      \draw[genome_col,    thick, arrows = {-Stealth[left]}] (11.4, -10.0) -- ++(1,   0);
      \draw[genome_col,    thick] (2, -10.0) -- (5, -10.0);

      %% brin <-
      \draw[gs_rec_col, very thick] (5, -10.2)   -- (5.5, -10.2);
      \draw[gs_don_col, solid,          very thick] (6.0, -10.2) -- ++(1.0,  0);
      \draw[gs_rec_col, solid,          very thick] (6.0, -10.2) -- ++(-0.5, 0);
      \draw[solid,      am1_col,        very thick] (7.0, -10.2) -- ++(0.2,  0);
      \draw[solid,      kanr_col,       very thick] (7.2, -10.2) -- ++(2.0,  0) node[midway, below, font = \tiny] {KanR};
      \draw[solid,      genome_col,     very thick] (9.2, -10.2) -- ++(0.2,  0);
      % \draw[solid,      ancre_don_col,  very thick] (9.4, -10.2) --   (9.9,  -10.2);
      \draw[solid,      ancre_don_col,  very thick] (9.4, -10.2) -- ++(2.0,  0) node[midway, below, font = \tiny] {Ancre};
      % génome
      \draw[genome_col, thick] (11.4, -10.2) -- ++(1, 0);
      \draw[genome_col, thick, arrows = {Stealth[left]-}] (2, -10.2) -- (5, -10.2);

	    \draw[Gray, thick, densely dotted, fill = Gray, opacity = 0.3] (6, -10.4) rectangle (7, -9.8);
      \node[Gray, font=\tiny] at (6.5, -10.55) {Hétéroduplex};

      \end{scope}

      \draw[dotted, Gray, line width = 1pt] (1 , -0.8) -- ++(2, +1.5);
      \draw[dotted, Gray, line width = 1pt] (10, -0.8) -- ++(-5, +1.5);

    % \node[trace legend, align = justify] at (13.5, -13) {%
    %   Chaque ligne horizontale représente une séquence. Les points représentent
    %   les positions des marqueurs sur les séquences. L'intensité et le diamètre
    %   des points représentent le score de qualité du site. La couleur des points
    %   représente leur polarité. Ils sont bleus lorsque le site est dans la
    %   région convertie : ils correspondent à l'haplotype du donneur. Ils sont
    %   rouges lorsque le site est dans la région conservée.

    %   Les alternances rouge / bleu marquent la transition de l'haplotype
    %   converti à l'haplotype sauvage : le point de recombinaison est localisé
    %   entre ces deux marqueurs.

    %   Les séquences sont triées par longueur de région convertie. Certains
    %   transformants ont converti tous les marqueurs\tikz[overlay]{\draw[gray,
    %     dotted, opacity = 0.9] [->] (0, 0) -- ++(-4, 10) -- ++(-1, 0);}.
    %   Certains transformants ont conservé tous les
    %   marqueurs\tikz[overlay]{\draw[gray, dotted, opacity = 0.7] [->] (0, 0) -- (-2, -1) -- (-3, -1);}.
    %   %
    % };
    %% DONE ajuster les flèches indiquant les séquences convertissant

  \end{tikzpicture}
\end{center}
  \caption{\textbf{Traces de conversion d'un locus génomique
      d'\emph{Acinetobacter baylyi} par un gène synthétique donneur d'allèles GC
    et AT.}%
  \rmfamily
    % TODO rajouter légende figure trace de conversion. Que tirer du graphe ?
  }%
  \label{img:convtract}
}

\section{Résultats}
\label{sec:resultats}

\subsection{Analyses visuelles des zones de recombinaison}
\label{subsec:visu}

Nous avons transformé une supension d'\emph{Acinetobacter} par des constructions
dont l'intégration dans le génome par recombinaison homologue entraîne la
réparation des mésappariemments, un mécanisme qui est biaisé vers l'introduction
des bases GC chez les eucaryotes. Pour chaque construction, nous avons
représenté graphiquement les zones de recombinaison (voire figure
\ref{img:convtract}). L'origine du graphique correspond au début de la région
convertie. Les 96 séquences obtenues sont triées par longueur de région
convertie. La figure \ref{img:convtract} représente les zones de recombinaison
obtenues en transformant avec une construction alternant CG et AT. Les figures
en annexes
% TODO référencer les figures en annexes.
représentent les zones de recombinaison des 96 clones transformés par la
construction \(\nicefrac{AT}{GC}\), celles des clones transformés par la
construction GC et celles des clones transformés par la construction AT.


\subsection{Quantification du rapport en base}
\label{subsec:tauxgc}

\begin{figure}[ht]
  \centering
  \includegraphics[width = \textwidth]{img/gc_content.pdf}
  \caption{\label{fig:tauxgc} \textbf{Modification du rapport en base par la
      conversion génique} \\
    Chaque point représente le taux de GC d'une séquence recombinante. Chaque
    panneau du graphique indique le type de construction donneuse. Comme
    attendu, le taux de GC des recombinants transformés par la construction AT
    est inférieur à celui de la séquence receveuse. De même, lorsque la
    construction donneuse est GC, le taux de GC moyen est supérieur à celui du
    receveur.

    Cependant, lorsque la construction donneuse a le même taux de GC que le
    receveur (dans les constructions $\nicefrac{AT}{GC}$ et
    $\nicefrac{GC}{AT}$), le taux de GC moyen des recombinants n'est pas
    statistiquement différent de celui du receveur.
  }

\end{figure}
