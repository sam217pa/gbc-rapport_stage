% \blankpage
% \null
% \vfill
% \begin{center}
  \rmfamily
  % \fontspec{Gill Sans}
  \setstretch{1.0}

  %% TODO indiquer l'origine de la région convertie.
  %% TODO décaler les flèches bleues et rouge pour qu'elles correspondent mieux.

  \tikzset{trace text/.style = {align = left, below right, font = \scriptsize}}
  \tikzset{trace legend/.style = {Black, opacity = 1, align = center, text width = 4.5cm,
      above, font = \scriptsize}} %
  \tikzset{trace fleche/.style={Gray, dotted, thick, opacity = 0.6}}
  % \tikzset{trace text/.style={black, text width = 7cm, font = \scriptsize, above}}

  \begin{tikzpicture}

    %% Trace de conversion des donneurs strong

    \node[trace text] at (0, 0) {%
      \textcolor{white}{cuicui}\\ % un titre était présent ici. doublon avec le titre en bas. remplacé
         % par un blank space pour ne pas avoir à tout décaler.
      \includegraphics[width = 0.65\textwidth]{img/trace_ws.pdf}};

    \coordinate (complexe strong) at (8, -6.3);
    \coordinate (complexe weak) at (6.7, -6.3);
    \coordinate (badqual) at (1.4, -14);
    \coordinate (tout bleu) at (12, -13.5);

    % \draw[opacity = 0.2, line width = 0.01pt, Gray] (0,0) grid (15, -15);
    % \draw[opacity = 0.2, line width = 0.005pt, step = 0.5, Gray] (0,0) grid (15, -15);

    \draw[trace fleche, gs_rec_col] [<-] (10.2, -1.0) -- (12, -1.0)
    node[trace legend] {\textcolor{gs_rec_col}{Génotype du receveur}}
    node {\(\bullet\)}
    ;

    \draw[trace fleche, gs_don_col] [<-] (10.2, -14.35) -- (12, -14.35)
    node[trace legend] {\textcolor{gs_don_col}{Génotype du donneur}}
    node {\(\bullet\)}
    ;

    \draw[trace fleche] (11.5, -5) %
    node{\(\bullet\)} %
    node[trace legend, darkgray] {Restauration de l'haplotype sauvage GC.}
    -- ++(-1, 0) -- (complexe strong);

    \draw[trace fleche]  (11.5, -8) %
    node{\(\bullet\)} %
    node[trace legend, darkgray] {Restauration de l'haplotype sauvage AT.}
    -- ++(-1, 0) -- (complexe weak);

    %% flèche sens du séquençage
    \draw[trace fleche, opacity = 0.4] [->] (5, -0.3)
    node[left, font=\scriptsize, opacity = 0.5] {Sens du séquençage}
    % node{\(\bullet\)}
    -- ++(2, 0)
    ;

    % \draw[opacity = 0.2, line width = 0.01pt, Gray] (0,0) grid (15, -19);
    % \draw[opacity = 0.2, line width = 0.005pt, step = 0.5, Gray] (0, 0) grid (15, -19);
    % \draw[]

    \begin{scope}[shift={(-2, 11.0)}]

      \draw[gs_rec_col,    very thick] (5,    -10.0) -- (7, -10.0);
      \draw[ancre_don_col, very thick] (11.4, -10.0) -- ++(-0.5, 0);
      \draw[solid,         am1_col,           very thick] (7.0, -10.0) -- ++(0.2, 0);
      \draw[solid,         kanr_col,          very thick] (7.2, -10.0) -- ++(2.0, 0);
      \draw[solid,         genome_col,        very thick] (9.2, -10.0) -- ++(0.2, 0);
      \draw[solid,         ancre_don_col,     very thick] (9.4, -10.0) -- ++(1.5, 0);
      \draw[genome_col,    thick, arrows = {-Stealth[left]}] (11.4, -10.0) -- ++(1,   0);
      \draw[genome_col,    thick] (2, -10.0) -- (5, -10.0);

      %% brin <-
      \draw[gs_rec_col, very thick] (5, -10.2)   -- (5.5, -10.2);
      \draw[gs_don_col, solid,          very thick] (6.0, -10.2) -- ++(1.0,  0);
      \draw[gs_rec_col, solid,          very thick] (6.0, -10.2) -- ++(-0.5, 0);
      \draw[solid,      am1_col,        very thick] (7.0, -10.2) -- ++(0.2,  0);
      \draw[solid,      kanr_col,       very thick] (7.2, -10.2) -- ++(2.0,  0) node[midway, below, font = \tiny] {KanR};
      \draw[solid,      genome_col,     very thick] (9.2, -10.2) -- ++(0.2,  0);
      % \draw[solid,      ancre_don_col,  very thick] (9.4, -10.2) --   (9.9,  -10.2);
      \draw[solid,      ancre_don_col,  very thick] (9.4, -10.2) -- ++(2.0,  0) node[midway, below, font = \tiny] {Ancre};
      % génome
      \draw[genome_col, thick] (11.4, -10.2) -- ++(1, 0);
      \draw[genome_col, thick, arrows = {Stealth[left]-}] (2, -10.2) -- (5, -10.2);

	    \draw[Gray, thick, densely dotted, fill = Gray, opacity = 0.3] (6, -10.4) rectangle (7, -9.8);
      \node[Gray, font=\tiny] at (6.5, -10.55) {Hétéroduplex};

      \end{scope}

      \draw[dotted, Gray, line width = 1pt] (1 , -0.8) -- ++(2, +1.5);
      \draw[dotted, Gray, line width = 1pt] (10, -0.8) -- ++(-5, +1.5);

    % \node[trace legend, align = justify] at (13.5, -13) {%
    %   Chaque ligne horizontale représente une séquence. Les points représentent
    %   les positions des marqueurs sur les séquences. L'intensité et le diamètre
    %   des points représentent le score de qualité du site. La couleur des points
    %   représente leur polarité. Ils sont bleus lorsque le site est dans la
    %   région convertie : ils correspondent à l'haplotype du donneur. Ils sont
    %   rouges lorsque le site est dans la région conservée.

    %   Les alternances rouge / bleu marquent la transition de l'haplotype
    %   converti à l'haplotype sauvage : le point de recombinaison est localisé
    %   entre ces deux marqueurs.

    %   Les séquences sont triées par longueur de région convertie. Certains
    %   transformants ont converti tous les marqueurs\tikz[overlay]{\draw[gray,
    %     dotted, opacity = 0.9] [->] (0, 0) -- ++(-4, 10) -- ++(-1, 0);}.
    %   Certains transformants ont conservé tous les
    %   marqueurs\tikz[overlay]{\draw[gray, dotted, opacity = 0.7] [->] (0, 0) -- (-2, -1) -- (-3, -1);}.
    %   %
    % };
    %% DONE ajuster les flèches indiquant les séquences convertissant

  \end{tikzpicture}
\end{center}
% \caption[Zones de recombinaison détaillée]{\textbf{Zones de recombinaison entre
%     un locus génomique d'\emph{Acinetobacter baylyi} et un gène synthétique
%     donneur d'allèles CG et AT.} \\ %
%   \rmfamily Chaque ligne horizontale représente une séquence. Les points
%   représentent les positions des marqueurs sur les séquences. L'intensité et la
%   taille des points représentent le score de qualité de séquençage du site. Les
%   points sont bleus lorsque le site est dans la région convertie : ils
%   correspondent à l'haplotype du donneur. Ils sont rouges lorsque le site
%   correspond à l'haplotype du donneur. La première alternance rouge~/~bleu en
%   partant de l'extrémité 5' marque la transition de l'haplotype converti à
%   l'haplotype receveur : le point de recombinaison est localisé entre ces deux
%   marqueurs. Certaines séquences sont des cas de conversion complexes, avec une
%   ou plusieurs alternances donneur~/~receveur (rouge~/~bleu). Ce sont des cas de
%   conversion accompagnés de restaurations de l'haplotype du receveur. Les
%   séquences ont été triées par longueur de région convertie.}%
% \vfill
% \label{fig:convtract}
% \thispagestyle{empty}
% \addtocounter{page}{-1}
% % \afterpage{\blankpage}
% % \newpage
\

%% fréquences de transformation et distirbutino des points de bascule
\null
\vfill
\begin{table}[h!]
  \rmfamily
  \centering

  \caption[Fréquences de transformation]{\textbf{Fréquences de transformation}}
  \label{tab:transfo-freq}
  \vspace{0.5cm}

  \begin{tabular}{cc}
    \toprule
    \thead{\normalsize Construction \\ \normalsize Donneuse} & \thead{\normalsize Fréquences de \\ \normalsize transformation} \\
    \midrule
    CG    & \num{6.53e-5} \\
    AT    & \num{2.42e-5} \\
    AT/CG & \num{4.97e-5} \\
    CG/AT & \num{1.74e-4} \\
    \bottomrule
  \end{tabular}
\end{table}

\vfill

% TODO placer figures pics secondaires là.
\begin{center}

  \begin{figure}

    \centering
    \tikzset{legende fleche/.style={Gray, dotted, thick, opacity = 0.6}}
    \rmfamily
    \scriptsize

    \begin{tikzpicture}[scale = 0.5]
      \node[above] at (0, 0) {\includegraphics[scale = 0.5]{img/pic_second.png}};
      % \draw[opacity = 0.2, line width = 1pt, Gray] (-15,0) grid (15, 15);
      % \draw[opacity = 0.2, line width = 0.005pt, step = 0.5, Gray] (-15,0) grid (15, +15);

      \draw[legende fleche] [->] (-6, 0)        node[ above] {Score de qualité} -- ++(4.5, 0) -- ++(1, 1);
      \draw[legende fleche, Black] [->] (-6, 7) node[ above] {Base G} -- ++(4.5, 0) -- ++(1, -1);
      \draw[legende fleche, Red3] [->] (-6, 11) node[ above] {Base T} -- ++(4.5, 0) -- ++(1, -1);

    \end{tikzpicture}

    \caption[Marqueur montrant des traces de contaminations]{%
      \textbf{Exemple de marqueur montrant des traces de
        contaminations} \\ \rmfamily Cet électrophérogramme montre les bases autour du
      marqueur à la position 200. Dans une région de qualité moyenne élevée (bases en
      5' et en 3'), le marqueur présente une trace de contamination par une autre
      base. La base déterminée est la base T mais une base G est présente dans la
      population d'amplicon séquencée.
      \label{fig:pic-second}
    }
  \end{figure}

  \begin{figure}[h!]

    \centering

    \tikzset{legende fleche/.style={Gray, thick}}
    \begin{tikzpicture}[scale = 0.3]
      \scriptsize
      \foreach \x in {1,2,...,12} {
        \foreach \y in {1,2,...,8} {
        \draw (\x, \y) circle (0.3cm) ;
      }}
    \draw[fill = Red] (8, 5) circle (0.3cm);
    \draw[fill = Red] (7, 6) circle (0.3cm);
    \draw[fill = Red] (7, 4) circle (0.3cm);
    \draw[fill = Red] (6, 5) circle (0.3cm);

    \draw[Gray, densely dotted] (5.5, 3.5) rectangle (8.5, 6.5);
    \draw[Gray, densely dotted] (9.5, 0.6) rectangle (12.4, 3.5);
    \draw[Gray] (0.5, 0.5) rectangle (12.5, 8.5);

    \draw[legende fleche] [<-] (7, 5) -- ++(1, 0.5) -- ++(5, 0)
    node[right, Black] {\(\bar{n} = \frac{4}{8} = \frac{1}{2}\)}
    ;

    \draw[legende fleche] [<-] (11, 2) -- ++(1, 2.0) -- ++(1, 0)
    node[right, Black] {\(\bar{n} = \frac{0}{8} = 0\)}
    ;

    \node[right] at (12.5, 8) {\(\bar{X} = 4\)};
    \node[right] at (12.5, 1) {\(\bar{N} = \sum_{i = 1}^{96}\frac{\bar{n_i}}{96}\)};

    \draw [->] (18.5, 4.5) -- ++(5, 0) node[midway, above] {\(\times 10000\)}
    node[right] {
      \includegraphics[width = 0.5\textwidth]{img/randomweak.png}
    };
    \end{tikzpicture}


    \caption[Des contaminations dues au hasard ?]{%
      \label{fig:simul-count}\textbf{Des contaminations dues au hasard ? } \\
      \rmfamily Par plaque de 96 puits, nous avons déterminé \(X\) le nombre de
      puits dont la séquence montre des traces de pics secondaires (voir
      figure~\ref{fig:pic-second}) et mesuré \(\bar{n}\) la moyenne du nombre de
      puits voisins contaminés. \(\bar{N}\) est la moyenne des 96 \(\bar{n}\)
      obtenus par plaque. Nous avons simulé \num{1e4} plaques avec \(X\) puits
      contaminés répartis aléatoirement, mesuré \(\bar{N}\) et comparé la valeur
      expérimentale de \(\bar{N}\) (trait vertical rouge) avec la distribution
      des \num{1e4} \(\bar{N}\) (en gris) Seules \(78 / 10000\) plaques simulées
      montrent un \(\bar{N}\) supérieur à la valeur expérimentale : la
      répartition des séquences contaminées dans les plaques ne peut pas être
      attribuée au hasard.
      %
    }
    % TODO légender figure pics secondaire simul
  \end{figure}

\end{center}

\vfill
\thispagestyle{empty}
\addtocounter{page}{-1}
\clearpage
\newpage


% ==============================================================================
\section{Résultats}
\label{sec:resultats}
% ==============================================================================

\afterpage{%
  \null
  \vfill


  \begin{center}
  \rmfamily
  % \fontspec{Gill Sans}
  \setstretch{1.0}

  %% TODO indiquer l'origine de la région convertie.
  %% TODO décaler les flèches bleues et rouge pour qu'elles correspondent mieux.

  \tikzset{trace text/.style = {align = left, below right, font = \scriptsize}}
  \tikzset{trace legend/.style = {Black, opacity = 1, align = center, text width = 4.5cm,
      above, font = \scriptsize}} %
  \tikzset{trace fleche/.style={Gray, dotted, thick, opacity = 0.6}}
  % \tikzset{trace text/.style={black, text width = 7cm, font = \scriptsize, above}}

  \begin{tikzpicture}

    %% Trace de conversion des donneurs strong

    \node[trace text] at (0, 0) {%
      \textcolor{white}{cuicui}\\ % un titre était présent ici. doublon avec le titre en bas. remplacé
         % par un blank space pour ne pas avoir à tout décaler.
      \includegraphics[width = 0.65\textwidth]{img/trace_ws.pdf}};

    \coordinate (complexe strong) at (8, -6.3);
    \coordinate (complexe weak) at (6.7, -6.3);
    \coordinate (badqual) at (1.4, -14);
    \coordinate (tout bleu) at (12, -13.5);

    % \draw[opacity = 0.2, line width = 0.01pt, Gray] (0,0) grid (15, -15);
    % \draw[opacity = 0.2, line width = 0.005pt, step = 0.5, Gray] (0,0) grid (15, -15);

    \draw[trace fleche, gs_rec_col] [<-] (10.2, -1.0) -- (12, -1.0)
    node[trace legend] {\textcolor{gs_rec_col}{Génotype du receveur}}
    node {\(\bullet\)}
    ;

    \draw[trace fleche, gs_don_col] [<-] (10.2, -14.35) -- (12, -14.35)
    node[trace legend] {\textcolor{gs_don_col}{Génotype du donneur}}
    node {\(\bullet\)}
    ;

    \draw[trace fleche] (11.5, -5) %
    node{\(\bullet\)} %
    node[trace legend, darkgray] {Restauration de l'haplotype sauvage GC.}
    -- ++(-1, 0) -- (complexe strong);

    \draw[trace fleche]  (11.5, -8) %
    node{\(\bullet\)} %
    node[trace legend, darkgray] {Restauration de l'haplotype sauvage AT.}
    -- ++(-1, 0) -- (complexe weak);

    %% flèche sens du séquençage
    \draw[trace fleche, opacity = 0.4] [->] (5, -0.3)
    node[left, font=\scriptsize, opacity = 0.5] {Sens du séquençage}
    % node{\(\bullet\)}
    -- ++(2, 0)
    ;

    % \draw[opacity = 0.2, line width = 0.01pt, Gray] (0,0) grid (15, -19);
    % \draw[opacity = 0.2, line width = 0.005pt, step = 0.5, Gray] (0, 0) grid (15, -19);
    % \draw[]

    \begin{scope}[shift={(-2, 11.0)}]

      \draw[gs_rec_col,    very thick] (5,    -10.0) -- (7, -10.0);
      \draw[ancre_don_col, very thick] (11.4, -10.0) -- ++(-0.5, 0);
      \draw[solid,         am1_col,           very thick] (7.0, -10.0) -- ++(0.2, 0);
      \draw[solid,         kanr_col,          very thick] (7.2, -10.0) -- ++(2.0, 0);
      \draw[solid,         genome_col,        very thick] (9.2, -10.0) -- ++(0.2, 0);
      \draw[solid,         ancre_don_col,     very thick] (9.4, -10.0) -- ++(1.5, 0);
      \draw[genome_col,    thick, arrows = {-Stealth[left]}] (11.4, -10.0) -- ++(1,   0);
      \draw[genome_col,    thick] (2, -10.0) -- (5, -10.0);

      %% brin <-
      \draw[gs_rec_col, very thick] (5, -10.2)   -- (5.5, -10.2);
      \draw[gs_don_col, solid,          very thick] (6.0, -10.2) -- ++(1.0,  0);
      \draw[gs_rec_col, solid,          very thick] (6.0, -10.2) -- ++(-0.5, 0);
      \draw[solid,      am1_col,        very thick] (7.0, -10.2) -- ++(0.2,  0);
      \draw[solid,      kanr_col,       very thick] (7.2, -10.2) -- ++(2.0,  0) node[midway, below, font = \tiny] {KanR};
      \draw[solid,      genome_col,     very thick] (9.2, -10.2) -- ++(0.2,  0);
      % \draw[solid,      ancre_don_col,  very thick] (9.4, -10.2) --   (9.9,  -10.2);
      \draw[solid,      ancre_don_col,  very thick] (9.4, -10.2) -- ++(2.0,  0) node[midway, below, font = \tiny] {Ancre};
      % génome
      \draw[genome_col, thick] (11.4, -10.2) -- ++(1, 0);
      \draw[genome_col, thick, arrows = {Stealth[left]-}] (2, -10.2) -- (5, -10.2);

	    \draw[Gray, thick, densely dotted, fill = Gray, opacity = 0.3] (6, -10.4) rectangle (7, -9.8);
      \node[Gray, font=\tiny] at (6.5, -10.55) {Hétéroduplex};

      \end{scope}

      \draw[dotted, Gray, line width = 1pt] (1 , -0.8) -- ++(2, +1.5);
      \draw[dotted, Gray, line width = 1pt] (10, -0.8) -- ++(-5, +1.5);

    % \node[trace legend, align = justify] at (13.5, -13) {%
    %   Chaque ligne horizontale représente une séquence. Les points représentent
    %   les positions des marqueurs sur les séquences. L'intensité et le diamètre
    %   des points représentent le score de qualité du site. La couleur des points
    %   représente leur polarité. Ils sont bleus lorsque le site est dans la
    %   région convertie : ils correspondent à l'haplotype du donneur. Ils sont
    %   rouges lorsque le site est dans la région conservée.

    %   Les alternances rouge / bleu marquent la transition de l'haplotype
    %   converti à l'haplotype sauvage : le point de recombinaison est localisé
    %   entre ces deux marqueurs.

    %   Les séquences sont triées par longueur de région convertie. Certains
    %   transformants ont converti tous les marqueurs\tikz[overlay]{\draw[gray,
    %     dotted, opacity = 0.9] [->] (0, 0) -- ++(-4, 10) -- ++(-1, 0);}.
    %   Certains transformants ont conservé tous les
    %   marqueurs\tikz[overlay]{\draw[gray, dotted, opacity = 0.7] [->] (0, 0) -- (-2, -1) -- (-3, -1);}.
    %   %
    % };
    %% DONE ajuster les flèches indiquant les séquences convertissant

  \end{tikzpicture}
\end{center}
  \caption[Zones de recombinaison détaillée]{\textbf{Zones de recombinaison entre
      un locus génomique d'\emph{Acinetobacter baylyi} et un gène synthétique
      donneur d'allèles CG et AT.} \\ %
    \rmfamily Chaque ligne horizontale représente une séquence. Les points
    représentent les positions des marqueurs sur les séquences. L'intensité et la
    taille des points représentent le score de qualité de séquençage du site. Les
    points sont bleus lorsque le site est dans la région convertie : ils
    correspondent à l'haplotype du donneur. Ils sont rouges lorsque le site
    correspond à l'haplotype du donneur. La première alternance rouge~/~bleu en
    partant de l'extrémité 5' marque la transition de l'haplotype converti à
    l'haplotype receveur : le point de recombinaison est localisé entre ces deux
    marqueurs. Certaines séquences sont des cas de conversion complexes, avec une
    ou plusieurs alternances donneur~/~receveur (rouge~/~bleu). Ce sont des cas de
    conversion accompagnés de restaurations de l'haplotype du receveur. Les
    séquences ont été triées par longueur de région convertie.}%
  \label{fig:convtract}
  \vfill
  \thispagestyle{empty}
  \addtocounter{page}{-1}
  \clearpage
  \newpage
}

\subsection{Validation de la démarche expérimentale}
\label{sub:result-proto}

Nous avons transformé une suspension d'\emph{Acinetobacter} par des constructions
dont l'intégration dans le génome par recombinaison homologue entraîne la
réparation des mésappariemments.

Les fréquences de transformations obtenues, représentées dans le tableau
\ref{tab:transfo-freq} ci-contre, sont de l'ordre de \num{1e-5} et ont permis
d'obtenir un grand nombre de recombinants, avec une moyenne de \num{9.2e+02}
transformants par boîte étalée, soit près de \num{3000} clones par
transformation. Ce résultat indique que le protocole expérimental est
suffisamment au point pour permetter d'obtenir le nombre de recombinants
souhaité. Pour chaque gène de synthèse donneur, \num{96} clones ont été
séquencés, soit un total de \num{384} recombinants.

\subsection{Validation de la qualité des séquences}
\label{sub:result-qualite}

Deux séquences trop courtes pour être alignées ont été exclues de l'analyse.
Neuf autres séquences présentaient une qualité médianne significativement
inférieure à celle des autres séquences et ont également été exclues. Au total,
\num{8561} positions variantes entre le donneur et le receveur ont été
analysées. Globalement, la qualité des \num{873} séquences suit un profil
attendu : les bases en début et fin de lecture sont d'une qualité inférieure à
la qualité médianne de \num{51}.

Cependant, \num{9.8}\% des marqueurs sont d'une qualité de séquençage inférieure
à la qualité de la région environnante. C'est la marque de pics secondaires sur
les électrophérogrammes (voir figure~\ref{fig:pic-second}). Des pics secondaires
apparaissent quand la population d'amplicon séquencée n'est pas homogène au site
considéré. Or les deux pics présents à un marqueur donné correspondent toujours
soit à la base sauvage, soit à la base synthétique introduite. Cette donnée peut
être interprétée de deux façons.

1) Cette hétérogénéité pourrait résulter d'un signal biologique authentique : si
les mésappariemments de l'hétéroduplex ne sont pas réparés, la division de la
cellule mère conduirait à une colonie hétérogène au site considéré et donc à la
ségrégation simultanée des deux allèles. Pour y répondre, nous avons séquencé à
nouveau 31 isolats issus d'un clone séquencé en premier lieu et qui montrait des
pics secondaires. Aucun des sous-clones séquencé ne montrent l'allèle
correspondant au pic secondaire (voir figure~\ref{fig:confirm-haplotype}). De
plus, nous avons adapté le protocole expérimental de façon à limiter
l'apparition de ces artéfacts. Nous avons supposé que des réminiscences de
construction donneuse dans la suspension étalée aurait pu conduire à plusieurs
évènements de transformation consécutifs et donc à ces profils hétérogènes. Les
suspensions ont donc été traitées par de la DNAse, de façon à dégrader les
plasmides résiduels. Enfin, pour s'affranchir de l'analyse de colonies
hétérogènes, nous avons isolé chacun des transformants, et analysé un clone
unique et purifié du transformant initial. La persistance de ces pics
secondaires après adaptation du protocole suggère qu'ils ne reflètent pas le
signal biologique envisagé.

2) S'il s'agit alors d'un artéfact méthodologique, la population d'amplicon
séquencée est hétérogène à la suite de contaminations entre les puits des
plaques, qui ont pu avoir lieu au cours du séquençage ou au cours des PCRs. Dans
ce cas, la répartition dans la plaque des puits dont la séquence montre des
traces de contamination ne devrait pas être déterminée seulement par le hasard.
Nous avons montré par simulation qu'elles ne l'étaient pas (voir
figure~\ref{fig:simul-count}). Les séquences montrant des pics secondaires sont
plus souvent voisines avec une autre séquence affectée que si elles étaient
réparties aléatoirement. En conséquence, nous avons décidé de n'étudier que les
marqueurs dont la qualité est supérieure à 40 : cela correspond à la probabilité
\num{1e-4} d'étudier une base déterminée à tort, et discrimine clairement les
bases litigieuses des autres. Nous avons donc obtenu un ensemble de \num{6818}
marqueurs de conversion, soit \SI{79.6}{\percent} de l'ensemble des marqueurs
obtenus.

% TODO mettre en page pour que les figures se retrouvent en face du texte.

\subsection{Description des régions recombinantes}
\label{subsec:result-descript}

\afterpage{%
  \null
  \vfill

  \begin{figure}
    \includegraphics[scale = 0.9]{img/distr_rcb_pt.pdf}
    \caption[Distribution de la position du dernier marqueur]{%
      \textbf{Distribution de la position du dernier marqueur converti} \\
      \rmfamily Ce graphique représente en ordonnées le nombre de transformants
      dont le dernier marqueur converti est à la position représentée en
      abscisse. Les panneaux du graphique représentent les quatres constructions
      donneuses. La position du dernier marqueur converti indique la position du
      point de recombinaison. Les transformants qui ne montrent aucun marqueur
      converti sont indiqués par des $N$.%
    }
    \label{fig:distrircb}
  \end{figure}

  \vfill
  \thispagestyle{empty}
  \addtocounter{page}{-1}
  \clearpage
  \newpage
}

La figure \ref{fig:convtract} représente le détail des zones de recombinaison
obtenues avec une construction donneuse alternant CG et AT. Les zones de
recombinaisons des clones transformés par les donneurs CG, AT et AT/CG sont
détaillées en annexe \ref{subsec:trac-de-conv}. En moyenne, à notre locus cible,
\num{393} \(\pm\) \num{228} nucléotides ont été transférés et intégrés dans le
génome. La position du dernier marqueur converti indique que le point de
recombinaison (voir figure~\ref{fig:recomb} et~\ref{fig:convtract}) se situe
entre les 30 bases le séparant du premier marqueur conservé. Nous avons donc
estimé la position du point de recombinaison par celle du dernier marqueur
converti, dont la distribution est représentée dans la
figure~\ref{fig:distrircb}. Les points de recombinaison se répartissent de façon
assez uniforme tout le long de la séquence. La majorité des recombinants
montrent des cas de conversion simple, avec une seule alternance entre
l'haplotype du donneur et celui du receveur. Cependant, 12 recombinants montrent
des cas de conversion complexes, avec au moins deux alternances entre
l'haplotype donneur et receveur. Ces alternances sont des restaurations de
l'haplotype receveur dans l'hétéroduplex.

Nous nous sommes intéressés à deux paramètres permettant d'estimer les
fréquences de conversion en faveur de GC chez \emph{A. baylyi} : la position du
point de recombinaison, et la correction ponctuelle des mésappariemments.

% tables de comptage des cas de restaurations
\afterpage{
  \null
  \vfill
  \begin{table}[h!]
    \rmfamily
    \centering

    \caption[Dénombrements des derniers marqueurs avant le point de
    recombinaison]{\textbf{Dénombrements des derniers marqueurs avant le point de
        recombinaison.}
    }
    \label{tab:doublets}

    % \begin{tabular}{@{}ccc@{}}
    %   \toprule
    %   & \multicolumn{2}{c}{Dernier marqueur converti} \\
    %   \cmidrule(r){2-3}
    %   \thead{\normalsize ADN synthétique \\ \normalsize donneur } & AT & CG \\
    %   \midrule
    %   AT/CG & 55 & 33 \\
    %   CG/AT & 37 & 50 \\
    %   \midrule[0.1pt]
    %   Total & 92 & 83 \\
    %   \bottomrule


    \end{tabular}

  \end{table}

  \vfill

  \begin{table}[h!]
    \centering
    \rmfamily

    \caption[Dénombrement des cas de restauration]{\textbf{Dénombrement des cas de
        restauration} \\
      \rmfamily       %
    }
    \label{tab:restaur}

    \begin{tabular}{@{}ccc@{}}
      \toprule
      & \multicolumn{2}{c}{Nucléotide Restauré} \\
      \cmidrule(r){2-3}
      \thead{\normalsize ADN synthétique \\ \normalsize donneur } & AT & CG \\
      \midrule
      AT    & - & 4 \\
      CG    & 0 & - \\
      AT/CG & 3 & 2 \\
      CG/AT & 1 & 4 \\
      \midrule[0.1pt]
      Total & 4 & 10 \\
      \bottomrule

    \end{tabular}
  \end{table}
  \vfill
  \thispagestyle{empty}
  \addtocounter{page}{-1}
  \clearpage
  \newpage
}

% ------------------------------------------------------------------------------
\subsection{Comparaison de la longueur des régions converties}
\label{subsec:longueur}
% ------------------------------------------------------------------------------

Selon l'hypothèse \ac{gbgc}, la région convertie devrait être plus longue
lorsque la construction donneuse induit des réparations vers CG que lorsqu'elle
induit des réparations vers AT. La différence entre la longueur moyenne de
région convertie par les donneurs respectivement CG et AT n'est pas
significative (test de Wilcoxon, probabilité critique~\(=\) \num{0.31}) (voir
figure \ref{fig:distrircb}). De la même façon, la différence entre la longueur
de région convertie par le donneur AT/CG et celle par le donneur CG/AT n'est pas
significative (test de Wilcoxon, probabilité critique~\(=\) \num{0.22}).
Globalement, le type de construction donneuse n'explique pas la variabilité de
la longueur de région convertie (test de Kruskal-Wallis, probabilité
critique~\(=\) \num{0.10}). Le type de donneur n'a pas d'influence sur la
longueur de région convertie.

\subsection{Distribution du dernier marqueur converti}
\label{subsec:distribution-points}

% Les constructions alternant AT et CG permettent de déterminer si le point de
% recombinaison se situe plus souvent après un marqueur introduisant une
% conversion vers CG qu'après un marqueur introduisant une conversion vers AT
% (voir table \ref{tab:doublets}). Le dernier marqueur converti est AT dans 92
% transformants ; il est CG dans 83 transformants. Cet écart n'est pas significatif
% (test du \(\chi^2\) d'homogénéité, probabilité critique~\(=\)~\num{0.49}) (voir
% table \ref{tab:doublets}).

Les constructions alternant AT et CG permettent de déterminer si davantage de
bases GC ont été introduites que de bases AT. Sur l'ensemble des marqueurs des
recombinants transformés par pAT/CG et pCG/AT, \num{720} bases C ou G ont été
introduites contre \num{677} bases A ou T. Cet écart n'est pas significatif
(test du $\chi^2$, probabilité critique~$=$~\num{0.25}).

% ------------------------------------------------------------------------------
\subsection{Restaurations de l'haplotype sauvage}
\label{subsec:restaur}
% ------------------------------------------------------------------------------

Certains transformants montrent des régions de conversions qui alternent entre
l'allèle sauvage receveur et l'allèle donneur. Ces alternances ponctuelles
affectent de 1 à 3 marqueurs consécutifs (voir figure \ref{fig:convtract} et
figure~\ref{fig:trace-4} en annexe~\ref{subsec:trac-de-conv}
page \pageref{fig:trace-4}). Nous avons confirmé qu'il s'agissait bien de
restaurations de l'allèle sauvage de deux façons. 1) Expérimentalement, nous
avons séquencé une sous-population de \num{31} clones issus d'un isolat séquencé en
premier lieu. Tous montrent la même alternance au même site (voir
figure~\ref{fig:confirm-haplotype} en annexe~\ref{subsec:confirm-haplotype}). 2)
Analytiquement, le score de qualité moyen des sites montrant des restaurations
de l'allèle sauvage permet de s'affranchir d'une possible erreur de séquençage :
celle-ci se traduit généralement par un indice de qualité plus faible au site
concerné. Le score de qualité moyen est de \num{49.36} aux sites restaurés,
contre \num{52.79} aux sites non-restaurés. La différence entre les deux n'est
pas significative (test de Wilcoxon, probabilité critique \(=\) \num{0.94}). Les
marqueurs correspondant à des restaurations de l'haplotype sauvage ne sont donc
pas des erreurs de séquençage, et correspondent à un signal biologique.

% ------------------------------------------------------------------------------
\subsection{Mutations \emph{de novo}}
\label{subsec:result-neomutat}
% ------------------------------------------------------------------------------

Nous avons détecté \num{20} cas de mutations \emph{de novo}. Elles ne sont pas
associés aux sites variants et sont d'une qualité moyenne supérieure à \num{41}.
Parmi ces \num{20} cas, \num{14} introduisent une base G ou C et 6 introduisent
une base A ou T. L'écart n'est pas statistiquement significatif (test du
$\chi^2$, probabilité critique \(=\) \num{0.07}). Les \num{14} cas de mutations
vers G ou C sont tous dans la région convertie, alors que les \num{6} cas de
mutations vers A ou T sont tous dans la région conservée.

% ------------------------------------------------------------------------------
\subsection{Estimation des fréquences de conversion en faveur de GC}
\label{subsec:result-freq}
% ------------------------------------------------------------------------------

\begin{table}[h!]
  % TODO placer tableau au bon endroit.
  % TODO légender tableau
  \centering
  \rmfamily
  \caption[Vivamus id enim.]{Fusce commodo. Lorem ipsum dolor sit amet, consectetuer adipiscing
    elit. Nulla facilisis, risus a rhoncus fermentum, tellus tellus lacinia
    purus, et dictum nunc justo sit amet elit. Nunc porta vulputate tellus. }
  \label{tab:comptages}

  \begin{tabular}{ccccc|c}
    \toprule
                            & dGC  & dAT  & dAT/GC & dGC/AT & \textbf{Total} \\
    \midrule
    $N_{GC}$                & -    & 1782 & 819    & 834    & 3435           \\
    $N_{GC \rightarrow AT}$ & -    & 1032 & 330    & 412    & 1774           \\
    $N_{AT}$                & 1673 & -    & 890    & 820    & 3383           \\
    $N_{AT \rightarrow GC}$ & 1060 & -    & 375    & 404    & 1839           \\
    \bottomrule
  \end{tabular}
\end{table}

Dans le but de comparer la quantité de bases A et T qui ont été converties en G
et C ($N_{AT \rightarrow GC}$) à celle des bases G et C qui ont été converties
en A et T ($N_{GC \rightarrow AT}$), nous avons échantillonné aléatoirement,
par construction donneuse, 80 clones pour lesquels on dispose des 23 marqueurs.
Cet échantillonnage permet de compenser les éventuelles différences de longueur
moyenne de région convertie entre les constructions donneuses.

$N_{AT \rightarrow GC}$ est significativement supérieure à $N_{GC \rightarrow
  AT}$ (test du $\chi^2$, probabilité critique \(=\) \num{0.029} au premier
tirage). Pour compenser un effet dû à la nature des transformants échantillonnés
pendant ce tirage, nous avons répété la mesure \num{1e4} fois. Dans
\SI{99.98}{\percent} des cas, $N_{AT \rightarrow GC}$ est supérieure à $N_{GC
  \rightarrow AT}$. Le test est significatif au seuil de \num{0.05} dans
\SI{44.01}{\percent} des cas. Ce résultat indique que sur l'ensemble des
transformants séquencés, un nombre plus élevé de bases AT ont été converties en
GC que l'inverse, et que l'écart est significatif dans \SI{44.01}{\percent} des
cas.
\newpage
