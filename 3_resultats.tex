
\null
\vfill
\begin{center}
  \rmfamily
  % \fontspec{Gill Sans}
  \setstretch{1.0}

  \tikzset{trace text/.style = {align = left, below right, font = \scriptsize}}
  \tikzset{trace legend/.style = {Black, opacity = 1, align = center, text width = 4.5cm,
      above, font = \scriptsize}} %
  \tikzset{trace fleche/.style={Gray, dotted, thick, opacity = 0.6}}
  % \tikzset{trace text/.style={black, text width = 7cm, font = \scriptsize, above}}

  \begin{tikzpicture}

    %% Trace de conversion des donneurs strong

    \node[trace text] at (0, 0) {%
      \textcolor{white}{cuicui}\\ % un titre était présent ici. doublon avec le titre en bas. remplacé
         % par un blank space pour ne pas avoir à tout décaler.
      \includegraphics[width = 0.65\textwidth]{img/trace_ws.pdf}};

    \coordinate (complexe strong) at (8, -6.3);
    \coordinate (complexe weak) at (6.7, -6.3);
    \coordinate (badqual) at (1.4, -14);
    \coordinate (tout bleu) at (12, -13.5);

    % \draw[opacity = 0.2, line width = 0.01pt, Gray] (0,0) grid (15, -15);
    % \draw[opacity = 0.2, line width = 0.005pt, step = 0.5, Gray] (0,0) grid (15, -15);

    \draw[trace fleche, gs_rec_col] [<-] (10.2, -1.0) -- (12, -1.0)
    node[trace legend] {\textcolor{gs_rec_col}{Génotype du receveur}}
    node {\(\bullet\)}
    ;

    \draw[trace fleche, gs_don_col] [<-] (10.2, -14.35) -- (12, -14.35)
    node[trace legend] {\textcolor{gs_don_col}{Génotype du donneur}}
    node {\(\bullet\)}
    ;

    \draw[trace fleche] (11.5, -5) %
    node{\(\bullet\)} %
    node[trace legend, darkgray] {Restauration de l'haplotype sauvage GC (\textsf{S}).}
    -- ++(-1, 0) -- (complexe strong);

    \draw[trace fleche]  (11.5, -8) %
    node{\(\bullet\)} %
    node[trace legend, darkgray] {Restauration de l'haplotype sauvage AT (\textsf{W}).}
    -- ++(-1, 0) -- (complexe weak);

    %% flèche sens du séquençage
    \draw[trace fleche, opacity = 0.4] [->] (5, -0.3)
    node[left, font=\scriptsize, opacity = 0.5] {Sens du séquençage}
    % node{\(\bullet\)}
    -- ++(2, 0)
    ;

    % \draw[opacity = 0.2, line width = 0.01pt, Gray] (0,0) grid (15, -19);
    % \draw[opacity = 0.2, line width = 0.005pt, step = 0.5, Gray] (0, 0) grid (15, -19);
    % \draw[]

    \begin{scope}[shift={(-2, 11.0)}]

      \draw[gs_rec_col,    very thick] (5,    -10.0) -- (7, -10.0);
      \draw[ancre_don_col, very thick] (11.4, -10.0) -- ++(-0.5, 0);
      \draw[solid,         am1_col,           very thick] (7.0, -10.0) -- ++(0.2, 0);
      \draw[solid,         kanr_col,          very thick] (7.2, -10.0) -- ++(2.0, 0);
      \draw[solid,         genome_col,        very thick] (9.2, -10.0) -- ++(0.2, 0);
      \draw[solid,         ancre_don_col,     very thick] (9.4, -10.0) -- ++(1.5, 0);
      \draw[genome_col,    thick, arrows = {-Stealth[left]}] (11.4, -10.0) -- ++(1,   0);
      \draw[genome_col,    thick] (2, -10.0) -- (5, -10.0);

      %% brin <-
      \draw[gs_rec_col, very thick] (5, -10.2)   -- (5.5, -10.2);
      \draw[gs_don_col, solid,          very thick] (6.0, -10.2) -- ++(1.0,  0);
      \draw[gs_rec_col, solid,          very thick] (6.0, -10.2) -- ++(-0.5, 0);
      \draw[solid,      am1_col,        very thick] (7.0, -10.2) -- ++(0.2,  0);
      \draw[solid,      kanr_col,       very thick] (7.2, -10.2) -- ++(2.0,  0) node[midway, below, font = \tiny] {KanR};
      \draw[solid,      genome_col,     very thick] (9.2, -10.2) -- ++(0.2,  0);
      % \draw[solid,      ancre_don_col,  very thick] (9.4, -10.2) --   (9.9,  -10.2);
      \draw[solid,      ancre_don_col,  very thick] (9.4, -10.2) -- ++(2.0,  0) node[midway, below, font = \tiny] {Ancre};
      % génome
      \draw[genome_col, thick] (11.4, -10.2) -- ++(1, 0);
      \draw[genome_col, thick, arrows = {Stealth[left]-}] (2, -10.2) -- (5, -10.2);

	    \draw[Gray, thick, densely dotted, fill = Gray, opacity = 0.3] (6, -10.4) rectangle (7, -9.8);
      \node[Gray, font=\tiny] at (6.5, -10.55) {Hétéroduplex};

      \end{scope}

      \draw[dotted, Gray, line width = 1pt] (1 , -0.8) -- ++(2, +1.5);
      \draw[dotted, Gray, line width = 1pt] (10, -0.8) -- ++(-5, +1.5);

    % \node[trace legend, align = justify] at (13.5, -13) {%
    %   Chaque ligne horizontale représente une séquence. Les points représentent
    %   les positions des marqueurs sur les séquences. L'intensité et le diamètre
    %   des points représentent le score de qualité du site. La couleur des points
    %   représente leur polarité. Ils sont bleus lorsque le site est dans la
    %   région convertie : ils correspondent à l'haplotype du donneur. Ils sont
    %   rouges lorsque le site est dans la région conservée.

    %   Les alternances rouge / bleu marquent la transition de l'haplotype
    %   converti à l'haplotype sauvage : le point de recombinaison est localisé
    %   entre ces deux marqueurs.

    %   Les séquences sont triées par longueur de région convertie. Certains
    %   transformants ont converti tous les marqueurs\tikz[overlay]{\draw[gray,
    %     dotted, opacity = 0.9] [->] (0, 0) -- ++(-4, 10) -- ++(-1, 0);}.
    %   Certains transformants ont conservé tous les
    %   marqueurs\tikz[overlay]{\draw[gray, dotted, opacity = 0.7] [->] (0, 0) -- (-2, -1) -- (-3, -1);}.
    %   %
    % };
    %% DONE ajuster les flèches indiquant les séquences convertissant

  \end{tikzpicture}
\end{center}
\caption[Zones de recombinaison détaillée]{\textbf{Zones de recombinaison entre un locus génomique
    d'\emph{Acinetobacter baylyi} et un gène synthétique donneur d'allèles CG
    et AT.}%
  \rmfamily

  Chaque ligne horizontale représente une séquence. Les points représentent
  les positions des marqueurs sur les séquences. L'intensité et le diamètre
  des points représentent le score de qualité du site. Les points sont bleus lorsque le site est dans la
  région convertie : ils correspondent à l'haplotype du donneur. Ils sont
  rouges lorsque le site est dans la région conservée.
  Les séquences sont triées par longueur de région convertie.
  Les alternances rouge~/~bleu marquent la transition de l'haplotype
  converti à l'haplotype sauvage : le point de recombinaison est localisé
  entre ces deux marqueurs.
}%
\vfill
\label{fig:convtract}
\thispagestyle{empty}
\addtocounter{page}{-1}
\newpage

\afterpage{%
  \null
  \vfill

  \begin{table}
    \rmfamily
    \centering

    \caption[Fréquences de transformation]{\textbf{Fréquences de transformation}}
    \label{tab:transfo-freq}
    \vspace{0.5cm}

    \begin{tabular}{cc}
      \toprule
      \thead{\normalsize Construction \\ \normalsize Donneuse} & \thead{\normalsize Fréquences de \\ \normalsize transformation} \\
      \midrule
      CG    & \num{6.53e-5} \\
      AT    & \num{2.42e-5} \\
      AT/CG & \num{4.97e-5} \\
      CG/AT & \num{1.74e-4} \\
      \bottomrule
    \end{tabular}
  \end{table}

  \vfill

  \includegraphics[scale = 0.9]{img/distr_rcb_pt.pdf}
  \caption[Distribution de la position du dernier marqueur]{%
    \textbf{Distribution de la position du dernier
      marqueur converti} \\ \rmfamily Ce graphique représente en ordonnées le
    nombre de transformants dont le dernier marqueur converti est à la
    position représentée en abscisse. Les panneaux du graphique représentent
    les quatres constructions donneuses. La position du dernier marqueur
    converti indique la position du point de recombinaison. Les transformants
    qui ne montrent aucun marqueur converti sont indiqués par des $N$.%
  }
  \label{fig:distrircb}
\end{figure}

\vfill
\thispagestyle{empty}
\addtocounter{page}{-1}
\newpage
}
% ==============================================================================
\section{Résultats}
\label{sec:resultats}
% ==============================================================================

Nous avons transformé une supension d'\emph{Acinetobacter} par des constructions
dont l'intégration dans le génome par recombinaison homologue entraîne la
réparation des mésappariemments, un mécanisme qui est biaisé vers l'introduction
des bases CG chez les eucaryotes.

Les fréquences de transformations obtenues sont représentées dans le tableau
\ref{tab:transfo-freq} page \pageref{tab:transfo-freq}. Le type de construction
donneuse n'a pas d'influence sur l'efficacité de la transformation. Les
fréquences de l'ordre de \num{1e-5} ont permis d'obtenir un grand nombre de
recombinants.

En moyenne, \num{393} \(\pm\) \num{228} nucléotides ont été transférés et
intégrés dans le génome.

La figure \ref{fig:convtract} représente le détail des zones de recombinaison
obtenues avec une construction donneuse alternant CG et AT. Les zones de
recombinaisons des clones transformés par les donneurs CG, AT et AT/CG sont
détaillées en annexe \ref{sec:trac-de-conv}.

Nous nous sommes intéressés à deux paramètres permettant d'estimer les
fréquences de conversion en faveur de GC chez \emph{A. baylyi} : la position du
point de recombinaison, et la correction ponctuelle des mésappariemments.

% ------------------------------------------------------------------------------
\subsection{Comparaison de la longueur des régions converties}
\label{subsec:longueur}
% ------------------------------------------------------------------------------

Selon l'hypothèse \ac{gbgc}, la région convertie devrait être plus longue
lorsque la construction donneuse induit des réparations vers CG que lorsqu'elle
induit des réparations vers AT. La différence entre la longueur moyenne de
région convertie par les donneurs respectivement CG et AT n'est pas
significative (test de Wilcoxon, probabilité critique~\(=\) \num{0.31}) (voir
figure \ref{fig:distrircb}). De la même façon, la différence entre la longueur
de région convertie par le donneur AT/CG et celle par le donneur CG/AT n'est pas
significative (test de Wilcoxon, probabilité critique~\(=\) \num{0.22}).
Globalement, le type de construction donneuse n'explique pas la variabilité de
la longueur de région convertie (test de Kruskal-Wallis, probabilité
critique~\(=\) \num{0.10}). Le type de donneur n'a pas d'influence sur la
longueur de région convertie.

\subsection{Distribution du dernier marqueur converti}
\label{subsec:distribution-points}

Les constructions alternant AT et CG permettent de déterminer si le point de
recombinaison se situe plus souvent après un marqueur introduisant une
conversion vers CG qu'après un marqueur introduisant une conversion vers AT
(voir table \ref{tab:doublets}). Le dernier marqueur converti est AT dans 92
transfomants ; il est CG dans 83 transformants. Cet écart n'est pas significatif
(test du \(\chi^2\) d'homogénéité, probabilité critique~\(=\)~\num{0.49}) (voir
table \ref{tab:doublets}).

Lorsque le premier marqueur donneur est AT, le dernier marqueur converti est
plus souvent AT que GC. De la même façon, lorsque le premier marqueur donneur
est GC, le dernier marqueur converti est plus souvent GC que AT.


% ------------------------------------------------------------------------------
\subsection{Restaurations de l'haplotype sauvage}
\label{subsec:restaur}
% ------------------------------------------------------------------------------

Certains transformants montrent des régions de conversions qui alternent entre
l'allèle sauvage receveur et l'allèle donneur. Ces alternances ponctuelles
affectent de 1 à 3 marqueurs consécutifs (voir figure \ref{fig:convtract}). Nous
avons confirmé qu'il s'agissait bien de restaurations de l'allèle sauvage de
deux façons. 1) Expérimentalement, nous avons séquencé une sous-population de
clones issues d'un isolat séquencé en premier lieu. Tous montrent la même
alternance au même site (voir figure~\ref{fig:confirm-haplotype} en
annexe~\ref{subsec:confirm-haplotype}). 2) Analytiquement, le score de qualité
moyen des sites montrant des restaurations de l'allèle sauvage permet de
s'affranchir d'une possible erreur de séquençage : celle-ci se traduit
généralement par un indice de qualité plus faible au site concerné. Le score de
qualité moyen est de \num{49.36} aux sites restaurés, contre \num{52.79} aux
sites non-restaurés. La différence entre les deux n'est pas significative (test
de Wilcoxon, probabilité critique \(=\) \num{0.94}). Les marqueurs correspondant
à des restaurations de l'haplotype sauvage ne sont donc pas des erreurs de
séquençage, et correspondent à un signal biologique.

% Certains transformants montrent des restaurations ponctuelles de l'allèle
% sauvage dans la région convertie. Ces restaurations peuvent être des
% restaurations de l'allèle AT ou de l'allèle CG. Sur les 14 cas observés,
% \num{10} sont des restaurations de l'allèle CG, \num{4} de l'allèle AT.
Sur les 14 cas de restaurations de l'haplotype sauvage, \num{4} sont des
restaurations de l'allèle AT, \num{10} sont des restaurations de l'allèle CG.
Cet écart n'est pas significatif (voir la table \ref{tab:restaur}). L'écart
n'est pas statistiquement significatif (test du \(\chi^2\) d'homogénéité,
probabilité critique~\(=\)~\num{0.11}).


\texttt{Si encore places après modifications, placer tableau suivant là. }

% tables de comptage des cas de restaurations
\afterpage{
  \null
  \vfill
  \begin{table}[h!]
    \rmfamily
    \centering

    \caption[Dénombrements des derniers marqueurs avant le point de
    recombinaison]{\textbf{Dénombrements des derniers marqueurs avant le point de
        recombinaison.}
      % \rmfamily       %
    }
    \label{tab:doublets}

    \begin{tabular}{@{}ccc@{}}
      \toprule
      & \multicolumn{2}{c}{Dernier marqueur converti} \\
      \cmidrule(r){2-3}
      \thead{\normalsize ADN synthétique \\ \normalsize donneur } & AT & CG \\
      \midrule
      AT/CG & 55 & 33 \\
      CG/AT & 37 & 50 \\
      \midrule[0.1pt]
      Total & 92 & 83 \\
      \bottomrule

    \end{tabular}

  \end{table}

  \vfill

  \begin{table}[htbp]
    \centering
    \rmfamily

    \caption[Dénombrement des cas de restauration]{\textbf{Dénombrement des cas de
        restauration} \\
      \rmfamily       %
    }
    \label{tab:restaur}

    \begin{tabular}{@{}ccc@{}}
      \toprule
      & \multicolumn{2}{c}{Nucléotide Restauré} \\
      \cmidrule(r){2-3}
      \thead{\normalsize ADN synthétique \\ \normalsize donneur } & AT & CG \\
      \midrule
      AT    & - & 4 \\
      CG    & 0 & - \\
      AT/CG & 3 & 2 \\
      CG/AT & 1 & 4 \\
      \midrule[0.1pt]
      Total & 4 & 10 \\
      \bottomrule

    \end{tabular}
  \end{table}
  \vfill
  \thispagestyle{empty}
  \addtocounter{page}{-1}
}