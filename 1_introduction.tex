\section*{Introduction}
\label{sec:introduction}

Le contenu en base est un paramètre clé de variation des séquences au sein des
génomes et entre génomes. Chez les procaryotes, le taux de guanine et cytosine
(\ac{gc}) varie de 16.5\% à 75\% dans les génomes séquencés à ce jour. Une telle
amplitude soulève plusieurs questions.

\begin{itemize}
\item Premièrement, comment ce taux de GC varie-t-il ? Est-ce qu'il varie
  aléatoirement, localement au niveau du gène ou globalement au niveau de
  l'espèce ?
\item Deuxièmement, quels sont les mécanismes responsables de ces variations ?
  Est-ce qu'ils sont associés à la réplication de l'ADN, à sa réparation ou à
  une pression de sélection sur l'usage du code génétique ?
\item Troisièmement, quelles sont les conséquences fonctionnelles des variations
  en \ac{gc} ? Est-ce qu'elles impactent la stabilité de la molécule d'ADN ?
  Est-ce qu'elles influent sur les patrons de mutations ?
\end{itemize}

Malgré

% Les génomes bactériens ont des compositions en base très variables. Les
% symbiotes en association obligatoire avec les insectes ont des génomes composés
% essentiellement d'adénine et de thymine. D'autres actinobactéries Divers facteurs ont été proposés pour
% expliquer ces variations : des variations dans l'intensité des pressions de
% sélection,
