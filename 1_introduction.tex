% vi: filetype=tex

\section*{Introduction}
\addcontentsline{toc}{section}{Introduction}
\label{sec:introduction}

Le contenu en base est un paramètre clé de variation des séquences au sein des
génomes et entre génomes. Chez les procaryotes, le taux de guanine et cytosine
(\ac{gc}) varie de 16.5\% à 75\% dans les génomes séquencés à ce jour. Une
telle amplitude soulève plusieurs questions. Premièrement, comment ce taux de
GC varie-t-il ? Est-ce qu'il varie aléatoirement, localement au niveau du gène
ou globalement au niveau de l'espèce ? Deuxièmement, quels sont les mécanismes
responsables de ces variations ? Est-ce qu'ils sont associés à la réplication
de l'ADN, à sa réparation ou à une pression de sélection sur l'usage du code
génétique ?  Troisièmement, quelles sont les conséquences fonctionnelles des
variations en \ac{gc} ? Est-ce qu'elles impactent la stabilité de la molécule
d'ADN ?  Est-ce qu'elles influent sur les patrons de mutations ?

Au cours de ces quinzes dernières années, il a été démontré que la
recombinaison homologue tend à augmenter localement le taux de GC dans les
régions fortement recombinantes des génomes
eucaryotes\cite{duret_biased_2009,lesecque_gc-biased_2013}. Au cours de la
réparation des cassures doubles brins par recombinaison homologue, les
mésappariemments locaux sont réparés par un mécanisme occasionnant de la
\emph{conversion génique}\cite{chen_gene_2007}. Ce mécanisme est biaisé vers
l'introduction préférentielle des bases C et G chez les mammifères et
probablement chez un grand nombre d'eucaryotes\cite{pessia_evidence_2012}.
C'est un processus neutre : il impacte les régions codantes et non-codantes. Il
peut s'opposer à l'action de la sélection en augmentant la probabilité de
fixation des allèles G et C\cite{ratnakumar_detecting_2010}.

En 2010, deux études simultanées\cite{hildebrand_evidence_2010,
hershberg_evidence_2010} démontrent deux choses : 1) la \emph{mutation}
est universellement biaisée vers A et T chez les procarytes, et 2) les bases G
et C ont une probabilité de fixation plus élevée.
