
\newpage

% figure de la recombinaison homologue
\afterpage{%
  \null
  \vfill
  \begin{center}
  \rmfamily
  \setstretch{1.0}
  \tikzset{plasmid lines/.style={xscale = 0.2, yscale = 0.2, very thick}}
  \tikzset{legende/.style={font=\scriptsize, color=Gray, align=right}}
  \tikzset{transition/.style={color=Gray, very thick}}
  \tikzset{legende text/.style={font=\scriptsize, color=Black, align=right}}

  \begin{tikzpicture}[line width = 2pt, join = round, scale = 0.8]

    \begin{scope}
      % représente la construction, avec l'insertion de la cassette de résistance
      \node[font=\bfseries\sffamily] at (1, 0)    {a};
      \node[font=\bfseries\sffamily] at (1, -1)   {b};
      \node[font=\bfseries\sffamily] at (1, -2.5) {c};
      \node[font=\bfseries\sffamily] at (1, -6)   {d};
      \node[font=\bfseries\sffamily] at (1, -6)   {d};
      \node[font=\bfseries\sffamily] at (1, -12)  {e};
      \node[font=\bfseries\sffamily] at (1, -16.5)  {f};

      \draw[gs_don_col] (4, 0) -- (6, 0) node[above, midway, font=\tiny, align = center] {Gène \\ de synthèse};
      \draw[am1_col, ultra thick] (6, 0) -- (6.2, 0);
      \draw[kanr_col] (6.2, 0) -- (7.8, 0) node[above, midway, font=\tiny] {KanR};
      \draw[genome_col, ultra thick] (7.8, 0) -- (8.0, 0);
      \draw[ancre_don_col] (8.0, 0) -- (9.8, 0) node[above, midway, font=\tiny] {Ancre};
      \draw[am3_col, ultra thick] (9.8, 0) -- (10.0, 0);
      \draw[plasmid_col, thick] (2, 0) -- (4, 0);

      \draw[arrows = {-Stealth[left]}, plasmid_col, thick] (10, 0) -- (12, 0)
      node[font = \tiny, above, align = right]
      {Construction \\ plasmidique};

      \draw[plasmid_col, dotted, thin, out=0, in = 15]
      [->] (12.1, 0) to (10.8, -4.6)
      node[Gray, legende, right]
      {Transformation};

      % Génome acinetobacter
      % brin 5' 3'
      \draw[gs_rec_col] (5, -1) -- (7, -1);
      \draw[ancre_don_col] (7.0, -1) -- (8.8, -1);
      \draw[genome_col, thick] (2, -1) -- (5, -1);
      \draw[arrows = {-Stealth[left]}, genome_col, thick] (8.8, -1) -- (12, -1)
      node[font=\tiny, above, align = right] {Génome \\ \emph{Acinetobacter}};

      %amorces
      % \draw[Red3, ultra thick] (4, -1) -- (4.2, -1) node[above left, font=\tiny] {Amorce};
      \draw[Gray, densely dotted, thin] [->] (2, -0.5) node[above right=-1pt , font=\tiny] {Homologues} -- ++(3, 0)  -- ++(0.5, +0.4);
      \draw[Gray, densely dotted, thin] [->] (2, -0.5) -- ++(3, 0) -- ++(0.5, -0.4);
      \draw[Gray, densely dotted, thin] [->] (2, -0.5) -- ++(6, 0) -- ++(0.5, +0.4);
      \draw[Gray, densely dotted, thin] [->] (2, -0.5) -- ++(6, 0) -- ++(0.5, -0.4);

      \draw[gs_rec_col] (5, -1.2) -- (7, -1.2);
      \draw[ancre_don_col] (7.0, -1.2) -- (8.8, -1.2);
      \draw[arrows = {Stealth[right, swap]-}, genome_col, thick] (2, -1.2) -- (5, -1.2);
      \draw[genome_col, thick] (8.8, -1.2) -- (12, -1.2);

      % Lignes reliant les fragments
      \draw[gs_don_col, thin, dotted] (4, 0) -- (5, -1);
      \draw[gs_don_col, thin, dotted] (6, 0) -- (7, -1);
      \draw[ancre_don_col,  thin, dotted] (8, 0) -- (7, -1);
      \draw[ancre_don_col,  thin, dotted] (9.8, 0) -- (8.8, -1);

      % petit éclair pour marquer la cassure
      \node[Gold2, font=\Large] at (7, -1.1) {\Lightning};
      \node[Gray, font=\tiny] at (7, -1.4) {Cassure Double Brin};

      %%
      %% RÉSECTION
      %%
      \draw[transition] [->] (7, -1.6) -- (7, -2.4) node[legende, left, midway] {Résection};

      % brin ->
      \draw[gs_rec_col, arrows = {-Stealth[left, scale = 1]}, very thick] (5, -2.5) -- (7, -2.5);
      \draw[ancre_don_col] (8.3, -2.5) -- (8.8, -2.5);
      \draw[genome_col, thick] (2, -2.5) -- (5, -2.5);
      \draw[arrows = {-Stealth[left]}, genome_col, thick] (8.8, -2.5) -- (12, -2.5);
      % brin <-
      \draw[gs_rec_col, ultra thick] (5, -2.7) -- (5.5, -2.7);
      \draw[arrows = {Stealth[right, swap, scale = 1]-}, ancre_don_col, very thick] (7.0, -2.7) -- (8.8, -2.7);
      \draw[arrows = {Stealth[right, swap]-}, genome_col, thick] (2, -2.7) -- (5, -2.7);
      \draw[genome_col, thick] (8.8, -2.7) -- (12, -2.7);

      %%
      %% RÉPARATION
      %%
      \draw[transition] [->] (7, -2.8) -- (7, -3.4) node[legende, midway, left] {Réparation};

      \draw[gs_rec_col, arrows = {-Stealth[left, scale = 1]}, very thick] (5, -3.5) -- (7, -3.5);
      \draw[ancre_don_col] (8.3, -3.5) -- (8.8, -3.5);
      \draw[genome_col, thick] (2, -3.5) -- (5, -3.5);
      \draw[arrows = {-Stealth[left]}, genome_col, thick] (8.8, -3.5) -- (12, -3.5);
      % brin <-
      \draw[gs_rec_col, ultra thick] (5, -3.7) -- (5.5, -3.7);
      \draw[ ancre_don_col, very thick] (8.8, -3.7) -- ++(-0.5, 0) -- ++(-0.2, -0.6) -- ++(-0.7, 0);
      \draw[arrows = {Stealth[right, swap]-}, genome_col, thick] (2, -3.7) -- (5, -3.7);
      \draw[genome_col, thick] (8.8, -3.7) -- (12, -3.7);

      %% fragment insert avec boucle pour cassette de résistance
      % je place l'origine de la construction au début da la région ancre sur le
      % génome d'acinetobacter. C'est plus facile pour s'aligner
      \draw[ancre_don_col, very thick]                     (8.8, -4.6) -- ++(-2, 0);
      \draw[genome_col, very thick]                  (6.8, -4.6) -- ++(-0.2, 0);
      \draw[kanr_col, very thick]                     (6.6, -4.6) -- ++(-2, 0);
      \draw[am1_col, very thick]                       (4.6, -4.6) -- ++(-0.2, 0);
      \draw[gs_don_col, very thick]                    (4.4, -4.6) -- ++(-2, 0);
      \draw[plasmid_col, thick]                            (2.4, -4.6) -- ++(-0.4, 0);
      \draw[plasmid_col, thick, arrows = {-Stealth[left]}] (8.8, -4.6) -- ++(2, 0);


      %%
      %% RESYNTHÈSE
      %%

      \draw[densely dotted, ancre_don_col, very thick]    (7.35, -4.3) --  (6.8, -4.3);
      \draw[densely dotted, genome_col, very thick] (6.8, -4.3) -- ++(-0.2, 0);
      \draw[densely dotted, kanr_col, very thick]    (6.6, -4.3) -- ++(-2, 0);
      \draw[densely dotted, am1_col, very thick]      (4.6, -4.3) -- ++(-0.2, 0);
      \draw[densely dotted, gs_don_col, very thick, arrows = {-Stealth[left, scale = 0.5]}]   (4.4, -4.3) -- ++(-1.5, 0);
      %% lignes permettant de montrer l'appariemment des régions
      \draw[gs_don_col, thin, dotted ] (2.9, -4.3) -- (5, -3.5);
      \draw[gs_don_col, thin, dotted ] (4.4, -4.3) -- (7, -3.5);

      %%
      %% RÉAPPARIEMMENT
      %%

      \draw[Gray, transition] [->] (7, -4.8) -- ++(0, -1)
      node[midway, left, legende] {Réappariemment};

      % brin ->
      \draw[gs_rec_col, arrows = {-Stealth[left, scale = 1]}, very thick] (5, -6) -- (7, -6);
      \draw[ancre_don_col, very thick] (11.4, -6) -- ++(-0.5, 0);
      \draw[genome_col, thick, arrows = {-Stealth[left]}] (11.4, -6) -- ++(1, 0);
      % \draw[ancre_don_col] (8.3, -6) -- (8.8, -6);
      \draw[genome_col, thick] (2, -6) -- (5, -6);

      %% brin <-
      % petit bout bleu
      \draw[gs_rec_col, ultra thick]                        (5, -6.2)   --   (5.5, -6.2);
      \draw[gs_don_col, densely dotted, very thick, arrows = {Stealth[left, scale = 0.5]-}] (6.0, -6.2) -- ++(1.0, 0);
      \draw[densely dotted, am1_col, very thick]            (7.0, -6.2) -- ++(0.2, 0);
      \draw[densely dotted, kanr_col, very thick]          (7.2, -6.2) -- ++(2.0, 0);
      \draw[densely dotted, genome_col, very thick]       (9.2, -6.2) -- ++(0.2, 0);
      \draw[densely dotted, ancre_don_col, very thick]          (9.4, -6.2) --   (9.9, -6.2);
      \draw[solid, ancre_don_col, very thick]                   (9.9, -6.2) -- ++(1.5, 0);
      % génome
      \draw[genome_col, thick]                            (11.4, -6.2) -- ++(1, 0);
      \draw[genome_col, thick, arrows = {Stealth[left]-}] (2, -6.2) --      (5, -6.2);

	    \draw[Gray, thick, densely dotted, fill = Gray, opacity = 0.3]
      (6, -8.4) rectangle ++(1, +0.6);
	    \draw[Gray, thick, densely dotted, fill = Gray, opacity = 0.3]
      (6, -6.4) rectangle ++(1, +0.6);

      %%
      %% SYNTHÈSE DU BRIN 5'
      %%
      \draw[transition] [->] (7, -6.5) -- ++(0, -1)
      node[midway, left, legende, align = right] {Synthèse du brin \\ complémentaire };

      %% brin ->
      \draw[gs_rec_col, arrows = {-Stealth[left, scale = 1]}, very thick] (5, -8.0) -- (7, -8.0);
      \draw[ancre_don_col, very thick] (11.4, -8.0) -- ++(-0.5, 0);
      \draw[genome_col, thick, arrows = {-Stealth[left]}] (11.4, -8.0) -- ++(1, 0);
      \draw[genome_col, thick] (2, -8.0) -- (5, -8.0);
      \draw[densely dotted, am1_col, very thick] (7.0, -8.0) -- ++(0.2, 0);
      \draw[densely dotted, kanr_col, very thick] (7.2, -8.0) -- ++(2.0, 0);
      \draw[densely dotted, genome_col, very thick] (9.2, -8.0) -- ++(0.2, 0);
      \draw[densely dotted, ancre_don_col, very thick] (9.4, -8.0) -- ++(1.5, 0);

      %% brin <-
      \draw[gs_rec_col, very thick] (5, -8.2)   --   (5.5, -8.2);
      \draw[gs_don_col, solid, very thick] (6.0, -8.2) -- ++(1.0, 0);
      \draw[gs_rec_col, densely dotted, very thick, arrows = {-Stealth[left, scale = 0.5]}] (6.0, -8.2) -- ++(-0.5, 0);
      \draw[solid, am1_col, very thick] (7.0, -8.2) -- ++(0.2, 0);
      \draw[solid, kanr_col, very thick] (7.2, -8.2) -- ++(2.0, 0);
      \draw[solid, genome_col, very thick] (9.2, -8.2) -- ++(0.2, 0);
      \draw[solid, ancre_don_col, very thick] (9.4, -8.2) --   (9.9, -8.2);
      \draw[solid, ancre_don_col, very thick] (9.9, -8.2) -- ++(1.5, 0);
      % génome
      \draw[genome_col, thick] (11.4, -8.2) -- ++(1, 0);
      \draw[genome_col, thick, arrows = {Stealth[left]-}] (2, -8.2) -- (5, -8.2);


      %%
      %% RÉSULTAT FINAL
      %%

      \draw[transition] [->] (7, -8.5) -- ++(0, -1) node[midway, left, legende] {Ligature};

      \draw[gs_rec_col, very thick] (5, -10.0) -- (7, -10.0);
      \draw[ancre_don_col, very thick] (11.4, -10.0) -- ++(-0.5, 0);
      \draw[genome_col, thick, arrows = {-Stealth[left]}] (11.4, -10.0) -- ++(1, 0);
      \draw[genome_col, thick] (2, -10.0) -- (5, -10.0);
      \draw[solid, am1_col, very thick] (7.0, -10.0) -- ++(0.2, 0);
      \draw[solid, kanr_col, very thick] (7.2, -10.0) -- ++(2.0, 0);
      \draw[solid, genome_col, very thick] (9.2, -10.0) -- ++(0.2, 0);
      \draw[solid, ancre_don_col, very thick] (9.4, -10.0) -- ++(1.5, 0);

      %% brin <-
      \draw[gs_rec_col, very thick] (5, -10.2)   -- (5.5, -10.2);
      \draw[gs_don_col, solid, very thick] (6.0, -10.2) -- ++(1.0, 0);
      \draw[gs_rec_col, solid, very thick] (6.0, -10.2) -- ++(-0.5, 0);
      \draw[solid, am1_col, very thick] (7.0, -10.2) -- ++(0.2, 0);
      \draw[solid, kanr_col, very thick] (7.2, -10.2) -- ++(2.0, 0);
      \draw[solid, genome_col, very thick] (9.2, -10.2) -- ++(0.2, 0);
      \draw[solid, ancre_don_col, very thick] (9.4, -10.2) --   (9.9, -10.2);
      \draw[solid, ancre_don_col, very thick] (9.9, -10.2) -- ++(1.5, 0);
      % génome
      \draw[genome_col, thick] (11.4, -10.2) -- ++(1, 0);
      \draw[genome_col, thick, arrows = {Stealth[left]-}] (2, -10.2) -- (5, -10.2);

      %% rectangle de zoom
	    \draw[Gray, thick, densely dotted, fill = Gray, opacity = 0.3] (6, -10.4) rectangle (7, -9.8);

      % \draw[Gray, very thick, densely dotted, opacity = 0.6, bend right = 5]
      % [->] (7, -10.4) to (14.75, -10.75);

	    % \draw[Gray, thick, densely dotted]
      % (4, -10.6) rectangle (7.6, -9.6);

      % \draw[Gray, very thick, densely dotted, opacity = 0.6, out = 180, in = 180]%, bend right = 75]
      % [->] (4, -10.6) to (3.8, -17);
    \end{scope}

    \begin{scope}[shift={(4, -12)}, scale = 0.5]
      %% Zoom montrant l'alignement des positions donneurs avec les
      %% positions receveuse

      \draw[Gray, dotted, thick] (-0.5, 2.1) to ++(+4.5, 1);
      \draw[Gray, dotted, thick] (10.5, 2.1) to ++(-4.5, 1);


      \draw[Gray, thick, densely dotted, fill = Gray, opacity = 0.3]
      (-0.5, -1.1) rectangle ++(11, 3.2);

      \draw[gs_rec_col, thick, arrows = {-Stealth[left]}] (0, 1) -- ++(10, 0)
      node[above, font=\scriptsize] {Brin receveur};
      \draw[gs_don_col, thick, arrows = {Stealth[left]-}] (0, 0) -- ++(10, 0)
      node[below, font=\scriptsize] {Brin donneur};

      \foreach \x in {1,2,...,9}
      {
        \draw[color = gs_rec_col, fill = gs_rec_col] (\x, 1) circle (0.05);
        \draw[color = gs_don_col, fill = gs_don_col] (\x, 0) circle (0.05);
      }

      \draw[Gray, very thick] [->] (5, -1.5) -- ++(0, -1)
      node[midway, left, font=\tiny, align = right]
      {Correction des \\ mésappariemments};
    \end{scope}


    \begin{scope}[shift={(4, -15)}, scale = 0.5]
      %% Zoom montrant la correction des mésappariemments dans les clones

     \draw[gs_rec_col, thick, dotted, fill = gs_rec_col, opacity = 0.3]
     (-0.5, -0.5) rectangle (4.5, 2.5)
     node[midway, opacity = 1, font = \tiny] {Région conservée};

     \draw[gs_don_col, thick, dotted, fill = gs_don_col, opacity = 0.3]
     (4.5, -0.5) rectangle (10.5, 2.5)
     node[midway, opacity = 1, font = \tiny] {Région convertie};

      \draw[gs_rec_col, thick, arrows = {-Stealth[left]}] (0, 2) -- ++(10, 0);
      \draw[gs_don_col, thick, arrows = {Stealth[left]-}] (0, 0) -- ++(10, 0);

      \foreach \x in {4,5,...,9}
      {
        \draw[color = gs_don_col, fill = gs_don_col] (\x, 2) circle (0.05);
        \draw[color = gs_don_col, fill = gs_don_col] (\x, 0) circle (0.05);
      }

      \foreach \x in {1,2,...,4}
      {
        \draw[color = gs_rec_col, fill = gs_rec_col] (\x, 2) circle (0.05);
        \draw[color = gs_rec_col, fill = gs_rec_col] (\x, 0) circle (0.05);
      }

      \draw[Gray, densely dash dot, very thick]
      (4.5, 2.6) -- ++(0, -3.6)
      node[below, font=\tiny] {Point de recombinaison};
    \end{scope}

    \begin{scope}[shift={(0, -6.5)}]
      \draw[Gray, dotted, thick, bend left=15] (6, -9.8) to ++(-2.2, +1.0);
      \draw[Gray, dotted, thick, bend right=15] (7, -9.8) to ++(+2.2, +1.0);
      % \draw[Gray, dotted, thick] () -- ++();
      % (6, -10.4) rectangle (7, -9.8);

      \draw[gs_rec_col,    very thick] (5,    -10.0) -- (7, -10.0);
      \draw[ancre_don_col, very thick] (11.4, -10.0) -- ++(-0.5, 0);
      \draw[solid,         am1_col,           very thick] (7.0, -10.0) -- ++(0.2, 0);
      \draw[solid,         kanr_col,          very thick] (7.2, -10.0) -- ++(2.0, 0);
      \draw[solid,         genome_col,        very thick] (9.2, -10.0) -- ++(0.2, 0);
      \draw[solid,         ancre_don_col,     very thick] (9.4, -10.0) -- ++(1.5, 0);
      \draw[genome_col,    thick, arrows = {-Stealth[left]}] (11.4, -10.0) -- ++(1,   0);
      \draw[genome_col,    thick] (2, -10.0) -- (5, -10.0);

      %% brin <-
      \draw[gs_rec_col, very thick] (5, -10.2)   -- (5.5, -10.2);
      \draw[gs_don_col, solid,          very thick] (6.0, -10.2) -- ++(1.0,  0);
      \draw[gs_rec_col, solid,          very thick] (6.0, -10.2) -- ++(-0.5, 0);
      \draw[solid,      am1_col,        very thick] (7.0, -10.2) -- ++(0.2,  0);
      \draw[solid,      kanr_col,       very thick] (7.2, -10.2) -- ++(2.0,  0) node[midway, below, font = \tiny] {KanR};
      \draw[solid,      genome_col,     very thick] (9.2, -10.2) -- ++(0.2,  0);
      % \draw[solid,      ancre_don_col,  very thick] (9.4, -10.2) --   (9.9,  -10.2);
      \draw[solid,      ancre_don_col,  very thick] (9.4, -10.2) -- ++(2.0,  0) node[midway, below, font = \tiny] {Ancre};
      % génome
      \draw[genome_col, thick] (11.4, -10.2) -- ++(1, 0);
      \draw[genome_col, thick, arrows = {Stealth[left]-}] (2, -10.2) -- (5, -10.2);


      \foreach \x in {0, 0.125,...,0.5} {
        \draw[color = gs_don_col, fill = gs_don_col] (\x + 6.5, -10) circle (0.01);
        \draw[color = gs_don_col, fill = gs_don_col] (\x + 6.5, -10.2) circle (0.01);
      }
      \foreach \x in {0.250, 0.375,...,1.5} {
        \draw[color = gs_rec_col, fill = gs_rec_col] (\x + 5, -10) circle (0.01);
        \draw[color = gs_rec_col, fill = gs_rec_col] (\x + 5, -10.2) circle (0.01);
      }

	    \draw[Gray, thick, densely dotted, fill = Gray, opacity = 0.3]
      (6, -10.4) rectangle (7, -9.8);
      \node[Gray, font=\tiny] at (6.5, -10.55) {Hétéroduplex};

      % \draw[Gray, dotted, thick] (4.7, -10.2) to ++(-2, -1);
      % \draw[Gray, dotted, thick] (7.2, -10.2) to ++(+2, -1);
	    \draw[Gray, thick, densely dotted]
      (4.7, -10.7) rectangle (7.2, -9.6);
      \end{scope}

  \end{tikzpicture}

\end{center}

  \caption[Mécanismes moléculaires de conversion génique]{\textbf{Constructions
      moléculaires et mécanismes de conversion génique au cours de la
      transformation naturelle} }%
  \label{img:construct}
  \vfill

  \thispagestyle{empty}
  \addtocounter{page}{-1}
  \newpage
}


%==============================================================================
\section*{Introduction}
\addcontentsline{toc}{section}{Introduction}
\label{sec:introduction}
%==============================================================================


%==============================================================================
% # INTRODUCTION
%==============================================================================
Chez les procaryotes, le taux de guanine et cytosine (\ac{gc}) varie de 16.5\% à
75\% dans les génomes séquencés à ce jour. Une telle amplitude soulève plusieurs
questions. Celles qui nous intéressent concernent les mécanismes responsables de
ces variations : est-ce qu'ils sont associés à la réplication de l'ADN, à sa
réparation ou à une pression de sélection sur l'usage du code génétique ? Au
cours des quinzes dernières années, il a été démontré que la recombinaison
homologue tend à augmenter localement le taux de GC dans les régions fortement
recombinantes des génomes
eucaryotes\cite{duret_biased_2009,lesecque_gc-biased_2013,williams_non-crossover_2015}.
Lors de la réparation des cassures doubles brins par recombinaison homologue,
les mésappariemments locaux sont réparés par un mécanisme occasionnant de la
\emph{conversion génique}\cite{chen_gene_2007}. Ce mécanisme est biaisé vers
l'introduction préférentielle des bases C et G chez les mammifères et
probablement chez un grand nombre d'eucaryotes\cite{pessia_evidence_2012}. C'est
un processus non adaptatif : il impacte les régions codantes et non-codantes. Il
peut s'opposer à l'action de la sélection en augmentant la probabilité de
fixation des allèles G et C\cite{ratnakumar_detecting_2010}.

En 2010, deux études simultanées\cite{hildebrand_evidence_2010,
  hershberg_evidence_2010} démontrent que : 1) le patron de \emph{mutation} est
universellement biaisé vers A et T chez les procaryotes, et 2) les bases G et C
ont une probabilité de fixation plus élevée, probablement sous l'effet d'un
processus à l'action semblable à celle de la sélection naturelle. Hildebrand
\emph{et al} avancent que le \ac{gc} est en soi un trait soumis à la sélection,
et rejettent l'hypothèse d'un biais de conversion génique biaisé vers GC chez
les procaryotes. Cependant, les analyses récentes de Lassalle \emph{et al}
\cite{lassalle_gc-content_2015} et de Yahara \emph{et
  al}\cite{yahara_landscape_2016} ont montré que les zones recombinantes des
génomes procaryotes ont un taux de GC plus élévé que les régions
non-recombinantes, une observation compatible avec l'hypothèse du biais de
conversion biaisé vers GC (\ac{gbgc}). Elles montrent également que la fixation
préférentielle des allèles GC va à l'encontre de la fixation des allèles
optimaux des codons, une signature caractéristique de l'action d'un processus
non adaptatif tel que le \ac{gbgc}.


Ces observations sont compatibles avec l'hypothèse d'une conversion génique
biaisée vers GC chez les procaryotes. L'objectif de ce travail était d'obtenir
une estimation expérimentale des fréquences de conversion vers GC chez la
bactérie modèle de transformation naturelle \emph{Acinetobacter baylyi
  ADP1}\footnote{\emph{A.baylyi} était appelée \emph{A.calcoaceticus} avant
  1995, mais la taxonomie du genre a fait l'objet d'une
  révision\cite{euzeby_list_1997}}. Les cellules d'\emph{A.baylyi} en phase
exponentielle de croissance sont naturellement compétentes à la transformation.
Elles intègrent activement de l'ADN exogène sous forme simple brin par un
mécanisme sous contrôle génétique. Cet ADN peut être intégré dans le génome par
recombinaison homologue. Nous avons utilisé ces propriétés pour forcer la
recombinaison homologue à un locus neutre, en introduisant artificiellement des
mésappariemments entre la séquence donneuse --- notre construction --- et la
séquence receveuse --- le génome (voir figure \ref{img:construct}). La
réparation de ces mésappariemments donne lieu à la conversion génique d'un brin
par l'autre : l'information génétique portée par un brin est transmise
unidirectionnellement du brin donneur au brin receveur. On distingue donc la
région convertie, dont le génotype correspond à celui de l'haplotype donneur, de
la région conservée, dont le génotype correspond à celui de l'haplotype
receveur. Elles sont séparées par le point de recombinaison (voir figure
\ref{img:construct}). Les constructions présentent des mésappariemments avec la
séquence receveuse. Ces marqueurs permettent 1) de localiser la position du
point de recombinaison et 2) de déterminer la polarité de la conversion. À un
site donné, une base peut être convertie ou restaurée lorsqu'elle conserve la
base de la séquence receveuse. La réparation des mésappariemments introduits
nous renseigne sur les fréquences de conversion en faveur de GC chez
\emph{A.baylyi}.

\todo[inline]{insérer débuts de conclusions ici}

\newpage
% TODO compléter en fonction des résultats.
% TODO dire pourquoi acinetobacter
% TODO dire pourquoi trente snp.