\section*{Introduction}
\addcontentsline{toc}{section}{Introduction}
\label{sec:introduction}

Le contenu en base est un paramètre clé de variation des séquences au sein des
génomes et entre génomes. Chez les procaryotes, le taux de guanine et cytosine
(\ac{gc}) varie de 16.5\% à 75\% dans les génomes séquencés à ce jour. Une telle
amplitude soulève plusieurs questions, notamment concernant le ou les mécanismes
responsables de ces variations : est-ce qu'ils sont associés à la réplication de
l'ADN, à sa réparation ou à une pression de sélection sur l'usage du code
génétique ?

Au cours des quinzes dernières années, il a été démontré que la recombinaison
homologue tend à augmenter localement le taux de GC dans les régions fortement
recombinantes des génomes
eucaryotes\cite{duret_biased_2009,lesecque_gc-biased_2013,williams_non-crossover_2015}.
Au cours de la réparation des cassures doubles brins par recombinaison
homologue, les mésappariemments locaux sont réparés par un mécanisme
occasionnant de la \emph{conversion génique}\cite{chen_gene_2007}. Ce mécanisme
est biaisé vers l'introduction préférentielle des bases C et G chez les
mammifères et probablement chez un grand nombre
d'eucaryotes\cite{pessia_evidence_2012}. C'est un processus neutre : il impacte
les régions codantes et non-codantes. Il peut s'opposer à l'action de la
sélection en augmentant la probabilité de fixation des allèles G et
C\cite{ratnakumar_detecting_2010}.

En 2010, deux études simultanées\cite{hildebrand_evidence_2010,
  hershberg_evidence_2010} démontrent que : 1) la \emph{mutation} est
universellement biaisée vers A et T chez les procaryotes, et 2) les bases G et C
ont une probabilité de fixation plus élevée, probablement sous l'effet d'un
processus à l'action semblable à celle de la sélection naturelle. Hildebrand
\emph{et al} avancent que le \ac{gc} est en soi un trait soumis à la Sélection,
et rejettent l'hypothèse d'un biais de conversion génique biaisé vers GC chez
les procaryotes. Cependant, les analyses récentes de Lassalle \emph{et al}
\cite{lassalle_gc-content_2015} et de Yahara \emph{et
  al}\cite{yahara_landscape_2016} ont montré que les zones recombinantes des
génomes procaryotes ont un taux de GC plus élévé que les régions
non-recombinantes, une observation compatible avec l'hypothèse du biais de
conversion biaisé vers GC (\ac{gbgc}). Elles montrent également que la fixation
préférentielle des allèles GC va à l'encontre de la fixation des allèles
optimaux des codons, une signature caractéristique de l'action d'un processus
non adaptatif tel que \ac{gbgc}.

\lipsum