\newpage
\pagenumbering{roman}
\setcounter{page}{1}

\begin{multicols}{2}
\setstretch{1.0}
\rmfamily
\footnotesize
\bibliography{references}
\end{multicols}

\newpage
% \setcounter{listoffigures}{1}

\setcounter{section}{0}
\section{Annexes}
\setcounter{subsection}{1}


\subsection{Amorces utilisées}
\label{subsec:amorces}

\begin{figure}[h]
\begin{center}

  \setstretch{1.0}
  % \fontspec{Gill Sans}
  \rmfamily

  \tikzset{>=Latex}
  \tikzset{amorce fleche/.style={arrows={-Latex[scale = 0.5]}, thin}}
  \tikzset{amorce/.style={font=\tiny, above, midway}}

\begin{tikzpicture}[line width = 2pt, scale = 0.5, join = round]
  % \draw[opacity = 0.2, line width = 0.01pt, Gray] (0,0) grid (15, 15);
  % \draw[opacity = 0.2, line width = 0.005pt, step = 0.5, Gray] (0,0) grid (15, -15);
  % \draw (0,)
  \draw[genome_col, thin] [->] (0, 0) -- (20, 0);

  \draw[gs_don_col, fill=white, thick]
  (4, -0.5) rectangle ++(4, 1)
  node[font=\tiny, pos=0.5, align = left] {Gène\\Synthétique};

  \draw[kanr_col, fill=white, thick]
  (8.5, -0.5) rectangle ++(4, 1)
  node[font=\tiny, pos=0.5, align = left] {KanR};

  \draw[ancre_don_col, fill=white, thick]
  (13, -0.5) rectangle ++(4, 1)
  node[font=\tiny, pos=0.5, align = left] {Ancre};

  \draw[amorce fleche] (1, 0.7) -- ++(1, 0) node[amorce] {1073}; % amorce 1073
  \draw[amorce fleche]  (4,    +0.7) -- ++(1, 0) node[amorce] {1393};
  \draw[amorce fleche]  (8.5,  -0.7) -- ++(-1, 0) node[amorce, below] {1392};
  \draw[amorce fleche]  (8.0,  +0.7) -- ++(+1, 0) node[amorce] {1408};
  \draw[amorce fleche]  (13.0, -0.7) -- ++(-1, 0) node[amorce, below] {1409};
  \draw[amorce fleche]  (12.5, +0.7) -- ++(+1, 0) node[amorce] {1410};
  \draw[amorce fleche]  (17.5, -0.7) -- ++(-1, 0) node[amorce, below] {1411};
\end{tikzpicture}

\vspace{0.5cm}

\begin{tabular}{cc>{\fontspec{Gill Sans} \scriptsize}ll}
  \toprule
  % \rowcolor{LightGray}
  Cible                     & Amorce & \thead{\rmfamily \normalsize Séquence} & Tm (\si{\celsius}) \\
  \midrule
  Génome d'\emph{A.~baylyi} & 1073   & CAGGCTGACGTGATTGTTCA                   & 56.6               \\
  % \midrule
  Gène Synthétique 3'       & 1392   & AAGGTGGAAGAGAAGGAGGC                   & 58.7               \\
  Gène Synthétique 5'       & 1393   & GCGAGGAGGAAAGCAAAGAG                   & 58.2               \\
  \emph{aphA}3, KanR 5'     & 1408   & CACCTTAATCACTAGTTAGACATCTAAATCTAGGTAC  & 61.50              \\
  \emph{aphA}3, KanR 3'     & 1409   & GGTAAAGTCAGAGGAGAGGATGAGGAGGCAGATTG    & 68.66              \\
  Région ancre, 5'          & 1410   & TCCTCTGACTTTACCAACAAC                  & 48.04              \\
  Région ancre, 3'          & 1411   & AGGCGGCCGCACTAGCTTTCTGAGGGGAACGATCA    & 71.62              \\
  Plasmide pGEM-T           & M13R   & GAGGAAACAGCTATGAC                      & 47.8               \\
  Plasmide pGEM-T           & M13F   & GTAAAACGACGGCCAGT                      & 53.9               \\
  \bottomrule
  %
\end{tabular}


\end{center}

\caption[Liste des amorces utilisées]{\label{img:amorces}\textbf{Liste des
    amorces utilisées}}
%% TODO completer la liste des amorces.
\end{figure}

\subsection{Carte des constructions donneuses}
\label{subsec:cartes-plasmides}

% TODO carte des plasmides

\vfill

% ==============================================================================
\subsection{Traces de conversions}
\label{sec:trac-de-conv}
% ==============================================================================

\begin{figure}[p]% will be the left-side figure
  \begin{leftfullpage}
    \includegraphics[width = 0.65\textwidth]{img/trace_w.pdf}
    \caption[Zones de recombinaison des clones transformés par la construction
    AT]{\label{fig:tracew}\textbf{Zones de recombinaisons des clones transformés
        par la construction AT}. \\
      \rmfamily Lorsqu'on introduit uniquement des
      polymorphismes AT, on observe quatre cas indépendants de restauration du
      génotype sauvage, et un cinquième pour une base de qualité inférieure. }
  \end{leftfullpage}
\end{figure}
\begin{figure}[p]% will be the right-side figure
  \begin{fullpage}
    \includegraphics[width = 0.65\textwidth]{img/trace_s.pdf}
    \caption[Zones de recombinaisons des clones transformés par la construction
    GC]{\label{fig:traces} \textbf{Zones de recombinaisons des clones
        transformés par la construction GC}. \\
      \rmfamily Lorsqu'on introduit uniquement des polymorphismes GC, on n'observe pas de
      cas de restauration du génotype sauvage. }
  \end{fullpage}
\end{figure}
\begin{figure}[p]% will be the left-side figure
  \begin{leftfullpage}
    \includegraphics[width = 0.65\textwidth]{img/trace_sw.pdf}
    \caption[Zones de recombinaisons des clones transformés par la construction
    AT/GC]{\label{fig:tracesw} }
  \end{leftfullpage}
\end{figure}

% ==============================================================================
\subsection{Confirmations des restaurations des haplotypes sauvages}
\label{subsec:confirm-haplotype}


\begin{figure}[ht]
  \centering
  \includegraphics[width = \textwidth]{img/confirmation_altern.pdf}
  \caption[Confirmation des restaurations]{\label{fig:confirm-haplotype}
    \textbf{Confirmation d'une restauration d'haplotype sauvage} \\
    \rmfamily Pour s'affranchir d'une possible erreur de séquençage sur les
    sites montrant des restaurations de l'haplotype sauvage dans la région convertie, nous avons
    sequencé la zone de recombinaison de 31 clones issus du clone séquencé en
    premier lieu. Nous avons également séquencé à nouveau la colonie ``mère'',
    de façon à comparer les sous-populations avec la population mère séquencée ;
    c'est le clone séquencé \num{32}.
    \\
    Les 31 clones séquencés montrent tous la même alternance au même marqueur.
    Nous avons ainsi confirmé que cette restauration de l'haplotype sauvage
    était bien due à une correction des mésappariemments dans la population
    mère.
    %
  }
\end{figure}

% ==============================================================================

%% DONE insérer reste des traces de conversion graphiques
%% TODO annoter les traces de conversion de la même façon que la figure 2.1