\newpage
\pagenumbering{roman}
\setcounter{page}{1}

\begin{multicols}{2}
\setstretch{1.0}
\rmfamily
\tiny
\bibliography{references}
\end{multicols}
\newpage


% ------------------------------------------------------------------------------
\setcounter{section}{0}
\section{Annexes}
% ------------------------------------------------------------------------------


% ------------------------------------------------------------------------------
\setcounter{subsection}{1}
\subsection{Amorces utilisées}
\label{subsec:annexe-amorces}
% ------------------------------------------------------------------------------

\begin{center}

  \setstretch{1.0}
  % \fontspec{Gill Sans}
  \rmfamily

  \tikzset{>=Latex}
  \tikzset{amorce fleche/.style={arrows={-Latex[scale = 0.5]}, thin}}
  \tikzset{amorce/.style={font=\tiny, above, midway}}

\begin{tikzpicture}[line width = 2pt, scale = 0.5, join = round]
  % \draw[opacity = 0.2, line width = 0.01pt, Gray] (0,0) grid (15, 15);
  % \draw[opacity = 0.2, line width = 0.005pt, step = 0.5, Gray] (0,0) grid (15, -15);
  % \draw (0,)
  \draw[genome_col, thin] [->] (0, 0) -- (20, 0);

  \draw[gs_don_col, fill=white, thick]
  (4, -0.5) rectangle ++(4, 1)
  node[font=\tiny, pos=0.5, align = left] {Gène\\Synthétique};

  \draw[kanr_col, fill=white, thick]
  (8.5, -0.5) rectangle ++(4, 1)
  node[font=\tiny, pos=0.5, align = left] {KanR};

  \draw[ancre_don_col, fill=white, thick]
  (13, -0.5) rectangle ++(4, 1)
  node[font=\tiny, pos=0.5, align = left] {Ancre};

  \draw[amorce fleche] (1, 0.7) -- ++(1, 0) node[amorce] {1073}; % amorce 1073
  \draw[amorce fleche]  (4,    +0.7) -- ++(1, 0) node[amorce] {1393};
  \draw[amorce fleche]  (8.5,  -0.7) -- ++(-1, 0) node[amorce, below] {1392};
  \draw[amorce fleche]  (8.0,  +0.7) -- ++(+1, 0) node[amorce] {1408};
  \draw[amorce fleche]  (13.0, -0.7) -- ++(-1, 0) node[amorce, below] {1409};
  \draw[amorce fleche]  (12.5, +0.7) -- ++(+1, 0) node[amorce] {1410};
  \draw[amorce fleche]  (17.5, -0.7) -- ++(-1, 0) node[amorce, below] {1411};
\end{tikzpicture}

\vspace{0.5cm}

\begin{tabular}{cc>{\fontspec{Gill Sans} \scriptsize}ll}
  \toprule
  % \rowcolor{LightGray}
  Cible                     & Amorce & \thead{\rmfamily \normalsize Séquence} & Tm (\si{\celsius}) \\
  \midrule
  Génome d'\emph{A.~baylyi} & 1073   & CAGGCTGACGTGATTGTTCA                   & 56.6               \\
  % \midrule
  Gène Synthétique 3'       & 1392   & AAGGTGGAAGAGAAGGAGGC                   & 58.7               \\
  Gène Synthétique 5'       & 1393   & GCGAGGAGGAAAGCAAAGAG                   & 58.2               \\
  \emph{aphA}3, KanR 5'     & 1408   & CACCTTAATCACTAGTTAGACATCTAAATCTAGGTAC  & 61.50              \\
  \emph{aphA}3, KanR 3'     & 1409   & GGTAAAGTCAGAGGAGAGGATGAGGAGGCAGATTG    & 68.66              \\
  Région ancre, 5'          & 1410   & TCCTCTGACTTTACCAACAAC                  & 48.04              \\
  Région ancre, 3'          & 1411   & AGGCGGCCGCACTAGCTTTCTGAGGGGAACGATCA    & 71.62              \\
  Plasmide pGEM-T           & M13R   & GAGGAAACAGCTATGAC                      & 47.8               \\
  Plasmide pGEM-T           & M13F   & GTAAAACGACGGCCAGT                      & 53.9               \\
  \bottomrule
  %
\end{tabular}


\end{center}

\caption[Liste des amorces utilisées]{\label{fig:amorces}\textbf{Liste des
    amorces utilisées}}

% ------------------------------------------------------------------------------
\subsection{Conditions de PCR}
\label{subsec:annexe-pcr}
% ------------------------------------------------------------------------------

\begin{center}
  \rmfamily
  \setstretch{1.0}
  \begin{tabular}{rcc}
    \toprule
                                   & Volume pour \SI{50}{\uL} (\si{\uL}) & Concentration finale             \\
    \midrule
    \ce{H_2O}                      & qsp \SI{50}{\uL}                    &                                  \\
    5X Phusion Green Buffer        & \num{10}                            & 1X                               \\
    \SI{10}{\mmol\per\liter} dNTPs & \num{1}                             & \SI{200}{\umol\per\liter} chacun \\
    Amorce sens                    & \num{2.5}                           & \SI{0.5}{\umol\per\liter}        \\
    Amorce antisens                & \num{2.5}                           & \SI{0.5}{\umol\per\liter}        \\
    ADN matrice                    & X \si{\uL}                          &                                  \\
    Phusion DNA Polymerase         & 0.5                                 & \SI{0.02}{U\per\uL}              \\
    \bottomrule
  \end{tabular}
  \\
  Programmation du thermocycleur :\\
  \includegraphics[width = 0.5\textwidth]{img/pcr_protocole.png}
\end{center}

\subsection{Carte des constructions donneuses}
\label{subsec:cartes-plasmides}

\begin{figure}[h!]
  \centering
  \includegraphics[width = 0.4\textwidth]{img/plasmid_map.png}
  \caption[Carte des constructions donneuses]{Carte des constructions donneuses}
  \label{fig:plasmid-map}
\end{figure}


% ==============================================================================

\subsection{Confirmations des restaurations des haplotypes sauvages}
\label{subsec:confirm-haplotype}

\begin{figure}[h!]
  \centering
  \includegraphics[width = 0.5\textwidth]{img/confirmation_altern.pdf}
  \caption[Confirmation des restaurations]{\label{fig:confirm-haplotype}
    \textbf{Confirmation d'une restauration d'haplotype sauvage} \\
    \rmfamily Pour s'affranchir d'une possible erreur de séquençage sur les
    sites montrant des restaurations de l'haplotype sauvage dans la région
    convertie, nous avons sequencé la zone de recombinaison de 31 clones issus
    du clone séquencé en premier lieu. Nous avons également séquencé à nouveau
    la colonie ``mère'', de façon à comparer les sous-populations avec la
    population mère séquencée ; c'est le clone séquencé \num{32}. Pour rappel,
    les points rouges correspondent à l'haplotype du receveur, les points bleus
    à l'haplotype du donneur. Les 31 clones séquencés montrent tous la même
    alternance au même marqueur que le clone 32. Nous avons ainsi confirmé que
    cette restauration de l'haplotype sauvage était bien due à une correction
    des mésappariemments dans la population mère. }
\end{figure}

\subsection{Régions de recombinaisons}
\label{subsec:trac-de-conv}
% ==============================================================================

La figure \ref{fig:trace-4} page \ref{fig:trace-4} montre l'alignement de toutes
les positions de marqueurs lorsque le donneur est GC, AT, GC/AT et AT/GC.
L'interprétation des figures est détaillée dans la figure \ref{fig:convtract} en
page \pageref{fig:convtract}.

\afterpage{%
  \clearpage
  \KOMAoptions{ pagesize, paper=landscape}
  \recalctypearea
  \begin{figure}
    \centering
    \includegraphics[height = \textheight, width = \textwidth]{img/trace_4.pdf}
    \caption%
    [Régions converties]%
    {\textbf{Régions converties pour les quatre types de construction donneuse}
      \\%
      \rmfamily Les figures \textsf{a}, \textsf{b}, \textsf{c}, \textsf{d}
      représentent les régions recombinantes des transformants obtenus avec les
      constructions donneuses respectivement dGC, dAT, dGC/AT et dAT/CG.  }%
    \label{fig:trace-4}
  \end{figure}
  \clearpage
  \KOMAoptions{paper=A4, paper=portrait, pagesize}
  \recalctypearea
  \newpage
  \blankpage
  \newpage
  % \blankpage
\null
\vfill

\section*{\large \centering Étude expérimentale des fréquences de conversion en faveur de GC au
  cours de la recombinaison homologue chez \emph{Acinetobacter baylyi}}
\thispagestyle{empty}

La réparation de l'ADN joue un rôle important dans l'évolution de la composition
en base des génomes. Chez les mammifères, la recombinaison homologue tend à
augmenter le taux de GC des régions recombinantes par un processus appelé
conversion génique biaisée (gBGC). L'hypothèse gBGC a été récemment étendue aux
procaryotes. L'objectif de ce travail était de valider une approche
expérimentale permettant de quantifier les fréquences de conversion en faveur de
GC chez \emph{Acinetobacter}. Nous avons transformé cette bactérie naturellement
compétente par des gènes de synthèse introduisant des sites variants avec un
locus génomique, sélectionné les recombinants et séquencé les produits de
recombinaison. Nous avons pu démontré la validité de l'approche expérimentale,
en obtenant des fréquences de transformation de l'ordre de \num{1e-5} et une
moyenne de \num{3000} transformants par construction donneuse. Le séquençage de
\num{384} recombinants a servi à décrire les régions recombinantes, la
distribution des changements d'haplotype entre donneur et receveur, et à
quantifier le nombre d'évènements de conversion dans le sens AT$\rightarrow$GC.
Les régions converties obtenues démontrent que les bases AT ont une probabilité
plus forte d'être convertis en GC que l'inverse, une observation qui correspond
précisément à l'hypothèse initiale. Ce résultat représente une seconde étape
importante dans l'extension du biais de conversion génique vers GC aux
procaryotes. Il vient appuyer la proposition du gBGC comme l'une des forces
permettant aux procaryotes de maintenir des taux de GC élevés alors que la
mutation est biaisée vers AT.

\vfill
}


% ==============================================================================
