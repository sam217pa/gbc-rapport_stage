\newpage
\pagenumbering{roman}
\setcounter{page}{1}

\begin{multicols}{2}
\setstretch{1.0}
\rmfamily
\footnotesize
\bibliography{references}
\end{multicols}
\newpage
% \afterpage{\blankpage}

% \newpage
% \setcounter{listoffigures}{1}

\setcounter{section}{0}
\section{Annexes}
\setcounter{subsection}{1}


\subsection{Amorces utilisées}
\label{subsec:amorces}

\begin{center}

  \setstretch{1.0}
  % \fontspec{Gill Sans}
  \rmfamily

  \tikzset{>=Latex}
  \tikzset{amorce fleche/.style={arrows={-Latex[scale = 0.5]}, thin}}
  \tikzset{amorce/.style={font=\tiny, above, midway}}

\begin{tikzpicture}[line width = 2pt, scale = 0.5, join = round]
  % \draw[opacity = 0.2, line width = 0.01pt, Gray] (0,0) grid (15, 15);
  % \draw[opacity = 0.2, line width = 0.005pt, step = 0.5, Gray] (0,0) grid (15, -15);
  % \draw (0,)
  \draw[genome_col, thin] [->] (0, 0) -- (20, 0);

  \draw[gs_don_col, fill=white, thick]
  (4, -0.5) rectangle ++(4, 1)
  node[font=\tiny, pos=0.5, align = left] {Gène\\Synthétique};

  \draw[kanr_col, fill=white, thick]
  (8.5, -0.5) rectangle ++(4, 1)
  node[font=\tiny, pos=0.5, align = left] {KanR};

  \draw[ancre_don_col, fill=white, thick]
  (13, -0.5) rectangle ++(4, 1)
  node[font=\tiny, pos=0.5, align = left] {Ancre};

  \draw[amorce fleche] (1, 0.7) -- ++(1, 0) node[amorce] {1073}; % amorce 1073
  \draw[amorce fleche]  (4,    +0.7) -- ++(1, 0) node[amorce] {1393};
  \draw[amorce fleche]  (8.5,  -0.7) -- ++(-1, 0) node[amorce, below] {1392};
  \draw[amorce fleche]  (8.0,  +0.7) -- ++(+1, 0) node[amorce] {1408};
  \draw[amorce fleche]  (13.0, -0.7) -- ++(-1, 0) node[amorce, below] {1409};
  \draw[amorce fleche]  (12.5, +0.7) -- ++(+1, 0) node[amorce] {1410};
  \draw[amorce fleche]  (17.5, -0.7) -- ++(-1, 0) node[amorce, below] {1411};
\end{tikzpicture}

\vspace{0.5cm}

\begin{tabular}{cc>{\fontspec{Gill Sans} \scriptsize}ll}
  \toprule
  % \rowcolor{LightGray}
  Cible                     & Amorce & \thead{\rmfamily \normalsize Séquence} & Tm (\si{\celsius}) \\
  \midrule
  Génome d'\emph{A.~baylyi} & 1073   & CAGGCTGACGTGATTGTTCA                   & 56.6               \\
  % \midrule
  Gène Synthétique 3'       & 1392   & AAGGTGGAAGAGAAGGAGGC                   & 58.7               \\
  Gène Synthétique 5'       & 1393   & GCGAGGAGGAAAGCAAAGAG                   & 58.2               \\
  \emph{aphA}3, KanR 5'     & 1408   & CACCTTAATCACTAGTTAGACATCTAAATCTAGGTAC  & 61.50              \\
  \emph{aphA}3, KanR 3'     & 1409   & GGTAAAGTCAGAGGAGAGGATGAGGAGGCAGATTG    & 68.66              \\
  Région ancre, 5'          & 1410   & TCCTCTGACTTTACCAACAAC                  & 48.04              \\
  Région ancre, 3'          & 1411   & AGGCGGCCGCACTAGCTTTCTGAGGGGAACGATCA    & 71.62              \\
  Plasmide pGEM-T           & M13R   & GAGGAAACAGCTATGAC                      & 47.8               \\
  Plasmide pGEM-T           & M13F   & GTAAAACGACGGCCAGT                      & 53.9               \\
  \bottomrule
  %
\end{tabular}


\end{center}

\caption[Liste des amorces utilisées]{\label{fig:amorces}\textbf{Liste des
    amorces utilisées}}
%% DONE completer la liste des amorces.

\subsection{Carte des constructions donneuses}
\label{subsec:cartes-plasmides}

% TODO carte des plasmides

% \vfill

% ==============================================================================
\subsection{Traces de conversions}
\label{subsec:trac-de-conv}
% ==============================================================================

La figure \ref{fig:trace-4} page \ref{fig:trace-4} montre l'alignement de toutes
les positions de marqueurs lorsque le donneur est CG, AT, CG/AT et AT/CG.
L'interprétation des figures est détaillée dans la figure \ref{fig:convtract} en
page \pageref{fig:convtract}.

% \includepdf[noautoscale, landscape = true]{img/trace_4.pdf}

\afterpage{%
  \clearpage
  \KOMAoptions{ pagesize, paper=landscape}
  \recalctypearea
  \begin{figure}
    \includegraphics[height = \textheight, width = \textwidth]{img/trace_4.pdf}
    \caption%
    [Régions converties]%
    {\label{fig:trace-4} \rmfamily Pellentesque dapibus suscipit ligula. Donec
      posuere augue in quam. Aliquam feugiat tellus ut neque. Nulla facilisis,
      risus a rhoncus fermentum, tellus tellus lacinia purus, et dictum nunc
      justo sit amet elit. }
    \centering
  \end{figure}
  \clearpage
  \KOMAoptions{paper=A4, paper=portrait, pagesize}
  \recalctypearea
}

\subsection{Confirmations des restaurations des haplotypes sauvages}
\label{subsec:confirm-haplotype}

\begin{figure}[ht]
  \centering
  \includegraphics[width = 0.6\textwidth]{img/confirmation_altern.pdf}
  \caption[Confirmation des restaurations]{\label{fig:confirm-haplotype}
    \textbf{Confirmation d'une restauration d'haplotype sauvage} \\
    \rmfamily Pour s'affranchir d'une possible erreur de séquençage sur les
    sites montrant des restaurations de l'haplotype sauvage dans la région convertie, nous avons
    sequencé la zone de recombinaison de 31 clones issus du clone séquencé en
    premier lieu. Nous avons également séquencé à nouveau la colonie ``mère'',
    de façon à comparer les sous-populations avec la population mère séquencée ;
    c'est le clone séquencé \num{32}.
    \\
    Les 31 clones séquencés montrent tous la même alternance au même marqueur
    que le clone 32.
    Nous avons ainsi confirmé que cette restauration de l'haplotype sauvage
    était bien due à une correction des mésappariemments dans la population
    mère.
    %
  }
\end{figure}

% ==============================================================================

%% DONE insérer reste des traces de conversion graphiques
%% DONE annoter les traces de conversion de la même façon que la figure 2.1
% TODO insérer protocoles des PCRs