\begin{center}
  \rmfamily
  \setstretch{1.0}

  \tikzset{
    culture/.pic={
      \node at (0, 0) {\includegraphics[width = 0.015\textwidth]{img/lb_liquide.png}};
    }
  }

  \tikzset{
    transfo/.pic={
      \node at (0, 0) {\includegraphics[width = 0.03\textwidth]{img/eppendorf_full.png}};
    }
  }

  \tikzset{
    etalement/.pic={
      \node at (0, 0) {\includegraphics[width = 0.05\textwidth]{img/etalement.png}};
    }
  }

  \tikzset{
    petri/.pic={
      \node at (0, 0) {\includegraphics[width = 0.05\textwidth]{img/petri_open.png}};
    }
  }

  \tikzset{
    boite/.pic={
      \node at (0, 0) {\includegraphics[width = 0.06\textwidth]{img/boite_96.png}};
    }
  }

  \tikzset{
    pcr plate/.pic={
      \node at (0, 0) {\includegraphics[width = 0.05\textwidth]{img/pcr_plate.jpeg}};
    }
  }

  \tikzset{legende fleche/.style={Gray, dotted, thick, opacity = 0.6}}
  \tikzset{legende text/.style={black, text width = 7cm, font = \scriptsize, above}}

  \begin{tikzpicture}

    % tube eppendorf seul
    \draw [->]
    (0, 0) node[above]
    {\includegraphics[width=0.05\textwidth]{img/eppendorf.png}}
    % boite de petri
    -- (0, -1) node[below]
    {\includegraphics[width=0.2\textwidth]{img/petri_open.png}}
    ;

    %% tubes de culture
    \draw[out = 210, in = 90] [->] (0, -3) to ++(-3.5, -1) pic[below] {culture};
    \draw[out = 210, in = 90] [->] (0, -3) to ++(-2.5, -1) pic[below] {culture};
    \draw[out = -90, in = 90] [->] (0, -3) to ++(-0.5, -1) pic[below] {culture};
    \draw[out = -90, in = 90] [->] (0, -3) to ++(+0.5, -1) pic[below] {culture};
    \draw[out = -30, in = 90] [->] (0, -3) to ++(+2.5, -1) pic[below] {culture};
    \draw[out = -30, in = 90] [->] (0, -3) to ++(+3.5, -1) pic[below] {culture};

    %% tubes de transfo
    \draw [->] (-3.5, -7) node[above, font = \scriptsize] {GC} -- ++(0, -0.7) pic[below] {transfo};
    \draw [->] (-2.5, -7) node[above, font = \scriptsize] {GC} -- ++(0, -0.7) pic[below] {transfo};
    \draw [->] (-0.5, -7) node[above, font = \scriptsize] {AT} -- ++(0, -0.7) pic[below] {transfo};
    \draw [->] (+0.5, -7) node[above, font = \scriptsize] {AT} -- ++(0, -0.7) pic[below] {transfo};
    \draw [->] (+2.5, -7) node[above, font = \scriptsize] {\(\nicefrac{GC}{AT}\)} -- ++(0, -0.7) pic[below] {transfo};
    \draw [->] (+3.5, -7) node[above, font = \scriptsize] {\(\nicefrac{AT}{GC}\)} -- ++(0, -0.7) pic[below] {transfo};

    %% étalement sur boiîtes de pétri
    \draw [->] (-3.5, -9) -- ++(0, -0.5) pic[below] {petri};
    \draw [->] (-2.5, -9) -- ++(0, -0.5) pic[below] {petri};
    \draw [->] (-0.5, -9) -- ++(0, -0.5) pic[below] {petri};
    \draw [->] (+0.5, -9) -- ++(0, -0.5) pic[below] {petri};
    \draw [->] (+2.5, -9) -- ++(0, -0.5) pic[below] {petri};
    \draw [->] (+3.5, -9) -- ++(0, -0.5) pic[below] {petri};

    %% purification sur boîtes
    \draw [->] (-3.5, -10.5) -- ++(0, -0.5) pic[below] {etalement};
    \draw [->] (-2.5, -10.5) -- ++(0, -0.5) pic[below] {etalement};
    \draw [->] (-0.5, -10.5) -- ++(0, -0.5) pic[below] {etalement};
    \draw [->] (+0.5, -10.5) -- ++(0, -0.5) pic[below] {etalement};
    \draw [->] (+2.5, -10.5) -- ++(0, -0.5) pic[below] {etalement};
    \draw [->] (+3.5, -10.5) -- ++(0, -0.5) pic[below] {etalement};

    %% suspension dans 50µL d'eau
    \draw [->] (-3.5, -12.2) -- ++(0, -0.5) pic[below] {boite};
    \draw [->] (-2.5, -12.2) -- ++(0, -0.5) pic[below] {boite};
    \draw [->] (-0.5, -12.2) -- ++(0, -0.5) pic[below] {boite};
    \draw [->] (+0.5, -12.2) -- ++(0, -0.5) pic[below] {boite};
    \draw [->] (+2.5, -12.2) -- ++(0, -0.5) pic[below] {boite};
    \draw [->] (+3.5, -12.2) -- ++(0, -0.5) pic[below] {boite};

    %% PCR
    \draw [->] (-3.5, -13.6) -- ++(0, -0.5) pic[below] {pcr plate};
    \draw [->] (-2.5, -13.6) -- ++(0, -0.5) pic[below] {pcr plate};
    \draw [->] (-0.5, -13.6) -- ++(0, -0.5) pic[below] {pcr plate};
    \draw [->] (+0.5, -13.6) -- ++(0, -0.5) pic[below] {pcr plate};
    \draw [->] (+2.5, -13.6) -- ++(0, -0.5) pic[below] {pcr plate};
    \draw [->] (+3.5, -13.6) -- ++(0, -0.5) pic[below] {pcr plate};

    %% gels électrophorèse
    \draw [->] (-3.5, -15) -- ++(0, -0.5) node[below] {\includegraphics[width = 0.06\textwidth]{img/s1_plate.jpg}};
    \draw [->] (-2.5, -15) -- ++(0, -0.5) node[below] {\includegraphics[width = 0.06\textwidth]{img/s3_plate.jpg}};
    \draw [->] (-0.5, -15) -- ++(0, -0.5) node[below] {\includegraphics[width = 0.06\textwidth]{img/w1_plate.jpg}};
    \draw [->] (+0.5, -15) -- ++(0, -0.5) node[below] {\includegraphics[width = 0.06\textwidth]{img/w2_plate.jpg}};
    \draw [->] (+2.5, -15) -- ++(0, -0.5) node[below] {\includegraphics[width = 0.06\textwidth]{img/sw_plate.jpg}};
    \draw [->] (+3.5, -15) -- ++(0, -0.5) node[below] {\includegraphics[width = 0.06\textwidth]{img/ws_plate.jpg}};

    %%
    %% ANNOTATIONS
    %%

    % %% trait de légende
    \draw[legende fleche] [<-] (1.5, -1.5) -- (2, -1.5) -- (3, 0) -- (8, 0)
    node[legende text] {%
      Une culture cryogénisée d'\emph{Acinetobacter baylyi BD413} est purifiée
      par étalement sur milieu non sélectif de Luria Bertani.
      %
    } node {$\bullet$} ;

    \draw[legende fleche] [<-] (4, -4.5) -- (8, -4.5) node[legende text] {%
      Une colonie isolée est pré-cultivée pendant \si{24\hour} à \si{30\celsius}
      en milieu LB. \si{50\uL} de cette pré-culture sont resuspensdus dans
      \si{5\mL} de LB.
      %
    } node {$\bullet$} ;

    \draw[legende fleche] [<-] (4, -8.0) -- (8, -8.0) node[legende text] {%
      Lorsque la culture a atteint une absorbance à \si{600\nm} de \(0.8\),
      \si{400\ng} de construction plasmidique linéarises sont ajoutées dans
      \si{390\uL} de culture, et incubés pendant \si{1\hour} à \si{30\celsius}.

      Les transformations ont été réalisées avec trois constructions différentes
      : les constructions introduisant des G et des C, celles introduisant des A
      et des T, et celles introduisant les deux à la fois en alternance.
      %
    } node {$\bullet$} ;

    \draw[legende fleche] [<-] (4, -10.0) -- (8, -10.0) node[legende text] {%
      Après \si{1\hour} d'incubation, les plasmides résiduels sont dégradés par
      l'ajout de DNAse à \SI{20}{\ug\per\mL}, et incubation \si{15\min} à
      \si{37\celsius}. La culture est ensuite étalée en spirale sur milieu
      sélectif LB additionné de kanamycine à \SI{50}{\ug\per\mL}, et incubés
      pendant \SI{24}{\hour} à \SI{37}{\celsius}.
      %
    } node {$\bullet$} ;

    \draw[legende fleche] [<-] (4, -11.5) -- (8, -11.5) node[legende text]
    {%
      \(96\) colonies transformantes, résistantes à la kanamycines sont isolées
      par étalement sur le même milieu sélectif LB+Kan.
      %
    } node {$\bullet$} ;


    \draw[legende fleche] [<-] (4, -13.0) -- (8, -13.0) node[legende text] {%
      Pour chaque transformant isolé, 1 colonie est prélevée à l'œse et
      suspendue dans \SI{50}{\uL} d'eau ultrapure.
      %
    } node {$\bullet$} ;

    \draw[legende fleche] [<-] (4, -14.5) -- (8, -14.5) node[legende text] {%
      Des PCRs cibles de la région d'intérêt sont réalisées en utilisant
      \SI{2}{\uL} des suspensions précédemment réalisées.
      %
    } node {$\bullet$} ;

    \draw[legende fleche] [<-] (4, -16.2) -- (8, -16.2) node[legende text] {%
      Les PCRs ont été contrôlées par électrophorèse sur gel d'agarose à
      \(1\%\). Les amplicons ont été ensuite séquencés par la technique de
      Sanger.
      %
    } node {$\bullet$} ;

  \end{tikzpicture}

\end{center}