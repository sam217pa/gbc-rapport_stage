\begin{center}
  \rmfamily
  \setstretch{1.0}
  \tikzset{>=stealth}
  \tikzset{plasmid lines/.style={xscale = 0.2, yscale = 0.2, very thick}}
  \tikzset{legende/.style={font=\scriptsize, color=Gray, align=right}}
  \tikzset{legende text/.style={font=\scriptsize, color=Black, align=right}}

  \tikzset{
    plasmid/.pic={
	    \draw[plasmid_col, plasmid lines, semithick, rounded corners = 0.5ex] (3, -1) rectangle (11, 0);
	    \draw[gs_don_col, plasmid lines ] (4, 0) -- (6, 0);
	    \draw[am1_col , plasmid lines] (6,  0) -- (6.2, 0);
	    \draw[kanr_col , plasmid lines] (6.2,  0) -- (7.8, 0);
	    \draw[genome_col , plasmid lines] (7.8, 0) -- (8.0, 0);
	    \draw[ancre_don_col , plasmid lines] (8.0, 0) -- (9.8, 0);
	    \draw[am3_col , plasmid lines] (9.8, 0) -- (10.0, 0);% node[font=\tiny] {$\bullet$};
    }
  }
  \tikzset{
    construct/.pic={
        \draw[plasmid lines, gs_don_col] (4, 0) -- (6, 0);
        \draw[plasmid lines, am1_col] (6, 0) -- (6.2, 0);
        \draw[plasmid lines, am1_col] (6, 0) -- (6.2, 0);
        \draw[plasmid lines, kanr_col] (6.2, 0) -- (7.8, 0);
        \draw[plasmid lines, genome_col] (7.8, 0) -- (8.0, 0);
        \draw[plasmid lines, genome_col] (7.8, 0) -- (8.0, 0);
        \draw[plasmid lines, ancre_don_col] (8.0, 0) -- (9.8, 0);
        \draw[plasmid lines, am3_col] (9.8, 0) -- (10.0, 0);
        \draw[plasmid lines, plasmid_col, thick] (2, 0) -- (4, 0);
        \draw[plasmid lines, plasmid_col, thick] (10, 0) -- (12, 0);
    }
  }

  \tikzset{
    gene interet/.pic={
      \draw[xscale = 0.5, gs_don_col] (0, 0) -- (1, 0);
      \draw[xscale = 0.5, am1_col] (0.8, 0) -- (1, 0);
    }
  }

  \begin{tikzpicture}[scale=0.7, line width = 2pt, join=round]
    \begin{scope}[shift={(-3.5, 12)}]
      \draw[gs_don_col] (0, 0) -- (1, 0);
      \draw[am1_col] (0.8, 0) -- (1, 0);
      \draw[Gray] [->] (1.5, 0) -- (2.7, 0) node[below, midway, font=\scriptsize] {PCR};
    \end{scope}

    \begin{scope}[shift={(-0.3, 10)}]
	    % \draw[plasmid_col] (0, 0) arc (0:30:3);
	    \draw[Gray]  (1, 0.5) node {\scriptsize +};
	    \draw[plasmid_col, thick] (0, 0) arc (150:390:1);
	    \draw[plasmid_col] (1, -0.5) node {\scriptsize pGEM-T};
	    \draw[plasmid_col] (0.1, 0.3) node {\tiny \sffamily T};
	    \draw[plasmid_col] (1.7, 0.3) node {\tiny \sffamily T};
    \end{scope}


    \begin{scope}[shift={(0, 11)}, line width = 0.5 pt]
	    \foreach \y in {0,0.2,...,2} {
        \draw[gs_don_col] (0, \y) -- (1, \y);
        \draw[am1_col   ] (0.9, \y) -- (1, \y);
        \draw[gs_don_col] (-0.1, \y) node {\tiny \sffamily A};
        \draw[gs_don_col] (1.1, \y) node {\tiny \sffamily A};
	    }
	    \draw[gs_don_col, align=center] (0.5, 2.5) node[font=\tiny] {Gène\\Synthétique};
      % essaie d'aligner
    \end{scope}

    \draw[Gray] [->] (3, 12) -- (4, 12)
    node[above, midway]
    {\includegraphics[width=0.1\textwidth, angle = 90]{schema/construct-img/ligation.png}}
    node[below, midway, font=\scriptsize]
    {Ligature};

    \begin{scope}[shift={(5.0, 13)}, scale = 0.8]
	    % plasmide pGEM-T avec construction, avant infusion
	    \draw[plasmid_col, semithick] (0, 0) arc (150:390:1);
	    \draw[am3_col] (0, 0) arc (150:20:1) ;
	    \draw[am1_col] (0, 0) arc (150:30:1);
	    \draw[gs_don_col] (0, 0) arc (150:40:1) ;
    \end{scope}


    \draw[Gray] [->] (7, 12) -- ++(2, -0.5)
    node[midway, above, font=\scriptsize]
    {Digestion};

    \begin{scope}[shift={(7, 10.5)}]
	    % plasmide ouvert, avec ancre et kana
	    \draw[plasmid_col, thick] (0, 0) -- (4, 0) node[auto, above, midway] {\tiny{pGEM-T}};
	    \draw[gs_don_col] (4, 0) -- (6, 0) node [auto, above, midway] {\tiny{Gène Synthétique}};
	    \draw[am1_col] (6, 0) -- (6.2, 0);

	    \draw[am1_col] (6, -0.3) -- (6.2, -0.3);
	    \draw[kanr_col] (6.2, -0.3) -- (7.8, -0.3) node[auto, above, midway] {\tiny{KanR}};
	    \draw[genome_col] (7.8, -0.3) -- (8.0, -0.3);

	    \draw[genome_col] (7.8, 0) -- (8.0, 0);
	    \draw[ancre_don_col] (8.0, 0) -- (9.8, 0) node[auto, above, midway] {\tiny{Ancre}};
	    \draw[am3_col] (9.8, 0) -- (10.0, 0);

      \draw[plasmid_col, dotted, thin] [->] (0, 0) .. controls (3, -1) and (8, -1) .. (10.0, 0);

	    \draw[am3_col] (0, 0) -- (0.2, 0);
    \end{scope}

    \draw[Gray] [->] (14, 9.5) -- ++(0, -1) node[legende, midway, left] {Ligature \\ InFusion};

    \begin{scope}[shift={(7, 8)}]
      % Plasmide fermé avec infusion
	    \draw[plasmid_col, rounded corners, thick] (3, -0.5) rectangle (11, 0);
	    \draw[gs_don_col ] (4, 0) -- (6, 0);
	    \draw[am1_col ] (6,  0) -- (6.2, 0);
	    \draw[am1_col ] (6,  0) -- (6.2, 0);
	    \draw[kanr_col ] (6.2,  0) -- (7.8, 0);
	    \draw[genome_col ] (7.8, 0) -- (8.0, 0);
	    \draw[genome_col ] (7.8, 0) -- (8.0, 0);
	    \draw[ancre_don_col ] (8.0, 0) -- (9.8, 0);
	    \draw[am3_col ] (9.8, 0) -- (10.0, 0);
    \end{scope}

    \draw[Gray] [->] (14, 7) -- (14, 6)
    node[legende, midway, left]
    {Amplification clonale\\dans \emph{E.coli}};

    \begin{scope}[shift={(14, 4.5)}]
      % Plasmide apmplifié avec infusion

      \matrix (A)[column sep=2mm, row sep=2mm] {
        \pic{plasmid}; & \pic{plasmid}; & \pic{plasmid}; \\
        \pic{plasmid}; & \pic{plasmid}; & \pic{plasmid}; \\
        \pic{plasmid}; & \pic{plasmid}; & \pic{plasmid}; \\
        \pic{plasmid}; & \pic{plasmid}; & \pic{plasmid}; \\
      };
    % pic {plasmid};

    \end{scope}

    \draw[Gray] [->] (14, 2.75) -- ++(0, -1)
    node[legende, midway, left]
    {Linéarisation par\\digestion enzymatique};

    \begin{scope}[shift={(14, 1)}, scale = 0.4]
      \matrix (A)[column sep=2mm, row sep=2mm] {
        \pic{construct}; & \pic{construct}; & \pic{construct}; \\
        \pic{construct}; & \pic{construct}; & \pic{construct}; \\
        \pic{construct}; & \pic{construct}; & \pic{construct}; \\
        \pic{construct}; & \pic{construct}; & \pic{construct}; \\
      };
    \end{scope}

    %% TEXTE DE DESCRIPTION
    \node[text width=9cm, legende text, align = left] at (3, 5) {
      Le gène d'intérêt \tikz[baseline = -0.5ex, line width=1.3pt] {\path (0, 0)
        pic {gene interet};} est amplifié par PCR spécifique. Des bases adénines
      sont ajoutées naturellement sur les fragments au cours des derniers cycles
      de PCR. Ces fragments sont ensuite ligaturés dans le plasmide pGEM-T.
      Celui-ci est disponible ouvert avec des bases thymines sortantes. La
      ligature est catalysée par la T4DNA Ligase. Le plasmide obtenu est
      linéarisé par digestion enzymatique au site d'insertion des fragments.

      La cassette de résistance à la kanamycine \tikz[baseline = -0.5ex, very thick, scale = 0.3] {
	    \draw[am1_col] (0, 0) -- (0.2, 0);
	    \draw[kanr_col] (0.2, 0) -- (1.8, 0);
	    \draw[genome_col] (1.8, 0) -- (2, 0);
      } ainsi que la région facilitant l'homologie \tikz[baseline = -0.5ex, very thick, scale = 0.3, xshift=-7.8] {
	    \draw[genome_col] (7.8, 0) -- (8.0, 0);
	    \draw[ancre_don_col] (8.0, 0) -- (9.8, 0) ;
	    \draw[am3_col] (9.8, 0) -- (10.0, 0);
      }
      sont préalablement amplifiées par PCR en utilisant des amorces porteuses de
      régions 3' flottantes, complémentaires des extrémités des fragments
      voisins. La ligature entre les trois fragments obtenus est réalisée à
      l'aide du kit Takara InFusion. Les constructions plasmidiques obtenues
      sont amplifiées clonalement par culture en milieu sélectif liquide, et
      linéarisées par digestion enzymatique.
    };


    %% Représente le génome d'acinetobacter en rouge, avec la construction
    %% associée.
    \begin{scope}[shift={(-5, 1)}]
      % représente la construction, avec l'insertion de la cassette de résistance
      \draw[gs_don_col] (4, 0) -- (6, 0) node[above, midway, font=\tiny] {Gène cible};
      \draw[am1_col, ultra thick] (6, 0) -- (6.2, 0);
      \draw[kanr_col] (6.2, 0) -- (7.8, 0) node[above, midway, font=\tiny] {KanR};
      \draw[genome_col, ultra thick] (7.8, 0) -- (8.0, 0);
      \draw[ancre_don_col] (8.0, 0) -- (9.8, 0) node[above, midway, font=\tiny] {Ancre};
      \draw[am3_col, ultra thick] (9.8, 0) -- (10.0, 0);
      \draw[plasmid_col, thick] (2, 0) -- (4, 0);

      \draw[arrows = {-Stealth[left]}, plasmid_col, thick] (10, 0) -- (12, 0)
      node[font = \tiny, above, align = right]
      {Construction \\ plasmidique};

      \draw[plasmid_col, dotted, thin, out=0, in = 15]
      [->] (12.1, 0) to (10.8, -4.6)
      node[Gray, legende, right]
      {Transformation};

      % Génome acinetobacter
      % brin 5' 3'
      \draw[gs_rec_col] (5, -1) -- (7, -1);
      \draw[ancre_don_col] (7.0, -1) -- (8.8, -1);
      \draw[genome_col, thick] (2, -1) -- (5, -1);
      \draw[arrows = {-Stealth[left]}, genome_col, thick] (8.8, -1) -- (12, -1) node[font=\tiny, above, align = right] {Génome \\ \emph{Acinetobacter}};
      %amorces
      \draw[Red3, ultra thick] (4, -1) -- (4.2, -1) node[above left, font=\tiny] {Amorce};
      \draw[gs_rec_col] (5, -1.2) -- (7, -1.2);
      \draw[ancre_don_col] (7.0, -1.2) -- (8.8, -1.2);
      \draw[arrows = {Stealth[right, swap]-}, genome_col, thick] (2, -1.2) -- (5, -1.2);
      \draw[genome_col, thick] (8.8, -1.2) -- (12, -1.2);

      % Lignes reliant les fragments
      \draw[gs_don_col, thin, dotted] (4, 0) -- (5, -1);
      \draw[gs_don_col, thin, dotted] (6, 0) -- (7, -1);
      \draw[ancre_don_col,  thin, dotted] (8, 0) -- (7, -1);
      \draw[ancre_don_col,  thin, dotted] (9.8, 0) -- (8.8, -1);

      % petit éclair pour marquer la cassure
      \node[Gold2, font=\Large] at (7, -1.1) {\Lightning};
      \node[Gray, font=\tiny] at (7, -1.4) {Cassure Double Brin};


      %%
      %% RÉSECTION
      %%
      \draw[Gray] [->] (7, -1.6) -- (7, -2.4) node[legende, left, midway] {Résection};

      % brin ->
      \draw[gs_rec_col, arrows = {-Stealth[left, scale = 1]}, very thick] (5, -2.5) -- (7, -2.5);
      \draw[ancre_don_col] (8.3, -2.5) -- (8.8, -2.5);
      \draw[genome_col, thick] (2, -2.5) -- (5, -2.5);
      \draw[arrows = {-Stealth[left]}, genome_col, thick] (8.8, -2.5) -- (12, -2.5);
      % brin <-
      \draw[gs_rec_col, ultra thick] (5, -2.7) -- (5.5, -2.7);
      \draw[arrows = {Stealth[right, swap, scale = 1]-}, ancre_don_col, very thick] (7.0, -2.7) -- (8.8, -2.7);
      \draw[arrows = {Stealth[right, swap]-}, genome_col, thick] (2, -2.7) -- (5, -2.7);
      \draw[genome_col, thick] (8.8, -2.7) -- (12, -2.7);

      %%
      %% RÉPARATION
      %%
      \draw[Gray] [->] (7, -2.8) -- (7, -3.4) node[legende, midway, left] {Réparation};

      \draw[gs_rec_col, arrows = {-Stealth[left, scale = 1]}, very thick] (5, -3.5) -- (7, -3.5);
      \draw[ancre_don_col] (8.3, -3.5) -- (8.8, -3.5);
      \draw[genome_col, thick] (2, -3.5) -- (5, -3.5);
      \draw[arrows = {-Stealth[left]}, genome_col, thick] (8.8, -3.5) -- (12, -3.5);
      % brin <-
      \draw[gs_rec_col, ultra thick] (5, -3.7) -- (5.5, -3.7);
      \draw[ ancre_don_col, very thick] (8.8, -3.7) -- ++(-0.5, 0) -- ++(-0.2, -0.6) -- ++(-0.7, 0);
      \draw[arrows = {Stealth[right, swap]-}, genome_col, thick] (2, -3.7) -- (5, -3.7);
      \draw[genome_col, thick] (8.8, -3.7) -- (12, -3.7);

      %% fragment insert avec boucle pour cassette de résistance
      % je place l'origine de la construction au début da la région ancre sur le
      % génome d'acinetobacter. C'est plus facile pour s'aligner
      \draw[ancre_don_col, very thick]                     (8.8, -4.6) -- ++(-2, 0);
      \draw[genome_col, very thick]                  (6.8, -4.6) -- ++(-0.2, 0);
      \draw[kanr_col, very thick]                     (6.6, -4.6) -- ++(-2, 0);
      \draw[am1_col, very thick]                       (4.6, -4.6) -- ++(-0.2, 0);
      \draw[gs_don_col, very thick]                    (4.4, -4.6) -- ++(-2, 0);
      \draw[plasmid_col, thick]                            (2.4, -4.6) -- ++(-0.4, 0);
      \draw[plasmid_col, thick, arrows = {-Stealth[left]}] (8.8, -4.6) -- ++(2, 0);


      %%
      %% RESYNTHÈSE
      %%

      \draw[densely dotted, ancre_don_col, very thick]    (7.35, -4.3) --  (6.8, -4.3);
      \draw[densely dotted, genome_col, very thick] (6.8, -4.3) -- ++(-0.2, 0);
      \draw[densely dotted, kanr_col, very thick]    (6.6, -4.3) -- ++(-2, 0);
      \draw[densely dotted, am1_col, very thick]      (4.6, -4.3) -- ++(-0.2, 0);
      \draw[densely dotted, gs_don_col, very thick, arrows = {-Stealth[left, scale = 0.5]}]   (4.4, -4.3) -- ++(-1.5, 0);
      %% lignes permettant de montrer l'appariemment des régions
      \draw[gs_don_col, thin, dotted ] (2.9, -4.3) -- (5, -3.5);
      \draw[gs_don_col, thin, dotted ] (4.4, -4.3) -- (7, -3.5);

      %%
      %% RÉAPPARIEMMENT
      %%

      \draw[Gray] [->] (7, -4.8) -- ++(0, -1) node[midway, left, legende] {Réappariemment};

      % brin ->
      \draw[gs_rec_col, arrows = {-Stealth[left, scale = 1]}, very thick] (5, -6) -- (7, -6);
      \draw[ancre_don_col, very thick] (11.4, -6) -- ++(-0.5, 0);
      \draw[genome_col, thick, arrows = {-Stealth[left]}] (11.4, -6) -- ++(1, 0);
      % \draw[ancre_don_col] (8.3, -6) -- (8.8, -6);
      \draw[genome_col, thick] (2, -6) -- (5, -6);
      %% brin <-
      % petit bout bleu
      \draw[gs_rec_col, ultra thick]                        (5, -6.2)   --   (5.5, -6.2);
      \draw[gs_don_col, densely dotted, very thick, arrows = {Stealth[left, scale = 0.5]-}] (6.0, -6.2) -- ++(1.0, 0);
      \draw[densely dotted, am1_col, very thick]            (7.0, -6.2) -- ++(0.2, 0);
      \draw[densely dotted, kanr_col, very thick]          (7.2, -6.2) -- ++(2.0, 0);
      \draw[densely dotted, genome_col, very thick]       (9.2, -6.2) -- ++(0.2, 0);
      \draw[densely dotted, ancre_don_col, very thick]          (9.4, -6.2) --   (9.9, -6.2);
      \draw[solid, ancre_don_col, very thick]                   (9.9, -6.2) -- ++(1.5, 0);
      % génome
      \draw[genome_col, thick]                            (11.4, -6.2) -- ++(1, 0);
      \draw[genome_col, thick, arrows = {Stealth[left]-}] (2, -6.2) --      (5, -6.2);

      %%
      %% SYNTHÈSE DU BRIN 5'
      %%
      \draw[Gray] [->] (7, -6.5) -- ++(0, -1)
      node[midway, left, legende, align = right] {Synthèse du brin \\ complémentaire };

      %% brin ->
      \draw[gs_rec_col, arrows = {-Stealth[left, scale = 1]}, very thick]
      (5, -8.0) -- (7, -8.0);
      \draw[ancre_don_col, very thick]
      (11.4, -8.0) -- ++(-0.5, 0);
      \draw[genome_col, thick, arrows = {-Stealth[left]}]
      (11.4, -8.0) -- ++(1, 0);
      \draw[genome_col, thick]
      (2, -8.0) -- (5, -8.0);
      \draw[densely dotted, am1_col, very thick]
      (7.0, -8.0) -- ++(0.2, 0);
      \draw[densely dotted, kanr_col, very thick]
      (7.2, -8.0) -- ++(2.0, 0);
      \draw[densely dotted, genome_col, very thick]
      (9.2, -8.0) -- ++(0.2, 0);
      \draw[densely dotted, ancre_don_col, very thick]
      (9.4, -8.0) -- ++(1.5, 0);

      %% brin <-
      \draw[gs_rec_col, very thick]
      (5, -8.2)   --   (5.5, -8.2);
      \draw[gs_don_col, solid, very thick]
      (6.0, -8.2) -- ++(1.0, 0);
      \draw[gs_rec_col, densely dotted, very thick, arrows = {-Stealth[left, scale = 0.5]}]
      (6.0, -8.2) -- ++(-0.5, 0);
      \draw[solid, am1_col, very thick]
      (7.0, -8.2) -- ++(0.2, 0);
      \draw[solid, kanr_col, very thick]
      (7.2, -8.2) -- ++(2.0, 0);
      \draw[solid, genome_col, very thick]
      (9.2, -8.2) -- ++(0.2, 0);
      \draw[solid, ancre_don_col, very thick]
      (9.4, -8.2) --   (9.9, -8.2);
      \draw[solid, ancre_don_col, very thick]
      (9.9, -8.2) -- ++(1.5, 0);
      % génome
      \draw[genome_col, thick]
      (11.4, -8.2) -- ++(1, 0);
      \draw[genome_col, thick, arrows = {Stealth[left]-}]
      (2, -8.2) -- (5, -8.2);


      %%
      %% RÉSULTAT FINAL
      %%

      \draw[Gray] [->] (7, -8.5) -- ++(0, -1) node[midway, left, legende] {Ligature};

      \draw[gs_rec_col, very thick]
      (5, -10.0) -- (7, -10.0);
      \draw[ancre_don_col, very thick]
      (11.4, -10.0) -- ++(-0.5, 0);
      \draw[genome_col, thick, arrows = {-Stealth[left]}]
      (11.4, -10.0) -- ++(1, 0);
      \draw[genome_col, thick]
      (2, -10.0) -- (5, -10.0);
      \draw[solid, am1_col, very thick]
      (7.0, -10.0) -- ++(0.2, 0);
      \draw[solid, kanr_col, very thick]
      (7.2, -10.0) -- ++(2.0, 0);
      \draw[solid, genome_col, very thick]
      (9.2, -10.0) -- ++(0.2, 0);
      \draw[solid, ancre_don_col, very thick]
      (9.4, -10.0) -- ++(1.5, 0);

      %% brin <-
      \draw[gs_rec_col, very thick]
      (5, -10.2)   -- (5.5, -10.2);
      \draw[gs_don_col, solid, very thick]
      (6.0, -10.2) -- ++(1.0, 0);
      \draw[gs_rec_col, solid, very thick]
      (6.0, -10.2) -- ++(-0.5, 0);
      \draw[solid, am1_col, very thick]
      (7.0, -10.2) -- ++(0.2, 0);
      \draw[solid, kanr_col, very thick]
      (7.2, -10.2) -- ++(2.0, 0);
      \draw[solid, genome_col, very thick]
      (9.2, -10.2) -- ++(0.2, 0);
      \draw[solid, ancre_don_col, very thick]
      (9.4, -10.2) --   (9.9, -10.2);
      \draw[solid, ancre_don_col, very thick]
      (9.9, -10.2) -- ++(1.5, 0);
      % génome
      \draw[genome_col, thick]
      (11.4, -10.2) -- ++(1, 0);
      \draw[genome_col, thick, arrows = {Stealth[left]-}]
      (2, -10.2) -- (5, -10.2);

      %% rectangle de zoom
	    \draw[Gray, thick, densely dotted, fill = Gray, opacity = 0.3]
      (6, -10.4) rectangle (7, -9.8);

      \draw[Gray, very thick, densely dotted, opacity = 0.6, bend right = 5]
      [->] (7, -10.4) to (14.75, -10.75);

	    \draw[Gray, thick, densely dotted]
      (4, -10.6) rectangle (7.6, -9.6);

      \draw[Gray, very thick, densely dotted, opacity = 0.6, out = 180, in = 180]%, bend right = 75]
      [->] (4, -10.6) to (3.8, -17);
    \end{scope}


    \begin{scope}[shift={(10, -10)}, scale = 0.5]
      %% Zoom montrant l'alignement des positions donneurs avec les
      %% positions receveuse
      \draw[Gray, thick, densely dotted, fill = Gray, opacity = 0.3] (-0.5, -1.1) rectangle ++(11, 3.2);

      \draw[gs_rec_col, thick, arrows = {-Stealth[left]}] (0, 1) -- ++(10, 0)  node[above, font=\scriptsize] {Brin receveur};
      \draw[gs_don_col, thick, arrows = {Stealth[left]-}] (0, 0) -- ++(10, 0) node[below, font=\scriptsize] {Brin donneur};

      \foreach \x in {1,2,...,9}
      {
        \draw[color = gs_rec_col, fill = gs_rec_col] (\x, 1) circle (0.05);
        \draw[color = gs_don_col, fill = gs_don_col] (\x, 0) circle (0.05);
      }

      \draw[Gray, very thick] [->] (5, -1.5) -- ++(0, -1)
      node[midway, left, font=\tiny, align = right]
      {Correction des \\ mésappariemments};
    \end{scope}

    \begin{scope}[shift={(9, -13.5)}, scale = 0.8]
      %% Zoom montrant la correction des mésappariemments dans les clones

     \draw[gs_rec_col, thick, dotted, fill = gs_rec_col, opacity = 0.3]
     (-0.5, -0.5) rectangle (4.5, 2.5)
     node[midway, opacity = 1, font = \scriptsize] {Région conservée};

     \draw[gs_don_col, thick, dotted, fill = gs_don_col, opacity = 0.3]
     (4.5, -0.5) rectangle (10.5, 2.5)
     node[midway, opacity = 1, font = \scriptsize] {Région convertie};

      \draw[gs_rec_col, thick, arrows = {-Stealth[left]}] (0, 2) -- ++(10, 0);
      \draw[gs_don_col, thick, arrows = {Stealth[left]-}] (0, 0) -- ++(10, 0);

      \foreach \x in {4,5,...,9}
      {
        \draw[color = gs_don_col, fill = gs_don_col] (\x, 2) circle (0.05);
        \draw[color = gs_don_col, fill = gs_don_col] (\x, 0) circle (0.05);
      }

      \foreach \x in {1,2,...,4}
      {
        \draw[color = gs_rec_col, fill = gs_rec_col] (\x, 2) circle (0.05);
        \draw[color = gs_rec_col, fill = gs_rec_col] (\x, 0) circle (0.05);
      }

      \draw[Gray, densely dash dot, very thick]
      (4.5, 2.6) -- ++(0, -3.6)
      node[below, font=\scriptsize] {Point de recombinaison};
    \end{scope}

    \begin{scope}[shift={(-3, -6)}, line width = 3pt]
      \draw[gs_rec_col]
      (5, -10.0) -- (6.3, -10.0);
      \draw[gs_don_col]
      (6.3, -10.0) -- (7.0, -10.0);
      \draw[am1_col, line width = 1pt, arrows = {<-}]
      (7.0, -10.0) -- ++(0.4, 0);
      \draw[genome_col, line width = 1.5pt]
      (2, -10.0) -- (5, -10.0);
      \draw[Red3, line width = 1pt, arrows = {->}]
      (4.0, -10.0) -- ++(0.4, 0);

      \draw[Gray, dotted, thick, out=-90, in=0]
      [->] (12.5, -8) to (7.6, -10);
    \end{scope}

    \node[text width=6.5cm, legende text, align = left] at (14, -4) {%
      % 14 lignes possibles
      Les constructions linéarisées sont assimilées par \emph{Acinetobacter} en
      phase exponentielle de croissance. Les cassures doubles-brins occasionnées
      aléatoirement au cours de la réplication au locus d'intérêt entraînent la
      résection de la région en \(3'\) du site de cassure. Cette résection est
      réparée par synthèse utilisant une matrice d'ADN homologue. En cultivant
      les transformants sur un milieu sélectif, nous avons sélectionné les
      clones qui ont utilisé la construction plasmidique pour matrice, ayant
      intégré la cassette de résistance à la kanamycine.

      Au cours du réappariemment des brins, un hétéroduplex se forme : les brins
      homologues ne sont pas rigoureusement complémentaires. Le donneur
      \tikz[baseline = -0.5ex, very thick, scale = 0.3]{\draw[gs_don_col] (0, 0)
        -- ++(1, 0);} introduit des mésappariemments avec la séquence du
      receveur \tikz[baseline = -0.5ex, very thick, scale =
      0.3]{\draw[gs_rec_col] (0, 0) -- ++(1, 0);} ; ils doivent être corrigés.
    };

    \node[text width = 7.4cm, legende text, align = left] at (3.2, -12.8) {%
      % 6 lignes possibles
      La correction des bases mésappariées entraîne de la \emph{conversion
        génique}. La région convertie a le même haplotype que celui du donneur,
      la région conservée conserve celui du receveur. Elles sont séparées par le
      point de recombinaison.

      Le brin donneur introduit des mésappariemments toutes les \(30\) paires de
      bases. Leur correction entraîne la conversion du site soit vers A ou T,
      soit vers G ou C, selon la construction donneuse.

      L'amplification spécifique de la région convertie puis le séquençage des
      amplicons permet de déterminer la polarité des évènements de conversion et
      la position des points de recombinaison.
      %
    };
  \end{tikzpicture}


\end{center}
