\begin{center}
  \setstretch{1.0}
    \rmfamily

    \tikzset{legende fleche/.style={Gray, dotted, thick, opacity = 0.6}}
    \tikzset{legende text/.style={black, text width = 5cm, font = \scriptsize, above}}
    \tikzset{align fleche/.style={midway, right, align = left, darkgray,
        font=\scriptsize}}

    \begin{tikzpicture}[join=round]

      \draw [->]
       (0, 0) node[above] {\includegraphics[width = 0.4\textwidth]{img/electropherogram.jpg}}
       -- ++(0, -1)
       node[align fleche, text width = 6cm ] {%
         \textsf{{\color{Black} a.}} %
         Analyses des électrophérogrammes obtenus par le programme
         \texttt{phred} et attribution du score de qualité aux bases.
         %
       }
       node[font = \footnotesize, below] {\texttt{\textcolor{kanr_col}{ATCG.......................ACCG}}}
       ;

       \node[font = \footnotesize] at (0, -2.5) {\texttt{\textcolor{gs_rec_col}{ATCG.......................ATCG}}}; % receveur
       \node[font = \footnotesize] at (0, -3.0) {\texttt{\textcolor{gs_don_col}{ACCG.......................ACCG}}}; % donneur
       \node[font = \footnotesize, opacity = 0.7] at (0, -3.5) {\texttt{\textcolor{kanr_col}{ATCG.......................ACCG}}}; % recombinant
       \node[gs_rec_col, font=\scriptsize] at (-4, -2.5) {Receveur :};
       \node[gs_don_col, font=\scriptsize] at (-4, -3.0) {Donneur :};
       \node[kanr_col, font=\scriptsize] at (-4, -3.5) {Recombinant :};

       \draw[bend left=90,  dotted, thick] [->]
       (+3.2, -1.2) to
       node[midway, right, align fleche, align = left, text width = 6cm] {%
         \textsf{{\color{Black} b.}} %
         Alignement avec la référence par \texttt{muscle}. %
         La référence est composée de l'alignement de la séquence receveuse
         sauvage et de la séquence donneuse synthétique.
         %
       } ++(0, -2.2) ;

       \draw[align fleche] [->] (0, -4.0) -- ++(0, -1.5) %
       node[align fleche, midway, right, align = left, text width = 7cm] {%
         \textsf{{\color{Black} c.}} %
         Détermination du sens des événèments de conversion et attribution du
         score de qualité aux bases. L'information de la position sur la
         séquence de référence permet de comparer les lectures de séquençage
         entre elles.} %
       node[font=\tiny, below] {%
         %
         \fontspec{Gill Sans}
         \begin{tabular}{>{\color{gs_rec_col}}c>{\color{gs_don_col}}c>{\color{kanr_col}}cccc}
           \toprule
           Receveur & Donneur & Recombinant & Position & Qualité & Conversion \\
           \midrule
           A        & A       & A           & 31      & 30      &          \\
           \rowcolor{LightGray}
           T        & C       & C           & 32      & 40      & oui      \\
           C        & C       & C           & 33      & 42      &          \\
           G        & G       & G           & 34      & 42      &          \\
           .        & .       & .           & .       & .       &          \\
           .        & .       & .           & .       & .       &          \\
           .        & .       & .           & .       & .       &          \\
           A        & A       & A           & 61      & 42      &          \\
           \rowcolor{LightGray}
           T        & C       & T           & 62      & 30      & non      \\
           C        & C       & C           & 63      & 40      &          \\
           G        & G       & G           & 64      & 28      &          \\
           \bottomrule
           %
         \end{tabular}
       };

       \draw[legende fleche] [<-] (3.8, -6.5) -- (4.4, -6.5) -- (5, -9.0);
       \draw[legende fleche] [<-] (3.8, -8.3) -- (4.4, -8.3) -- (5, -9.0) -- (8, -9.0) %
       node[opacity = 1, Gray, text width = 5cm, font=\scriptsize, above] {%
         Ces positions sont les positions d'intérêt. Elles correspondent à
         l'introduction d'une base G ou C par le donneur, alors que le receveur
         présente une base A ou T. La base présente chez le clone recombinant
         permet de déterminer la polarité de la conversion. Le score de qualité
         permet d'accorder plus ou moins de confiance à la base appelée par le
         programme \texttt{phred}.
         %
       } node {$\bullet$} ;


       %%
       %% Flèches montrant le basculement de ligne à colonne.
       %%
       \draw[dotted, kanr_col, thin, bend left ] [->] (+2.9, -3.7) to (-0.5, -8.8);
       \draw[dotted, kanr_col, thin, bend right] [->] (-2.9, -3.7) to (-0.7, -6.3);

  \end{tikzpicture}

\end{center}
