\begin{center}
  \rmfamily
  % \fontspec{Gill Sans}
  \setstretch{1.0}

  %% TODO indiquer l'origine de la région convertie.
  %% TODO décaler les flèches bleues et rouge pour qu'elles correspondent mieux.

  \tikzset{trace text/.style = {align = left, below right, font = \scriptsize}}
  \tikzset{trace legend/.style = {Black, opacity = 1, align = center, text width = 4.5cm,
      above, font = \scriptsize}} %
  \tikzset{trace fleche/.style={Gray, dotted, thick, opacity = 0.6}}
  % \tikzset{trace text/.style={black, text width = 7cm, font = \scriptsize, above}}

  \begin{tikzpicture}

    %% Trace de conversion des donneurs strong

    \node[trace text] at (0, 0) {%
      \textcolor{white}{cuicui}\\ % un titre était présent ici. doublon avec le titre en bas. remplacé
         % par un blank space pour ne pas avoir à tout décaler.
      \includegraphics[width = 0.65\textwidth]{img/trace_ws.pdf}};

    \coordinate (complexe) at (7, -14);
    \coordinate (complexe weak) at (1.6, -9);
    \coordinate (badqual) at (1.4, -14);
    \coordinate (tout bleu) at (12, -13.5);

    % \draw[opacity = 0.2, line width = 0.01pt, Gray] (0,0) grid (15, -15);
    % \draw[opacity = 0.2, line width = 0.005pt, step = 0.5, Gray] (0,0) grid (15, -15);

    \draw[trace fleche, gs_rec_col] [<-] (10.2, -1.0) -- (12, -1.0)
    node[trace legend] {\textcolor{gs_rec_col}{Génotype du receveur}}
    node {\(\bullet\)}
    ;

    \draw[trace fleche, gs_don_col] [<-] (10.2, -14.35) -- (12, -14.35)
    node[trace legend] {\textcolor{gs_don_col}{Génotype du donneur}}
    node {\(\bullet\)}
    ;

    \draw[trace fleche] (10.5, -16) %
    node{\(\bullet\)} %
    node[trace legend, darkgray] {Restauration de l'haplotype sauvage GC.}
    -- (7.8, -16) -- (complexe);

    \draw[trace fleche]  (11.5, -5) %
    node{\(\bullet\)} %
    node[trace legend, darkgray] {Restauration de l'haplotype sauvage AT.}
    -- (8, -5) -- (complexe weak);

    %% flèche sens du séquençage
    \draw[trace fleche, opacity = 0.4] [->] (8, -0.3)
    node[right, font=\scriptsize, opacity = 0.5] {Sens du séquençage}
    % node{\(\bullet\)}
    -- ++(-2, 0)
    ;

    %% flèche indiquant les bases de mauvaise qualité
    \draw[trace fleche, opacity = 0.3] (5, -16) %
    node[trace legend, opacity = 0.4] {%
      Fin des lectures de séquençage, les bases sont de moindre qualité.%
    }%
    node {\(\bullet\)}
    -- (3, -16) -- (badqual)
    node {\(\bullet\)}
    ;

    \node[trace legend, align = left] at (13.5, -13) {%
      Chaque ligne horizontale représente une séquence. Les points représentent
      les positions des marqueurs sur les séquences. La couleur des points
      représente leur polarité. Ils sont \tikzcircle[gs_don_col,
      fill=gs_don_col, opacity = 0.4]{3pt} lorsque le site est dans la région
      convertie : ils correspondent à l'haplotype du donneur. Ils sont
      \tikzcircle[gs_rec_col, fill = gs_rec_col, opacity = 0.4]{3pt} lorsque le
      site est dans la région conservée.

      Les alternances \tikzcircle[gs_don_col, fill=gs_don_col, opacity =
      0.4]{3pt} \tikzcircle[gs_rec_col, fill=gs_rec_col, opacity = 0.4]{3pt}
      marquent la transition de l'haplotype converti à l'haplotype sauvage : le
      point de recombinaison est localisé entre ces deux marqueurs.

      Les séquences sont triées par longueur de région convertie. Certains
      transformants ont converti tous les marqueurs\tikz[overlay]{\draw[gray,
        dotted, opacity = 0.9] [->] (0, 0) -- ++(-4, 10) -- ++(-1, 0);}.
      Certains transformants ont conservé tous les
      marqueurs\tikz[overlay]{\draw[gray, dotted, opacity = 0.7] [->] (0, 0) -- (-2, -1) -- (-3, -1);}.
      %
    };
    %% TODO ajuster les flèches indiquant les séquences convertissant

  \end{tikzpicture}
\end{center}