\begin{center}

  \begin{figure}

    \centering
    \tikzset{legende fleche/.style={Gray, dotted, thick, opacity = 0.6}}
    \rmfamily
    \scriptsize

    \begin{tikzpicture}[scale = 0.5]
      \node[above] at (0, 0) {\includegraphics[scale = 0.5]{img/pic_second.png}};
      % \draw[opacity = 0.2, line width = 1pt, Gray] (-15,0) grid (15, 15);
      % \draw[opacity = 0.2, line width = 0.005pt, step = 0.5, Gray] (-15,0) grid (15, +15);

      \draw[legende fleche] [->] (-6, 0)        node[ above] {Score de qualité} -- ++(4.5, 0) -- ++(1, 1);
      \draw[legende fleche, Black] [->] (-6, 7) node[ above] {Base G} -- ++(4.5, 0) -- ++(1, -1);
      \draw[legende fleche, Red3] [->] (-6, 11) node[ above] {Base T} -- ++(4.5, 0) -- ++(1, -1);

    \end{tikzpicture}

    \caption[Marqueur montrant des traces de contaminations]{%
      \textbf{Exemple de marqueur montrant des traces de
        contaminations} \\ \rmfamily Cet électrophérogramme montre les bases autour du
      marqueur à la position 200. Dans une région de qualité moyenne élevée (bases en
      5' et en 3'), le marqueur présente une trace de contamination par une autre
      base. La base déterminée est la base T mais une base G est présente dans la
      population d'amplicon séquencée.
      \label{fig:pic-second}
    }
  \end{figure}

  \begin{figure}[h!]

    \centering

    \tikzset{legende fleche/.style={Gray, thick}}
    \begin{tikzpicture}[scale = 0.3]
      \scriptsize
      \foreach \x in {1,2,...,12} {
        \foreach \y in {1,2,...,8} {
        \draw (\x, \y) circle (0.3cm) ;
      }}
    \draw[fill = Red] (8, 5) circle (0.3cm);
    \draw[fill = Red] (7, 6) circle (0.3cm);
    \draw[fill = Red] (7, 4) circle (0.3cm);
    \draw[fill = Red] (6, 5) circle (0.3cm);

    \draw[Gray, densely dotted] (5.5, 3.5) rectangle (8.5, 6.5);
    \draw[Gray, densely dotted] (9.5, 0.6) rectangle (12.4, 3.5);
    \draw[Gray] (0.5, 0.5) rectangle (12.5, 8.5);

    \draw[legende fleche] [<-] (7, 5) -- ++(1, 0.5) -- ++(5, 0)
    node[right, Black] {\(\bar{n} = \frac{4}{8} = \frac{1}{2}\)}
    ;

    \draw[legende fleche] [<-] (11, 2) -- ++(1, 2.0) -- ++(1, 0)
    node[right, Black] {\(\bar{n} = \frac{0}{8} = 0\)}
    ;

    \node[right] at (12.5, 8) {\(X = 4\)};
    \node[right] at (12.5, 1) {
      % \begin{minipage}[htbp]{1.0\linewidth}
        \(\bar{N} = \frac{1}{96}\sum n_i\)
      % \end{minipage}%
    };
    \draw [->] (18.5, 4.5) -- ++(5, 0) node[midway, above] {\(\times 10000\)}
    node[right] {
      \includegraphics[width = 0.5\textwidth]{img/randomweak.png}
    };
    \end{tikzpicture}


    \caption[Des contaminations dues au hasard ?]{%
      \label{fig:simul-count}\textbf{Des contaminations dues au hasard ? } \\
      \rmfamily Par plaque de 96 puits, nous avons déterminé \(X\) le nombre de
      puits dont la séquence montre des traces de pics secondaires (voir
      figure~\ref{fig:pic-second}) et mesuré \(\bar{n}\) la moyenne du nombre de
      puits voisins contaminés. \(\bar{N}\) est la moyenne des 96 \(\bar{n}\)
      obtenus par plaque. Nous avons simulé \num{1e4} plaques avec \(X\) puits
      contaminés répartis aléatoirement, mesuré \(\bar{N}\) et comparé la valeur
      expérimentale de \(\bar{N}\) (trait vertical rouge) avec la distribution
      des \num{1e4} \(\bar{N}\) (en gris) Seules \(78 / 10000\) plaques simulées
      montrent un \(\bar{N}\) supérieur à la valeur expérimentale : la
      répartition des séquences contaminées dans les plaques ne peut pas être
      attribuée au hasard.
      %
    }
    % TODO légender figure pics secondaire simul
  \end{figure}

\end{center}