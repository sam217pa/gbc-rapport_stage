% \blankpage
\null
\vfill

\section*{\large \centering Étude expérimentale des fréquences de conversion en faveur de GC au
  cours de la recombinaison homologue chez \emph{Acinetobacter baylyi}}
\thispagestyle{empty}

La réparation de l'ADN joue un rôle important dans l'évolution de la composition
en base des génomes. Chez les mammifères, la recombinaison homologue tend à
augmenter le taux de GC des régions recombinantes par un processus appelé
conversion génique biaisée (gBGC). L'hypothèse gBGC a été récemment étendue aux
procaryotes. L'objectif de ce travail était de valider une approche
expérimentale permettant de quantifier les fréquences de conversion en faveur de
GC chez \emph{Acinetobacter}. Nous avons transformé cette bactérie naturellement
compétente par des gènes de synthèse introduisant des sites variants avec un
locus génomique, sélectionné les recombinants et séquencé les produits de
recombinaison. Nous avons pu démontré la validité de l'approche expérimentale,
en obtenant des fréquences de transformation de l'ordre de \num{1e-5} et une
moyenne de \num{3000} transformants par construction donneuse. Le séquençage de
\num{384} recombinants a servi à décrire les régions recombinantes, la
distribution des changements d'haplotype entre donneur et receveur, et à
quantifier le nombre d'évènements de conversion dans le sens AT$\rightarrow$GC.
Les régions converties obtenues démontrent que les bases AT ont une probabilité
plus forte d'être convertis en GC que l'inverse, une observation qui correspond
précisément à l'hypothèse initiale. Ce résultat représente une seconde étape
importante dans l'extension du biais de conversion génique vers GC aux
procaryotes. Il vient appuyer la proposition du gBGC comme l'une des forces
permettant aux procaryotes de maintenir des taux de GC élevés alors que la
mutation est biaisée vers AT.

\vfill