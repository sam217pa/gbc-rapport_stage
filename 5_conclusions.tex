\section*{Conclusion}
\label{sec:conclusion}
\addcontentsline{toc}{section}{Conclusion}

Nous avons pu au cours de cette étude valider l'approche expérimentale
envisagée. L'utilisation d'une cassette de résistance pour sélectionner les
clones et l'utilisation d'une ancre recombinogène ont permis d'obtenir un grand
nombre de recombinants, pour toutes les constructions donneuses, et ce malgré
les \SI{3}{\percent} de divergence entre le locus cible et le gène de synthèse
introduit. La fiabilité de la méthode de construction des plasmides donneurs a
été éprouvée et devrait permettre de mettre au point d'autres constructions
donneuses dans un temps raisonnable, à moindre coût et avec des résultats
convaincants.

Nous avons séquencé 384 clones obtenus avec ces approches, dont l'étude a permis
de décrire les zones de recombinaison, la distribution des points de
recombinaison, la longueur des régions converties et les fréquences de
restauration de l'haplotype sauvage. Ces informations sont capitales pour mettre
au point un plan expérimental permettant de confirmer ou non l'existence du
biais de conversion vers GC chez les procaryotes. Les données obtenues à ce jour
ne permettent pas de conclure sur ce point. Il semble cependant que les cas de
restaurations montrent une tendance à la restauration préférentielle des bases C
et G. De plus, sur l'ensemble des recombinants obtenus, davantage de bases AT
ont été converties en GC que l'inverse.

Ce travail devra être étendu à d'autres locus du génome d'\emph{A. baylyi}, plus
proches et plus éloignés de l'origine de réplication. D'autres modèles
bactériens sont également à l'étude : le modèle de la recombinaison homologue
après transmission d'un plasmide porteur de sites variants par conjugaison chez
\emph{E. coli} et \emph{Burkholderia NOMDESPECE?}. La confrontation des
résultats obtenus pour ces trois modèles devrait permettre de quantifier
formellement les fréquences de conversion en faveur de GC chez les procaryotes.

La démonstration d'une conversion génique biaisée vers GC chez les procaryotes
serait importante, elle conduirait à repenser globalement les hypothèses qui
attribuent au taux de GC un rôle adaptatif, et à intégrer la conversion génique
dans l'ensemble des forces évolutives qui régissent la composition en base des
génomes.
