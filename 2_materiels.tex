\section{Matériels \& Méthodes}
\label{sec:materiel}

\subsection{Constructions des plasmides}
\label{subsec:constructions}

Les constructions suivantes ont été réalisées par bioinformatique.

Les PCRs spécifiques du gène d'intérêt ont été réalisées. Les amplicons ainsi
obtenus ont été insérés par ligature dans le plasmide pGEM-T. Le plasmide a été
inséré dans la souche d'\textit{E.coli} One Shot\textsuperscript{{\textregistered}}
\texttt{TOP-10} (Invitrogen, \texttt{C404-10}), suivant le protocole du
fabriquant. Les cellules transformantes possédant l'insert ont été discriminées
par un crible bleu / blanc sur gélose solide Luria-Bertani additionnée
d'Ampicilline à \SI{75}{\ug\per\mL}, de XGal et d' IPTG. Le sens d'insertion du
fragment a été vérifié par PCR M13R-1392. Les inserts des clones positifs ont
été séquencés par séquençage Sanger (GATC Biotech AG). Les plasmides des clones
validés par séquençage ont été extraits par le kit Nucleospin-Plasmid
(Macherey-Nagel), et ouverts par digestion \texttt{SpeI}, pendant \si{12\hour} à
\si{37\celsius}.

La cassette de résistance à la kanamycine \texttt{aphA3} est amplifiée par PCR,
de même que la région \emph{ancre} facilitant la recombinaison. Les deux
amplicons obtenus ont été ligaturés dans le plasmide pGEM-T porteur des
constructions ouvert par \texttt{SpeI}. La ligature a été réalisée simultanément
par le kit InFusion (Takara Clontech). Le produit de ligature obtenu a été
inséré dans la souche optimisée pour la chimio-compétence \emph{E.coli} Stellar
(Takara Clontech). Les transformants ont été séléctionnés sur milieu LB solide
additionné d'Ampicilline à \SI{75}{\ug\per\mL} de Kanamycine à
\SI{50}{\ug\per\mL}. Les transformants ont été confirmés par PCR spécifique de
l'insert.

\subsection{Transformations d'\emph{Acinetobacter baylyi}}
\label{subsec:transfo}

\si{1\ug} de plasmide a été extrait et linéarisé par digestion \texttt{ScaI}
\si{12\hour} à \si{37\celsius}. L'enzyme a été inactivée par incubation
\si{10\min} à \si{70\celsius}. \si{390\uL} d'une culture pure
d'\emph{Acinetobacter baylyi} \texttt{BD413}, avec une absorbance de \si{0,8} à
\si{600\nm} ont été incubés en présence de \si{200\ng} de plasmide linéarisé,
pendant \si{1\hour} à \si{28\celsius}. Pour éliminer les traces de plasmides
résiduels, les suspensions ont été incubées \si{15\min} à \si{37\celsius} en
présence de DNAse à \SI{20}{\umol\per\L}. Les cellules recombinantes ont été
sélectionnées par étalement en spirales (InterScience) sur milieu LB solide
additionné de Kanamycine à \SI{50}{\ug\per\mL} et incubation \SI{24}{\hour} à
\SI{37}{\celsius}. Les dénombrements ont été effectués via un compteur
automatique Scan\textsuperscript{\textregistered}1200 (InterScience).

96 clones isolés ont été purifiés par isolement sur milieu LBK et incubation
\SI{24}{\hour} à \si{37°C}. Une colonie isolée par clone a été suspendue dans
\SI{50}{\uL} d'\ce{H_2O} ultra-pure. Les locus d'intérêt ont été amplifiés en
PCR par la Taq polymérase haute fidélité Phusion (ThermoFischer). Les PCRs ont
été réalisées avec \SI{2}{\uL} de suspension pour matrice. L'amplification a été
confirmée par migration sur gel d'agarose à 1\%. Les amplicons ont été séquencés
par la technique de Sanger.
% TODO citation

\subsection{Alignements}
\label{subsec:align}

Les spectrogrammes de séquençage reçus au format propriétaire \texttt{abi}
(Applied Biosystem) ont été analysés par le programme \texttt{phred}
% TODO citation
. Les séquences \texttt{FASTA} obtenues ont été alignées à la référence par
\texttt{muscle}
% TODO citation
. La référence en question correspond à l'alignement de la séquence sauvage et
de la séquence du gène synthétique, respectivement \emph{receveur} et
\emph{donneur} de l'évènement de recombinaison. Un script Python
% TODO citation
a été développé pour analyser les alignements obtenus. Il infère les positions
des SNPs d'intérêt dans l'alignement de référence, et détermine le génotype du
clone séquencé. Les alignements par paire en colonne obtenus ont été analysés
par \texttt{R}
% TODO citation
. Les scripts développés sont accessibles à l'adresse
\url{http://sam217pa.github.io/gbc-seq_mars/project.html}.