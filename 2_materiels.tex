%==============================================================================
% début de la page de figure.
% - stratégies envisagées
% - protocole de constructions moléculaires
%
\null
\vfill

% ------------------------------------------------------------
\begin{center}
  \rmfamily
  \setstretch{1.0}
  \tikzset{>=stealth}
  \tikzset{plasmid lines/.style={xscale = 0.2, yscale = 0.2, very thick}}
  \tikzset{legende/.style={font=\scriptsize, color=Gray, align=right}}
  \tikzset{legende text/.style={font=\scriptsize, color=Black, align=right}}

  \tikzset{
    plasmid/.pic={
	    \draw[plasmid_col, plasmid lines, semithick, rounded corners = 0.5ex] (3, -1) rectangle (11, 0);
	    \draw[gs_don_col, plasmid lines ] (4, 0) -- (6, 0);
	    \draw[am1_col , plasmid lines] (6,  0) -- (6.2, 0);
	    \draw[kanr_col , plasmid lines] (6.2,  0) -- (7.8, 0);
	    \draw[genome_col , plasmid lines] (7.8, 0) -- (8.0, 0);
	    \draw[ancre_don_col , plasmid lines] (8.0, 0) -- (9.8, 0);
	    \draw[am3_col , plasmid lines] (9.8, 0) -- (10.0, 0);% node[font=\tiny] {$\bullet$};
    }
  }
  \tikzset{
    construct/.pic={
        \draw[plasmid lines, gs_don_col] (4, 0) -- (6, 0);
        \draw[plasmid lines, am1_col] (6, 0) -- (6.2, 0);
        \draw[plasmid lines, am1_col] (6, 0) -- (6.2, 0);
        \draw[plasmid lines, kanr_col] (6.2, 0) -- (7.8, 0);
        \draw[plasmid lines, genome_col] (7.8, 0) -- (8.0, 0);
        \draw[plasmid lines, genome_col] (7.8, 0) -- (8.0, 0);
        \draw[plasmid lines, ancre_don_col] (8.0, 0) -- (9.8, 0);
        \draw[plasmid lines, am3_col] (9.8, 0) -- (10.0, 0);
        \draw[plasmid lines, plasmid_col, thick] (2, 0) -- (4, 0);
        \draw[plasmid lines, plasmid_col, thick] (10, 0) -- (12, 0);
    }
  }

  \tikzset{
    gene interet/.pic={
      \draw[xscale = 0.5, gs_don_col] (0, 0) -- (1, 0);
      \draw[xscale = 0.5, am1_col] (0.8, 0) -- (1, 0);
    }
  }

  \begin{tikzpicture}[scale=0.7, line width = 2pt, join=round]
    \begin{scope}[shift={(-3.5, 12)}]
      \draw[gs_don_col] (0, 0) -- (1, 0);
      \draw[am1_col] (0.8, 0) -- (1, 0);
      \draw[Gray] [->] (1.5, 0) -- (2.7, 0) node[below, midway, font=\scriptsize] {PCR};
    \end{scope}

    \begin{scope}[shift={(-0.3, 10)}]
	    % \draw[plasmid_col] (0, 0) arc (0:30:3);
	    \draw[Gray]  (1, 0.5) node {\scriptsize +};
	    \draw[plasmid_col, thick] (0, 0) arc (150:390:1);
	    \draw[plasmid_col] (1, -0.5) node {\scriptsize pGEM-T};
	    \draw[plasmid_col] (0.1, 0.3) node {\tiny \sffamily T};
	    \draw[plasmid_col] (1.7, 0.3) node {\tiny \sffamily T};
    \end{scope}


    \begin{scope}[shift={(0, 11)}, line width = 0.5 pt]
	    \foreach \y in {0,0.2,...,2} {
        \draw[gs_don_col] (0, \y) -- (1, \y);
        \draw[am1_col   ] (0.9, \y) -- (1, \y);
        \draw[gs_don_col] (-0.1, \y) node {\tiny \sffamily A};
        \draw[gs_don_col] (1.1, \y) node {\tiny \sffamily A};
	    }
	    \draw[gs_don_col, align=center] (0.5, 2.5) node[font=\tiny] {Gène\\Synthétique};
      % essaie d'aligner
    \end{scope}

    \draw[Gray] [->] (3, 12) -- (4, 12)
    node[above, midway]
    {\includegraphics[width=0.1\textwidth, angle = 90]{schema/construct-img/ligation.png}}
    node[below, midway, font=\scriptsize]
    {Ligature};

    \begin{scope}[shift={(5.0, 13)}, scale = 0.8]
	    % plasmide pGEM-T avec construction, avant infusion
	    \draw[plasmid_col, semithick] (0, 0) arc (150:390:1);
	    \draw[am3_col] (0, 0) arc (150:20:1) ;
	    \draw[am1_col] (0, 0) arc (150:30:1);
	    \draw[gs_don_col] (0, 0) arc (150:40:1) ;
    \end{scope}


    \draw[Gray] [->] (7, 12) -- ++(2, -0.5)
    node[midway, above, font=\scriptsize]
    {Digestion};

    \begin{scope}[shift={(7, 10.5)}]
	    % plasmide ouvert, avec ancre et kana
	    \draw[plasmid_col, thick] (0, 0) -- (4, 0) node[auto, above, midway] {\tiny{pGEM-T}};
	    \draw[gs_don_col] (4, 0) -- (6, 0) node [auto, above, midway] {\tiny{Gène Synthétique}};
	    \draw[am1_col] (6, 0) -- (6.2, 0);

	    \draw[am1_col] (6, -0.3) -- (6.2, -0.3);
	    \draw[kanr_col] (6.2, -0.3) -- (7.8, -0.3) node[auto, above, midway] {\tiny{KanR}};
	    \draw[genome_col] (7.8, -0.3) -- (8.0, -0.3);

	    \draw[genome_col] (7.8, 0) -- (8.0, 0);
	    \draw[ancre_don_col] (8.0, 0) -- (9.8, 0) node[auto, above, midway] {\tiny{Ancre}};
	    \draw[am3_col] (9.8, 0) -- (10.0, 0);

      \draw[plasmid_col, dotted, thin] [->] (0, 0) .. controls (3, -1) and (8, -1) .. (10.0, 0);

	    \draw[am3_col] (0, 0) -- (0.2, 0);
    \end{scope}

    \draw[Gray] [->] (14, 9.5) -- ++(0, -1) node[legende, midway, left] {Ligature \\ InFusion};

    \begin{scope}[shift={(7, 8)}]
      % Plasmide fermé avec infusion
	    \draw[plasmid_col, rounded corners, thick] (3, -0.5) rectangle (11, 0);
	    \draw[gs_don_col ] (4, 0) -- (6, 0);
	    \draw[am1_col ] (6,  0) -- (6.2, 0);
	    \draw[am1_col ] (6,  0) -- (6.2, 0);
	    \draw[kanr_col ] (6.2,  0) -- (7.8, 0);
	    \draw[genome_col ] (7.8, 0) -- (8.0, 0);
	    \draw[genome_col ] (7.8, 0) -- (8.0, 0);
	    \draw[ancre_don_col ] (8.0, 0) -- (9.8, 0);
	    \draw[am3_col ] (9.8, 0) -- (10.0, 0);
    \end{scope}

    \draw[Gray] [->] (14, 7) -- (14, 6)
    node[legende, midway, left]
    {Amplification clonale\\dans \emph{E.coli}};

    \begin{scope}[shift={(14, 4.5)}]
      % Plasmide apmplifié avec infusion

      \matrix (A)[column sep=2mm, row sep=2mm] {
        \pic{plasmid}; & \pic{plasmid}; & \pic{plasmid}; \\
        \pic{plasmid}; & \pic{plasmid}; & \pic{plasmid}; \\
        \pic{plasmid}; & \pic{plasmid}; & \pic{plasmid}; \\
        \pic{plasmid}; & \pic{plasmid}; & \pic{plasmid}; \\
      };
    % pic {plasmid};

    \end{scope}

    \draw[Gray] [->] (14, 2.75) -- ++(0, -1)
    node[legende, midway, left]
    {Linéarisation par\\digestion enzymatique};

    \begin{scope}[shift={(14, 1)}, scale = 0.4]
      \matrix (A)[column sep=2mm, row sep=2mm] {
        \pic{construct}; & \pic{construct}; & \pic{construct}; \\
        \pic{construct}; & \pic{construct}; & \pic{construct}; \\
        \pic{construct}; & \pic{construct}; & \pic{construct}; \\
        \pic{construct}; & \pic{construct}; & \pic{construct}; \\
      };
    \end{scope}

    %% TEXTE DE DESCRIPTION
    \node[text width=9cm, legende text, align = left] at (3, 5) {
      Le gène d'intérêt \tikz[baseline = -0.5ex, line width=1.3pt] {\path (0, 0)
        pic {gene interet};} est amplifié par PCR spécifique. Des bases adénines
      sont ajoutées naturellement sur les fragments au cours des derniers cycles
      de PCR. Ces fragments sont ensuite ligaturés dans le plasmide pGEM-T.
      Celui-ci est disponible ouvert avec des bases thymines sortantes. La
      ligature est catalysée par la T4DNA Ligase. Le plasmide obtenu est
      linéarisé par digestion enzymatique au site d'insertion des fragments.

      La cassette de résistance à la kanamycine \tikz[baseline = -0.5ex, very thick, scale = 0.3] {
	    \draw[am1_col] (0, 0) -- (0.2, 0);
	    \draw[kanr_col] (0.2, 0) -- (1.8, 0);
	    \draw[genome_col] (1.8, 0) -- (2, 0);
      } ainsi que la région facilitant l'homologie \tikz[baseline = -0.5ex, very thick, scale = 0.3, xshift=-7.8] {
	    \draw[genome_col] (7.8, 0) -- (8.0, 0);
	    \draw[ancre_don_col] (8.0, 0) -- (9.8, 0) ;
	    \draw[am3_col] (9.8, 0) -- (10.0, 0);
      }
      sont préalablement amplifiées par PCR en utilisant des amorces porteuses de
      régions 3' flottantes, complémentaires des extrémités des fragments
      voisins. La ligature entre les trois fragments obtenus est réalisée à
      l'aide du kit Takara InFusion. Les constructions plasmidiques obtenues
      sont amplifiées clonalement par culture en milieu sélectif liquide, et
      linéarisées par digestion enzymatique.
    };


    %% Représente le génome d'acinetobacter en rouge, avec la construction
    %% associée.
    \begin{scope}[shift={(-5, 1)}]
      % représente la construction, avec l'insertion de la cassette de résistance
      \draw[gs_don_col] (4, 0) -- (6, 0) node[above, midway, font=\tiny] {Gène cible};
      \draw[am1_col, ultra thick] (6, 0) -- (6.2, 0);
      \draw[kanr_col] (6.2, 0) -- (7.8, 0) node[above, midway, font=\tiny] {KanR};
      \draw[genome_col, ultra thick] (7.8, 0) -- (8.0, 0);
      \draw[ancre_don_col] (8.0, 0) -- (9.8, 0) node[above, midway, font=\tiny] {Ancre};
      \draw[am3_col, ultra thick] (9.8, 0) -- (10.0, 0);
      \draw[plasmid_col, thick] (2, 0) -- (4, 0);

      \draw[arrows = {-Stealth[left]}, plasmid_col, thick] (10, 0) -- (12, 0)
      node[font = \tiny, above, align = right]
      {Construction \\ plasmidique};

      \draw[plasmid_col, dotted, thin, out=0, in = 15]
      [->] (12.1, 0) to (10.8, -4.6)
      node[Gray, legende, right]
      {Transformation};

      % Génome acinetobacter
      % brin 5' 3'
      \draw[gs_rec_col] (5, -1) -- (7, -1);
      \draw[ancre_don_col] (7.0, -1) -- (8.8, -1);
      \draw[genome_col, thick] (2, -1) -- (5, -1);
      \draw[arrows = {-Stealth[left]}, genome_col, thick] (8.8, -1) -- (12, -1) node[font=\tiny, above, align = right] {Génome \\ \emph{Acinetobacter}};
      %amorces
      \draw[Red3, ultra thick] (4, -1) -- (4.2, -1) node[above left, font=\tiny] {Amorce};
      \draw[gs_rec_col] (5, -1.2) -- (7, -1.2);
      \draw[ancre_don_col] (7.0, -1.2) -- (8.8, -1.2);
      \draw[arrows = {Stealth[right, swap]-}, genome_col, thick] (2, -1.2) -- (5, -1.2);
      \draw[genome_col, thick] (8.8, -1.2) -- (12, -1.2);

      % Lignes reliant les fragments
      \draw[gs_don_col, thin, dotted] (4, 0) -- (5, -1);
      \draw[gs_don_col, thin, dotted] (6, 0) -- (7, -1);
      \draw[ancre_don_col,  thin, dotted] (8, 0) -- (7, -1);
      \draw[ancre_don_col,  thin, dotted] (9.8, 0) -- (8.8, -1);

      % petit éclair pour marquer la cassure
      \node[Gold2, font=\Large] at (7, -1.1) {\Lightning};
      \node[Gray, font=\tiny] at (7, -1.4) {Cassure Double Brin};


      %%
      %% RÉSECTION
      %%
      \draw[Gray] [->] (7, -1.6) -- (7, -2.4) node[legende, left, midway] {Résection};

      % brin ->
      \draw[gs_rec_col, arrows = {-Stealth[left, scale = 1]}, very thick] (5, -2.5) -- (7, -2.5);
      \draw[ancre_don_col] (8.3, -2.5) -- (8.8, -2.5);
      \draw[genome_col, thick] (2, -2.5) -- (5, -2.5);
      \draw[arrows = {-Stealth[left]}, genome_col, thick] (8.8, -2.5) -- (12, -2.5);
      % brin <-
      \draw[gs_rec_col, ultra thick] (5, -2.7) -- (5.5, -2.7);
      \draw[arrows = {Stealth[right, swap, scale = 1]-}, ancre_don_col, very thick] (7.0, -2.7) -- (8.8, -2.7);
      \draw[arrows = {Stealth[right, swap]-}, genome_col, thick] (2, -2.7) -- (5, -2.7);
      \draw[genome_col, thick] (8.8, -2.7) -- (12, -2.7);

      %%
      %% RÉPARATION
      %%
      \draw[Gray] [->] (7, -2.8) -- (7, -3.4) node[legende, midway, left] {Réparation};

      \draw[gs_rec_col, arrows = {-Stealth[left, scale = 1]}, very thick] (5, -3.5) -- (7, -3.5);
      \draw[ancre_don_col] (8.3, -3.5) -- (8.8, -3.5);
      \draw[genome_col, thick] (2, -3.5) -- (5, -3.5);
      \draw[arrows = {-Stealth[left]}, genome_col, thick] (8.8, -3.5) -- (12, -3.5);
      % brin <-
      \draw[gs_rec_col, ultra thick] (5, -3.7) -- (5.5, -3.7);
      \draw[ ancre_don_col, very thick] (8.8, -3.7) -- ++(-0.5, 0) -- ++(-0.2, -0.6) -- ++(-0.7, 0);
      \draw[arrows = {Stealth[right, swap]-}, genome_col, thick] (2, -3.7) -- (5, -3.7);
      \draw[genome_col, thick] (8.8, -3.7) -- (12, -3.7);

      %% fragment insert avec boucle pour cassette de résistance
      % je place l'origine de la construction au début da la région ancre sur le
      % génome d'acinetobacter. C'est plus facile pour s'aligner
      \draw[ancre_don_col, very thick]                     (8.8, -4.6) -- ++(-2, 0);
      \draw[genome_col, very thick]                  (6.8, -4.6) -- ++(-0.2, 0);
      \draw[kanr_col, very thick]                     (6.6, -4.6) -- ++(-2, 0);
      \draw[am1_col, very thick]                       (4.6, -4.6) -- ++(-0.2, 0);
      \draw[gs_don_col, very thick]                    (4.4, -4.6) -- ++(-2, 0);
      \draw[plasmid_col, thick]                            (2.4, -4.6) -- ++(-0.4, 0);
      \draw[plasmid_col, thick, arrows = {-Stealth[left]}] (8.8, -4.6) -- ++(2, 0);


      %%
      %% RESYNTHÈSE
      %%

      \draw[densely dotted, ancre_don_col, very thick]    (7.35, -4.3) --  (6.8, -4.3);
      \draw[densely dotted, genome_col, very thick] (6.8, -4.3) -- ++(-0.2, 0);
      \draw[densely dotted, kanr_col, very thick]    (6.6, -4.3) -- ++(-2, 0);
      \draw[densely dotted, am1_col, very thick]      (4.6, -4.3) -- ++(-0.2, 0);
      \draw[densely dotted, gs_don_col, very thick, arrows = {-Stealth[left, scale = 0.5]}]   (4.4, -4.3) -- ++(-1.5, 0);
      %% lignes permettant de montrer l'appariemment des régions
      \draw[gs_don_col, thin, dotted ] (2.9, -4.3) -- (5, -3.5);
      \draw[gs_don_col, thin, dotted ] (4.4, -4.3) -- (7, -3.5);

      %%
      %% RÉAPPARIEMMENT
      %%

      \draw[Gray] [->] (7, -4.8) -- ++(0, -1) node[midway, left, legende] {Réappariemment};

      % brin ->
      \draw[gs_rec_col, arrows = {-Stealth[left, scale = 1]}, very thick] (5, -6) -- (7, -6);
      \draw[ancre_don_col, very thick] (11.4, -6) -- ++(-0.5, 0);
      \draw[genome_col, thick, arrows = {-Stealth[left]}] (11.4, -6) -- ++(1, 0);
      % \draw[ancre_don_col] (8.3, -6) -- (8.8, -6);
      \draw[genome_col, thick] (2, -6) -- (5, -6);
      %% brin <-
      % petit bout bleu
      \draw[gs_rec_col, ultra thick]                        (5, -6.2)   --   (5.5, -6.2);
      \draw[gs_don_col, densely dotted, very thick, arrows = {Stealth[left, scale = 0.5]-}] (6.0, -6.2) -- ++(1.0, 0);
      \draw[densely dotted, am1_col, very thick]            (7.0, -6.2) -- ++(0.2, 0);
      \draw[densely dotted, kanr_col, very thick]          (7.2, -6.2) -- ++(2.0, 0);
      \draw[densely dotted, genome_col, very thick]       (9.2, -6.2) -- ++(0.2, 0);
      \draw[densely dotted, ancre_don_col, very thick]          (9.4, -6.2) --   (9.9, -6.2);
      \draw[solid, ancre_don_col, very thick]                   (9.9, -6.2) -- ++(1.5, 0);
      % génome
      \draw[genome_col, thick]                            (11.4, -6.2) -- ++(1, 0);
      \draw[genome_col, thick, arrows = {Stealth[left]-}] (2, -6.2) --      (5, -6.2);

      %%
      %% SYNTHÈSE DU BRIN 5'
      %%
      \draw[Gray] [->] (7, -6.5) -- ++(0, -1)
      node[midway, left, legende, align = right] {Synthèse du brin \\ complémentaire };

      %% brin ->
      \draw[gs_rec_col, arrows = {-Stealth[left, scale = 1]}, very thick]
      (5, -8.0) -- (7, -8.0);
      \draw[ancre_don_col, very thick]
      (11.4, -8.0) -- ++(-0.5, 0);
      \draw[genome_col, thick, arrows = {-Stealth[left]}]
      (11.4, -8.0) -- ++(1, 0);
      \draw[genome_col, thick]
      (2, -8.0) -- (5, -8.0);
      \draw[densely dotted, am1_col, very thick]
      (7.0, -8.0) -- ++(0.2, 0);
      \draw[densely dotted, kanr_col, very thick]
      (7.2, -8.0) -- ++(2.0, 0);
      \draw[densely dotted, genome_col, very thick]
      (9.2, -8.0) -- ++(0.2, 0);
      \draw[densely dotted, ancre_don_col, very thick]
      (9.4, -8.0) -- ++(1.5, 0);

      %% brin <-
      \draw[gs_rec_col, very thick]
      (5, -8.2)   --   (5.5, -8.2);
      \draw[gs_don_col, solid, very thick]
      (6.0, -8.2) -- ++(1.0, 0);
      \draw[gs_rec_col, densely dotted, very thick, arrows = {-Stealth[left, scale = 0.5]}]
      (6.0, -8.2) -- ++(-0.5, 0);
      \draw[solid, am1_col, very thick]
      (7.0, -8.2) -- ++(0.2, 0);
      \draw[solid, kanr_col, very thick]
      (7.2, -8.2) -- ++(2.0, 0);
      \draw[solid, genome_col, very thick]
      (9.2, -8.2) -- ++(0.2, 0);
      \draw[solid, ancre_don_col, very thick]
      (9.4, -8.2) --   (9.9, -8.2);
      \draw[solid, ancre_don_col, very thick]
      (9.9, -8.2) -- ++(1.5, 0);
      % génome
      \draw[genome_col, thick]
      (11.4, -8.2) -- ++(1, 0);
      \draw[genome_col, thick, arrows = {Stealth[left]-}]
      (2, -8.2) -- (5, -8.2);


      %%
      %% RÉSULTAT FINAL
      %%

      \draw[Gray] [->] (7, -8.5) -- ++(0, -1) node[midway, left, legende] {Ligature};

      \draw[gs_rec_col, very thick]
      (5, -10.0) -- (7, -10.0);
      \draw[ancre_don_col, very thick]
      (11.4, -10.0) -- ++(-0.5, 0);
      \draw[genome_col, thick, arrows = {-Stealth[left]}]
      (11.4, -10.0) -- ++(1, 0);
      \draw[genome_col, thick]
      (2, -10.0) -- (5, -10.0);
      \draw[solid, am1_col, very thick]
      (7.0, -10.0) -- ++(0.2, 0);
      \draw[solid, kanr_col, very thick]
      (7.2, -10.0) -- ++(2.0, 0);
      \draw[solid, genome_col, very thick]
      (9.2, -10.0) -- ++(0.2, 0);
      \draw[solid, ancre_don_col, very thick]
      (9.4, -10.0) -- ++(1.5, 0);

      %% brin <-
      \draw[gs_rec_col, very thick]
      (5, -10.2)   -- (5.5, -10.2);
      \draw[gs_don_col, solid, very thick]
      (6.0, -10.2) -- ++(1.0, 0);
      \draw[gs_rec_col, solid, very thick]
      (6.0, -10.2) -- ++(-0.5, 0);
      \draw[solid, am1_col, very thick]
      (7.0, -10.2) -- ++(0.2, 0);
      \draw[solid, kanr_col, very thick]
      (7.2, -10.2) -- ++(2.0, 0);
      \draw[solid, genome_col, very thick]
      (9.2, -10.2) -- ++(0.2, 0);
      \draw[solid, ancre_don_col, very thick]
      (9.4, -10.2) --   (9.9, -10.2);
      \draw[solid, ancre_don_col, very thick]
      (9.9, -10.2) -- ++(1.5, 0);
      % génome
      \draw[genome_col, thick]
      (11.4, -10.2) -- ++(1, 0);
      \draw[genome_col, thick, arrows = {Stealth[left]-}]
      (2, -10.2) -- (5, -10.2);

      %% rectangle de zoom
	    \draw[Gray, thick, densely dotted, fill = Gray, opacity = 0.3]
      (6, -10.4) rectangle (7, -9.8);

      \draw[Gray, very thick, densely dotted, opacity = 0.6, bend right = 5]
      [->] (7, -10.4) to (14.75, -10.75);

	    \draw[Gray, thick, densely dotted]
      (4, -10.6) rectangle (7.6, -9.6);

      \draw[Gray, very thick, densely dotted, opacity = 0.6, out = 180, in = 180]%, bend right = 75]
      [->] (4, -10.6) to (3.8, -17);
    \end{scope}


    \begin{scope}[shift={(10, -10)}, scale = 0.5]
      %% Zoom montrant l'alignement des positions donneurs avec les
      %% positions receveuse
      \draw[Gray, thick, densely dotted, fill = Gray, opacity = 0.3] (-0.5, -1.1) rectangle ++(11, 3.2);

      \draw[gs_rec_col, thick, arrows = {-Stealth[left]}] (0, 1) -- ++(10, 0)  node[above, font=\scriptsize] {Brin receveur};
      \draw[gs_don_col, thick, arrows = {Stealth[left]-}] (0, 0) -- ++(10, 0) node[below, font=\scriptsize] {Brin donneur};

      \foreach \x in {1,2,...,9}
      {
        \draw[color = gs_rec_col, fill = gs_rec_col] (\x, 1) circle (0.05);
        \draw[color = gs_don_col, fill = gs_don_col] (\x, 0) circle (0.05);
      }

      \draw[Gray, very thick] [->] (5, -1.5) -- ++(0, -1)
      node[midway, left, font=\tiny, align = right]
      {Correction des \\ mésappariemments};
    \end{scope}

    \begin{scope}[shift={(9, -13.5)}, scale = 0.8]
      %% Zoom montrant la correction des mésappariemments dans les clones

     \draw[gs_rec_col, thick, dotted, fill = gs_rec_col, opacity = 0.3]
     (-0.5, -0.5) rectangle (4.5, 2.5)
     node[midway, opacity = 1, font = \scriptsize] {Région conservée};

     \draw[gs_don_col, thick, dotted, fill = gs_don_col, opacity = 0.3]
     (4.5, -0.5) rectangle (10.5, 2.5)
     node[midway, opacity = 1, font = \scriptsize] {Région convertie};

      \draw[gs_rec_col, thick, arrows = {-Stealth[left]}] (0, 2) -- ++(10, 0);
      \draw[gs_don_col, thick, arrows = {Stealth[left]-}] (0, 0) -- ++(10, 0);

      \foreach \x in {4,5,...,9}
      {
        \draw[color = gs_don_col, fill = gs_don_col] (\x, 2) circle (0.05);
        \draw[color = gs_don_col, fill = gs_don_col] (\x, 0) circle (0.05);
      }

      \foreach \x in {1,2,...,4}
      {
        \draw[color = gs_rec_col, fill = gs_rec_col] (\x, 2) circle (0.05);
        \draw[color = gs_rec_col, fill = gs_rec_col] (\x, 0) circle (0.05);
      }

      \draw[Gray, densely dash dot, very thick]
      (4.5, 2.6) -- ++(0, -3.6)
      node[below, font=\scriptsize] {Point de recombinaison};
    \end{scope}

    \begin{scope}[shift={(-3, -6)}, line width = 3pt]
      \draw[gs_rec_col]
      (5, -10.0) -- (6.3, -10.0);
      \draw[gs_don_col]
      (6.3, -10.0) -- (7.0, -10.0);
      \draw[am1_col, line width = 1pt, arrows = {<-}]
      (7.0, -10.0) -- ++(0.4, 0);
      \draw[genome_col, line width = 1.5pt]
      (2, -10.0) -- (5, -10.0);
      \draw[Red3, line width = 1pt, arrows = {->}]
      (4.0, -10.0) -- ++(0.4, 0);

      \draw[Gray, dotted, thick, out=-90, in=0]
      [->] (12.5, -8) to (7.6, -10);
    \end{scope}

    \node[text width=6.5cm, legende text, align = left] at (14, -4) {%
      % 14 lignes possibles
      Les constructions linéarisées sont assimilées par \emph{Acinetobacter} en
      phase exponentielle de croissance. Les cassures doubles-brins occasionnées
      aléatoirement au cours de la réplication au locus d'intérêt entraînent la
      résection de la région en \(3'\) du site de cassure. Cette résection est
      réparée par synthèse utilisant une matrice d'ADN homologue. En cultivant
      les transformants sur un milieu sélectif, nous avons sélectionné les
      clones qui ont utilisé la construction plasmidique pour matrice, ayant
      intégré la cassette de résistance à la kanamycine.

      Au cours du réappariemment des brins, un hétéroduplex se forme : les brins
      homologues ne sont pas rigoureusement complémentaires. Le donneur
      \tikz[baseline = -0.5ex, very thick, scale = 0.3]{\draw[gs_don_col] (0, 0)
        -- ++(1, 0);} introduit des mésappariemments avec la séquence du
      receveur \tikz[baseline = -0.5ex, very thick, scale =
      0.3]{\draw[gs_rec_col] (0, 0) -- ++(1, 0);} ; ils doivent être corrigés.
    };

    \node[text width = 7.4cm, legende text, align = left] at (3.2, -12.8) {%
      % 6 lignes possibles
      La correction des bases mésappariées entraîne de la \emph{conversion
        génique}. La région convertie a le même haplotype que celui du donneur,
      la région conservée conserve celui du receveur. Elles sont séparées par le
      point de recombinaison.

      Le brin donneur introduit des mésappariemments toutes les \(30\) paires de
      bases. Leur correction entraîne la conversion du site soit vers A ou T,
      soit vers G ou C, selon la construction donneuse.

      L'amplification spécifique de la région convertie puis le séquençage des
      amplicons permet de déterminer la polarité des évènements de conversion et
      la position des points de recombinaison.
      %
    };
  \end{tikzpicture}


\end{center}

\caption[Constructions moléculaires]{\textbf{Constructions moléculaires et
    préparation de l'ADN recombinant} \\ \rmfamily Les gènes de synthèse sont
  amplifiés par PCR spécifique. Les fragments sont ensuite ligaturés dans le
  plasmide pGEM-T. Les plasmides obtenus sont linéarisés par digestions
  enzymatiques \emph{Spe}I au
  niveau des sites d'insertion de la cassette de résistance à la kanamycine et de
  l'ancre. Ces fragments sont préalablement amplifiés par PCR en utilisant des
  amorces porteuses de régions 3' flottantes, complémentaires des extrémités des
  fragments voisins. La ligature entre les trois fragments obtenus est réalisée à
  l'aide du kit InFusion. Les constructions plasmidiques obtenues sont amplifiées
  par culture, extraites et linéarisées par digestion enzymatique \emph{Sca}I.
  \label{fig:construct}
}
% ------------------------------------------------------------
\vfill

\thispagestyle{empty}
\addtocounter{page}{-1}
\newpage
% fin de la page de figure
% ==============================================================================

% ==============================================================================
\section{Matériels \& Méthodes}
\label{sec:materiel}
% ==============================================================================

\afterpage{%
  \null
  \vfill
  \begin{center}
  \rmfamily
  \setstretch{1.0}

  \tikzset{
    culture/.pic={
      \node at (0, 0) {\includegraphics[width = 0.015\textwidth]{img/lb_liquide.png}};
    }
  }

  \tikzset{
    transfo/.pic={
      \node at (0, 0) {\includegraphics[width = 0.03\textwidth]{img/eppendorf_full.png}};
    }
  }

  \tikzset{
    etalement/.pic={
      \node at (0, 0) {\includegraphics[width = 0.05\textwidth]{img/etalement.png}};
    }
  }

  \tikzset{
    petri/.pic={
      \node at (0, 0) {\includegraphics[width = 0.05\textwidth]{img/petri_open.png}};
    }
  }

  \tikzset{
    boite/.pic={
      \node at (0, 0) {\includegraphics[width = 0.06\textwidth]{img/boite_96.png}};
    }
  }

  \tikzset{
    pcr plate/.pic={
      \node at (0, 0) {\includegraphics[width = 0.05\textwidth]{img/pcr_plate.jpeg}};
    }
  }

  \tikzset{legende fleche/.style={Gray, dotted, thick, opacity = 0.6}}
  \tikzset{legende text/.style={black, text width = 7cm, font = \scriptsize, above}}

  \begin{tikzpicture}

    % tube eppendorf seul
    \draw [->]
    (0, 0) node[above]
    {\includegraphics[width=0.05\textwidth]{img/eppendorf.png}}
    % boite de petri
    -- (0, -1) node[below]
    {\includegraphics[width=0.2\textwidth]{img/petri_open.png}}
    ;

    %% tubes de culture
    \draw[out = 210, in = 90] [->] (0, -3) to ++(-3.5, -1) pic[below] {culture};
    \draw[out = 210, in = 90] [->] (0, -3) to ++(-2.5, -1) pic[below] {culture};
    \draw[out = -90, in = 90] [->] (0, -3) to ++(-0.5, -1) pic[below] {culture};
    \draw[out = -90, in = 90] [->] (0, -3) to ++(+0.5, -1) pic[below] {culture};
    \draw[out = -30, in = 90] [->] (0, -3) to ++(+2.5, -1) pic[below] {culture};
    \draw[out = -30, in = 90] [->] (0, -3) to ++(+3.5, -1) pic[below] {culture};

    %% tubes de transfo
    \draw [->] (-3.5, -7) node[above, font = \scriptsize] {GC} -- ++(0, -0.7) pic[below] {transfo};
    \draw [->] (-2.5, -7) node[above, font = \scriptsize] {GC} -- ++(0, -0.7) pic[below] {transfo};
    \draw [->] (-0.5, -7) node[above, font = \scriptsize] {AT} -- ++(0, -0.7) pic[below] {transfo};
    \draw [->] (+0.5, -7) node[above, font = \scriptsize] {AT} -- ++(0, -0.7) pic[below] {transfo};
    \draw [->] (+2.5, -7) node[above, font = \scriptsize] {\(\nicefrac{GC}{AT}\)} -- ++(0, -0.7) pic[below] {transfo};
    \draw [->] (+3.5, -7) node[above, font = \scriptsize] {\(\nicefrac{AT}{GC}\)} -- ++(0, -0.7) pic[below] {transfo};

    %% étalement sur boiîtes de pétri
    \draw [->] (-3.5, -9) -- ++(0, -0.5) pic[below] {petri};
    \draw [->] (-2.5, -9) -- ++(0, -0.5) pic[below] {petri};
    \draw [->] (-0.5, -9) -- ++(0, -0.5) pic[below] {petri};
    \draw [->] (+0.5, -9) -- ++(0, -0.5) pic[below] {petri};
    \draw [->] (+2.5, -9) -- ++(0, -0.5) pic[below] {petri};
    \draw [->] (+3.5, -9) -- ++(0, -0.5) pic[below] {petri};

    %% purification sur boîtes
    \draw [->] (-3.5, -10.5) -- ++(0, -0.5) pic[below] {etalement};
    \draw [->] (-2.5, -10.5) -- ++(0, -0.5) pic[below] {etalement};
    \draw [->] (-0.5, -10.5) -- ++(0, -0.5) pic[below] {etalement};
    \draw [->] (+0.5, -10.5) -- ++(0, -0.5) pic[below] {etalement};
    \draw [->] (+2.5, -10.5) -- ++(0, -0.5) pic[below] {etalement};
    \draw [->] (+3.5, -10.5) -- ++(0, -0.5) pic[below] {etalement};

    %% suspension dans 50µL d'eau
    \draw [->] (-3.5, -12.2) -- ++(0, -0.5) pic[below] {boite};
    \draw [->] (-2.5, -12.2) -- ++(0, -0.5) pic[below] {boite};
    \draw [->] (-0.5, -12.2) -- ++(0, -0.5) pic[below] {boite};
    \draw [->] (+0.5, -12.2) -- ++(0, -0.5) pic[below] {boite};
    \draw [->] (+2.5, -12.2) -- ++(0, -0.5) pic[below] {boite};
    \draw [->] (+3.5, -12.2) -- ++(0, -0.5) pic[below] {boite};

    %% PCR
    \draw [->] (-3.5, -13.6) -- ++(0, -0.5) pic[below] {pcr plate};
    \draw [->] (-2.5, -13.6) -- ++(0, -0.5) pic[below] {pcr plate};
    \draw [->] (-0.5, -13.6) -- ++(0, -0.5) pic[below] {pcr plate};
    \draw [->] (+0.5, -13.6) -- ++(0, -0.5) pic[below] {pcr plate};
    \draw [->] (+2.5, -13.6) -- ++(0, -0.5) pic[below] {pcr plate};
    \draw [->] (+3.5, -13.6) -- ++(0, -0.5) pic[below] {pcr plate};

    %% gels électrophorèse
    \draw [->] (-3.5, -15) -- ++(0, -0.5) node[below] {\includegraphics[width = 0.06\textwidth]{img/s1_plate.jpg}};
    \draw [->] (-2.5, -15) -- ++(0, -0.5) node[below] {\includegraphics[width = 0.06\textwidth]{img/s3_plate.jpg}};
    \draw [->] (-0.5, -15) -- ++(0, -0.5) node[below] {\includegraphics[width = 0.06\textwidth]{img/w1_plate.jpg}};
    \draw [->] (+0.5, -15) -- ++(0, -0.5) node[below] {\includegraphics[width = 0.06\textwidth]{img/w2_plate.jpg}};
    \draw [->] (+2.5, -15) -- ++(0, -0.5) node[below] {\includegraphics[width = 0.06\textwidth]{img/sw_plate.jpg}};
    \draw [->] (+3.5, -15) -- ++(0, -0.5) node[below] {\includegraphics[width = 0.06\textwidth]{img/ws_plate.jpg}};

    %%
    %% ANNOTATIONS
    %%

    % %% trait de légende
    \draw[legende fleche] [<-] (1.5, -1.5) -- (2, -1.5) -- (3, 0) -- (8, 0)
    node[legende text] {%
      Une culture cryogénisée d'\emph{Acinetobacter baylyi ADP1} est purifiée
      par étalement sur milieu non sélectif de Luria Bertani.
      %
    } node {$\bullet$} ;

    \draw[legende fleche] [<-] (4, -4.5) -- (8, -4.5) node[legende text] {%
      Une colonie isolée est pré-cultivée pendant \SI{24}{\hour} à \SI{30}{\celsius}
      en milieu LB. \SI{50}{\uL} de cette pré-culture sont mis en suspension dans
      \SI{5}{\mL} de LB.
      %
    } node {$\bullet$} ;

    \draw[legende fleche] [<-] (4, -8.0) -- (8, -8.0) node[legende text] {%
      Lorsque la culture a atteint une absorbance à \SI{600}{\nm} de \(0.8\),
      \SI{400}{\ng} de construction plasmidique linéarises sont ajoutées dans
      \SI{390}{\uL} de culture, et incubés pendant \SI{1}{\hour} à \SI{30}{\celsius}.

      Les transformations sont réalisées avec les quatres constructions
      différentes~: CG, AT, AT/CG et CG/AT (voir figure~\ref{fig:construct}) %
    } node {$\bullet$} ;

    \draw[legende fleche] [<-] (4, -10.0) -- (8, -10.0) node[legende text] {%
      Après \SI{1}{\hour} d'incubation, les plasmides résiduels sont dégradés par
      l'ajout de DNAse à \SI{20}{\ug\per\mL}, et incubation \SI{15}{\minute} à
      \SI{37}{\celsius}. La culture est ensuite étalée en spirale sur milieu
      sélectif LB additionné de kanamycine à \SI{50}{\ug\per\mL}, et incubés
      pendant \SI{24}{\hour} à \SI{30}{\celsius}.
      %
    } node {$\bullet$} ;

    \draw[legende fleche] [<-] (4, -11.5) -- (8, -11.5) node[legende text]
    {%
      \(96\) colonies transformantes, résistantes à la kanamycines sont isolées
      par étalement sur le même milieu sélectif LB additionné de kanamycine.
      %
    } node {$\bullet$} ;


    \draw[legende fleche] [<-] (4, -13.0) -- (8, -13.0) node[legende text] {%
      Pour chaque transformant isolé, 1 colonie est prélevée à l'œse et
      mise en suspension dans \SI{50}{\uL} d'eau ultrapure.
      %
    } node {$\bullet$} ;

    \draw[legende fleche] [<-] (4, -14.5) -- (8, -14.5) node[legende text] {%
      Des PCRs cibles de la région d'intérêt sont réalisées en utilisant
      \SI{2}{\uL} des suspensions précédemment réalisées pour matrice.
      %
    } node {$\bullet$} ;

    \draw[legende fleche] [<-] (4, -16.2) -- (8, -16.2) node[legende text] {%
      Les PCRs sont contrôlées par électrophorèse sur gel d'agarose à
      \(1\%\). Les amplicons sont ensuite séquencés par la technique de
      Sanger.
      %
    } node {$\bullet$} ;

  \end{tikzpicture}

\end{center}
  \caption[Protocoles de transformation]{\textbf{Protocole de transformation et
      d'obtention des amplicons des zones de recombinaison.} \rmfamily%
    \setstretch{1.1} %
  }
  \label{fig:manip}
  \vfill
  \thispagestyle{empty}
  \addtocounter{page}{-1}
  \newpage
}


% ------------------------------------------------------------------------------
\subsection{Conception des séquences de synthèse}
\label{subsec:concept}
% ------------------------------------------------------------------------------

Pour étudier la correction des mésappariemments au cours de la recombinaison
homologue chez \emph{A. baylyi}, nous avons conçu des séquences contenant des
sites variant par rapport au génome receveur. Nous avons choisi un locus neutre
du génome connu au laboratoire pour permettre de fortes efficacités de
transformation. Il code putativement une 3-oxoacyl-ACP reductase, impliqué dans
la synthèse des acides gras\cite{vallenet_microscopeintegrated_2013} et est
localisé à la position \num{47184} du génome de référence. Ce locus mesure
\num{800} paires de bases dans lesquelles nous avons substitué \num{23}
positions réparties toutes les \num{30} paires de bases. La densité en position
que nous avons choisie correspond à un compromis entre l'efficacité de la
recombinaison et la précision de la détection des évènements de conversion ; une
densité plus forte diminue les fréquences de transformation, une densité plus
faible réduit le nombre d'évènements détectables.

Nous avons conçu 4 gènes de synthèse, nommés respectivement dCG\footnote{dCG
  pour donneur GC.}, dAT, dCG/AT et dAT/CG (voir figure \ref{fig:construct}). Le
gène dAT introduit uniquement des bases A et T ; le gène dGC introduit
uniquement des bases G et C. Les sites variants dans le gène dAT sont adjacents
aux sites variants dans le gène GC. Pour s'affranchir d'un éventuel effet dû à
l'écart de \ac{gc} entre les séquences de synthèse et la séquence sauvage, nous
avons également conçu deux séquences qui introduisent en alternance un
mésappariemment GC $|$ AT et AT $|$ GC. Le gène de synthèse dCG/AT introduit
d'abord un mésappariemment GC $|$ AT, puis un mésappariemment AT $|$ GC, le gène
de synthèse dAT/CG suit l'ordre inverse. (voir figure~\ref{fig:construct}). Les
quatre séquences ont été synthétisées par ThermoFischer (Waltham, États-Unis).

% ------------------------------------------------------------------------------
\subsection{Constructions des plasmides}
\label{subsec:constructions}
% ------------------------------------------------------------------------------

De façon à pouvoir sélectionner les clones recombinants au locus d'intérêt, nous
avons construit les plasmides représentés dans la figure \ref{fig:construct}. Le
gène synthétique est associé avec une cassette de résistance à la kanamycine,
ainsi qu'une région ``ancre''. Cette région recombinogène est complètement
homologue à la séquence en 3' du locus d'intérêt et permet d'augmenter les
fréquences de transformation \cite{de_vries_integration_2002,meier_mechanisms_2003}.

Les gènes synthétiques ont d'abord été amplifiés par PCR \footnote{Voir
  l'annexe~\ref{subsec:annexe-amorces} pour le détail des amorces et
  l'annexe~\ref{subsec:annexe-pcr} pour les conditions de PCR} avec les amorces
1392 et 1393. Les plasmides obtenus ont été insérés dans la souche \emph{E.coli}
OneShot\textsuperscript{{\textregistered}} \texttt{TOP-10} (Invitrogen,
Carlsbad, États-Unis), suivant le protocole du fabriquant. Le sens d'insertion
du gène synthétique dans le plasmide a été vérifié par PCR avec les amorces M13R
et 1392, et l'absence de mutation a été vérifié par séquençage (GATC Biotech,
Constance, Allemagne). Les plasmides des clones validés par séquençage ont été
extraits par le kit Nucleospin-Plasmid (Macherey-Nagel, Düren, Allemagne), et
linéarisés par l'enzyme \emph{Spe}I (ThermoFischer).

La cassette de résistance à la kanamycine \emph{aphA}3 et l'ancre ont été
respectivement amplifiés par PCR avec le couple d'amorce 1408 / 1409 et 1410 /
1411. Les deux amplicons obtenus ont été ligaturés dans le plasmide pGEM-T
porteurs des gènes de synthèse. La ligature a été réalisée simultanément par le
kit InFusion (Takara Clontech, Saint Germain en Layes, France). Le produit de
ligature obtenu a été inséré dans la souche optimisée pour la chimio-compétence
\emph{E.coli} Stellar (Takara Clontech). Les transformants ont été séléctionnés
sur milieu LB solide additionné d'ampicilline à \SI{75}{\ug\per\mL} de
kanamycine à \SI{50}{\ug\per\mL}. Les transformants ont été confirmés par PCR
spécifique de l'insert avec les amorces 1393 et 1411.

% ------------------------------------------------------------------------------
\afterpage{%
 \null
 \vfill
  \begin{center}
  \setstretch{1.0}
    \rmfamily

    \tikzset{legende fleche/.style={Gray, dotted, thick, opacity = 0.6}}
    \tikzset{legende text/.style={black, text width = 5cm, font = \scriptsize, above}}
    \tikzset{align fleche/.style={midway, right, align = left, darkgray,
        font=\scriptsize}}

    \begin{tikzpicture}[join=round]

      \draw [->]
       (0, 0) node[above] {\includegraphics[width = 0.4\textwidth]{img/electropherogram.jpg}}
       -- ++(0, -1)
       node[align fleche, text width = 6cm ] {%
         \textsf{{\color{Black} a.}} %
         Analyses des électrophérogrammes obtenus par le programme
         \texttt{phred} et attribution du score de qualité aux bases.
         %
       }
       node[font = \footnotesize, below] {\texttt{\textcolor{kanr_col}{ATCG.......................ACCG}}}
       ;

       \node[font = \footnotesize] at (0, -2.5) {\texttt{\textcolor{gs_rec_col}{ATCG.......................ATCG}}}; % receveur
       \node[font = \footnotesize] at (0, -3.0) {\texttt{\textcolor{gs_don_col}{ACCG.......................ACCG}}}; % donneur
       \node[font = \footnotesize, opacity = 0.7] at (0, -3.5) {\texttt{\textcolor{kanr_col}{ATCG.......................ACCG}}}; % recombinant
       \node[gs_rec_col, font=\scriptsize] at (-4, -2.5) {Receveur :};
       \node[gs_don_col, font=\scriptsize] at (-4, -3.0) {Donneur :};
       \node[kanr_col, font=\scriptsize] at (-4, -3.5) {Recombinant :};

       \draw[bend left=90,  dotted, thick] [->]
       (+3.2, -1.2) to
       node[midway, right, align fleche, align = left, text width = 6cm] {%
         \textsf{{\color{Black} b.}} %
         Alignement avec la référence par \texttt{muscle}. %
         La référence est composée de l'alignement de la séquence receveuse
         sauvage et de la séquence donneuse synthétique.
         %
       } ++(0, -2.2) ;

       \draw[align fleche] [->] (0, -4.0) -- ++(0, -1.5) %
       node[align fleche, midway, right, align = left, text width = 7cm] {%
         \textsf{{\color{Black} c.}} %
         Détermination du sens des événèments de conversion et attribution du
         score de qualité aux bases. L'information de la position sur la
         séquence de référence permet de comparer les lectures de séquençage
         entre elles.} %
       node[font=\tiny, below] {%
         %
         \fontspec{Gill Sans}
         \begin{tabular}{>{\color{gs_rec_col}}c>{\color{gs_don_col}}c>{\color{kanr_col}}cccc}
           \toprule
           Receveur & Donneur & Recombinant & Position & Qualité & Conversion \\
           \midrule
           A        & A       & A           & 31      & 30      &          \\
           \rowcolor{LightGray}
           T        & C       & C           & 32      & 40      & oui      \\
           C        & C       & C           & 33      & 42      &          \\
           G        & G       & G           & 34      & 42      &          \\
           .        & .       & .           & .       & .       &          \\
           .        & .       & .           & .       & .       &          \\
           .        & .       & .           & .       & .       &          \\
           A        & A       & A           & 61      & 42      &          \\
           \rowcolor{LightGray}
           T        & C       & T           & 62      & 30      & non      \\
           C        & C       & C           & 63      & 40      &          \\
           G        & G       & G           & 64      & 28      &          \\
           \bottomrule
           %
         \end{tabular}
       };

       \draw[legende fleche] [<-] (3.8, -6.5) -- (4.4, -6.5) -- (5, -9.0);
       \draw[legende fleche] [<-] (3.8, -8.3) -- (4.4, -8.3) -- (5, -9.0) -- (8, -9.0) %
       node[opacity = 1, Gray, text width = 5.5cm, font=\scriptsize, above] {%
         Ces positions sont les positions d'intérêt. Elles correspondent à
         l'introduction d'une base G ou C par le donneur, alors que le receveur
         présente une base A ou T. La base présente chez le clone recombinant
         permet de déterminer la polarité de la conversion. Le score de qualité
         permet d'accorder plus ou moins de confiance à la base appelée par le
         programme \texttt{phred}.
         %
       } node {$\bullet$} ;


       %%
       %% Flèches montrant le basculement de ligne à colonne.
       %%
       \draw[dotted, kanr_col, thin, bend left ] [->] (+2.9, -3.7) to (-0.5, -8.8);
       \draw[dotted, kanr_col, thin, bend right] [->] (-2.9, -3.7) to (-0.7, -6.3);

  \end{tikzpicture}

\end{center}

  \caption[Analyses des zones de recombinaison]{\textbf{Exemple d'analyse de la
      zone de recombinaison pour un clone
      transformant} \\
    \rmfamily%
    \setstretch{1.1} %
    Les électrophérogrammes de séquençage obtenus ont été analysés en utilisant
    un programme permettant d'attribuer à chaque position un score de qualité (\textsf{a}).
    Les séquences obtenues ont été alignées à la référence (\textsf{b}), ce qui a permis
    d'inférer pour chaque site polymorphe le sens de la conversion du
    recombinant (\textsf{c}).}
  \label{fig:align}
  \vfill
  \thispagestyle{empty}
  \addtocounter{page}{-1}
  \newpage
}
% ------------------------------------------------------------------------------

% -----------------------------------------------------------------------------
\subsection{Transformations d'\emph{Acinetobacter baylyi}}
\label{subsec:transfo}
% -----------------------------------------------------------------------------

\SI{1}{\ug} de plasmide a été extrait et linéarisé par l'enzyme \emph{Sca}I
(ThermoFisher). L'enzyme a été ensuite inactivée par incubation \SI{10}{\minute}
à \SI{70}{\celsius}. \SI{390}{\uL} d'une culture pure d'\emph{Acinetobacter
  baylyi} ADP1, avec une absorbance de \SI{0,8} à \SI{600}{\nm} ont été incubés
pendant \SI{1}{\hour} à \SI{28}{\celsius} en présence de \SI{200}{\ng} de
plasmide linéarisé. Les suspensions ont été ensuite incubées \SI{15}{\minute} à
\SI{37}{\celsius} en présence de DNAse à \SI{20}{\umol\per\L} pour éliminer les
plasmides résiduels. Les cellules recombinantes ont été sélectionnées par
étalement en spirales (InterScience, St Nom la Bretêche, France) sur milieu LB
solide additionné de kanamycine à \SI{50}{\ug\per\mL} et incubées \SI{24}{\hour}
à \SI{30}{\celsius}. Les dénombrements ont été effectués via un compteur
automatique Scan\textsuperscript{\textregistered}1200 (InterScience). Les
fréquences de transformations ont été calculées en divisant le nombre de
cellules transformantes, résistantes à la kanamycine par le nombre de cellules
totales. 96 clones recombinants ont été ensuite isolés et incubés \SI{24}{\hour}
à \SI{30}{\celsius} sur milieu LB solide additionné de kanamycine à
\SI{50}{\umol\per\mL}. Une colonie isolée par clone a été mise en suspension
dans \SI{50}{\uL} d'\ce{H_2O} ultra-pure. \SI{2}{\uL} de ces suspensions ont
servi de matrice pour amplifier par PCR les régions recombinantes avec la Taq
polymérase haute fidélité Phusion \footnote{Voir conditions de PCR dans l'annexe
  \ref{subsec:annexe-pcr}} (ThermoFischer, Waltham, États-Unis). Les amplicons
ont été vérifiés par migration sur gel d'agarose à 1\% et séquencés par la
technique de Sanger\cite{sanger_dna_1977} (GATC Biotech).

% ------------------------------------------------------------------------------
\subsection{Alignements}
\label{subsec:align}
% ------------------------------------------------------------------------------

Les spectrogrammes de séquençage reçus au format propriétaire \texttt{abi}
(Applied Biosystem, Foster City, États-Unis) ont été analysés par le programme
\texttt{phred} \cite{ewing_base-calling_1998} et convertis en format universel
\texttt{FASTA} (voir figure \ref{fig:align}). Les séquences obtenues ont été
alignées aux références par \texttt{ muscle v3.8.31} \cite{edgar_muscle:_2004}.
Les références en question correspondent à la séquence sauvage et la séquence du
gène de synthèse, respectivement \emph{receveur} et \emph{donneur} de
l'évènement de recombinaison. Un programme Python \cite{cock_biopython:_2009} a
été développé pour analyser les alignements obtenus. Il détermine les positions
des SNPs d'intérêt dans l'alignement de référence et infère le génotype du clone
séquencé. Les alignements par paire en colonne obtenus ont été analysés par
\textrm{R} \texttt{3.2.3} \cite{r_core_team_r:_2015}. Les programmes développés
sont accessibles à l'adresse \url{https://github.com/sam217pa/gbc-seq_mars}. Les
données formattées et les fonctions d'analyse ont été assemblées dans le package
\textrm{R} \texttt{gcbiasr} disponible à l'adresse
\url{https://github.com/sam217pa/gbc-gcbiasr}.
