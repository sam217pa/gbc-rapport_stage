

\section{Matériels \& Méthodes}
\label{sec:materiel}

\addfig{%
  \begin{center}
  \rmfamily
  \setstretch{1.0}

  \tikzset{
    culture/.pic={
      \node at (0, 0) {\includegraphics[width = 0.015\textwidth]{img/lb_liquide.png}};
    }
  }

  \tikzset{
    transfo/.pic={
      \node at (0, 0) {\includegraphics[width = 0.03\textwidth]{img/eppendorf_full.png}};
    }
  }

  \tikzset{
    etalement/.pic={
      \node at (0, 0) {\includegraphics[width = 0.05\textwidth]{img/etalement.png}};
    }
  }

  \tikzset{
    petri/.pic={
      \node at (0, 0) {\includegraphics[width = 0.05\textwidth]{img/petri_open.png}};
    }
  }

  \tikzset{
    boite/.pic={
      \node at (0, 0) {\includegraphics[width = 0.06\textwidth]{img/boite_96.png}};
    }
  }

  \tikzset{
    pcr plate/.pic={
      \node at (0, 0) {\includegraphics[width = 0.05\textwidth]{img/pcr_plate.jpeg}};
    }
  }

  \tikzset{legende fleche/.style={Gray, dotted, thick, opacity = 0.6}}
  \tikzset{legende text/.style={black, text width = 7cm, font = \scriptsize, above}}

  \begin{tikzpicture}

    % tube eppendorf seul
    \draw [->]
    (0, 0) node[above]
    {\includegraphics[width=0.05\textwidth]{img/eppendorf.png}}
    % boite de petri
    -- (0, -1) node[below]
    {\includegraphics[width=0.2\textwidth]{img/petri_open.png}}
    ;

    %% tubes de culture
    \draw[out = 210, in = 90] [->] (0, -3) to ++(-3.5, -1) pic[below] {culture};
    \draw[out = 210, in = 90] [->] (0, -3) to ++(-2.5, -1) pic[below] {culture};
    \draw[out = -90, in = 90] [->] (0, -3) to ++(-0.5, -1) pic[below] {culture};
    \draw[out = -90, in = 90] [->] (0, -3) to ++(+0.5, -1) pic[below] {culture};
    \draw[out = -30, in = 90] [->] (0, -3) to ++(+2.5, -1) pic[below] {culture};
    \draw[out = -30, in = 90] [->] (0, -3) to ++(+3.5, -1) pic[below] {culture};

    %% tubes de transfo
    \draw [->] (-3.5, -7) node[above, font = \scriptsize] {GC} -- ++(0, -0.7) pic[below] {transfo};
    \draw [->] (-2.5, -7) node[above, font = \scriptsize] {GC} -- ++(0, -0.7) pic[below] {transfo};
    \draw [->] (-0.5, -7) node[above, font = \scriptsize] {AT} -- ++(0, -0.7) pic[below] {transfo};
    \draw [->] (+0.5, -7) node[above, font = \scriptsize] {AT} -- ++(0, -0.7) pic[below] {transfo};
    \draw [->] (+2.5, -7) node[above, font = \scriptsize] {\(\nicefrac{GC}{AT}\)} -- ++(0, -0.7) pic[below] {transfo};
    \draw [->] (+3.5, -7) node[above, font = \scriptsize] {\(\nicefrac{AT}{GC}\)} -- ++(0, -0.7) pic[below] {transfo};

    %% étalement sur boiîtes de pétri
    \draw [->] (-3.5, -9) -- ++(0, -0.5) pic[below] {petri};
    \draw [->] (-2.5, -9) -- ++(0, -0.5) pic[below] {petri};
    \draw [->] (-0.5, -9) -- ++(0, -0.5) pic[below] {petri};
    \draw [->] (+0.5, -9) -- ++(0, -0.5) pic[below] {petri};
    \draw [->] (+2.5, -9) -- ++(0, -0.5) pic[below] {petri};
    \draw [->] (+3.5, -9) -- ++(0, -0.5) pic[below] {petri};

    %% purification sur boîtes
    \draw [->] (-3.5, -10.5) -- ++(0, -0.5) pic[below] {etalement};
    \draw [->] (-2.5, -10.5) -- ++(0, -0.5) pic[below] {etalement};
    \draw [->] (-0.5, -10.5) -- ++(0, -0.5) pic[below] {etalement};
    \draw [->] (+0.5, -10.5) -- ++(0, -0.5) pic[below] {etalement};
    \draw [->] (+2.5, -10.5) -- ++(0, -0.5) pic[below] {etalement};
    \draw [->] (+3.5, -10.5) -- ++(0, -0.5) pic[below] {etalement};

    %% suspension dans 50µL d'eau
    \draw [->] (-3.5, -12.2) -- ++(0, -0.5) pic[below] {boite};
    \draw [->] (-2.5, -12.2) -- ++(0, -0.5) pic[below] {boite};
    \draw [->] (-0.5, -12.2) -- ++(0, -0.5) pic[below] {boite};
    \draw [->] (+0.5, -12.2) -- ++(0, -0.5) pic[below] {boite};
    \draw [->] (+2.5, -12.2) -- ++(0, -0.5) pic[below] {boite};
    \draw [->] (+3.5, -12.2) -- ++(0, -0.5) pic[below] {boite};

    %% PCR
    \draw [->] (-3.5, -13.6) -- ++(0, -0.5) pic[below] {pcr plate};
    \draw [->] (-2.5, -13.6) -- ++(0, -0.5) pic[below] {pcr plate};
    \draw [->] (-0.5, -13.6) -- ++(0, -0.5) pic[below] {pcr plate};
    \draw [->] (+0.5, -13.6) -- ++(0, -0.5) pic[below] {pcr plate};
    \draw [->] (+2.5, -13.6) -- ++(0, -0.5) pic[below] {pcr plate};
    \draw [->] (+3.5, -13.6) -- ++(0, -0.5) pic[below] {pcr plate};

    %% gels électrophorèse
    \draw [->] (-3.5, -15) -- ++(0, -0.5) node[below] {\includegraphics[width = 0.06\textwidth]{img/s1_plate.jpg}};
    \draw [->] (-2.5, -15) -- ++(0, -0.5) node[below] {\includegraphics[width = 0.06\textwidth]{img/s3_plate.jpg}};
    \draw [->] (-0.5, -15) -- ++(0, -0.5) node[below] {\includegraphics[width = 0.06\textwidth]{img/w1_plate.jpg}};
    \draw [->] (+0.5, -15) -- ++(0, -0.5) node[below] {\includegraphics[width = 0.06\textwidth]{img/w2_plate.jpg}};
    \draw [->] (+2.5, -15) -- ++(0, -0.5) node[below] {\includegraphics[width = 0.06\textwidth]{img/sw_plate.jpg}};
    \draw [->] (+3.5, -15) -- ++(0, -0.5) node[below] {\includegraphics[width = 0.06\textwidth]{img/ws_plate.jpg}};

    %%
    %% ANNOTATIONS
    %%

    % %% trait de légende
    \draw[legende fleche] [<-] (1.5, -1.5) -- (2, -1.5) -- (3, 0) -- (8, 0)
    node[legende text] {%
      Une culture cryogénisée d'\emph{Acinetobacter baylyi ADP1} est purifiée
      par étalement sur milieu non sélectif de Luria Bertani.
      %
    } node {$\bullet$} ;

    \draw[legende fleche] [<-] (4, -4.5) -- (8, -4.5) node[legende text] {%
      Une colonie isolée est pré-cultivée pendant \SI{24}{\hour} à \SI{30}{\celsius}
      en milieu LB. \SI{50}{\uL} de cette pré-culture sont mis en suspension dans
      \SI{5}{\mL} de LB.
      %
    } node {$\bullet$} ;

    \draw[legende fleche] [<-] (4, -8.0) -- (8, -8.0) node[legende text] {%
      Lorsque la culture a atteint une absorbance à \SI{600}{\nm} de \(0.8\),
      \SI{400}{\ng} de construction plasmidique linéarises sont ajoutées dans
      \SI{390}{\uL} de culture, et incubés pendant \SI{1}{\hour} à \SI{30}{\celsius}.

      Les transformations sont réalisées avec les quatres constructions
      différentes~: CG, AT, AT/CG et CG/AT (voir figure~\ref{fig:construct}) %
    } node {$\bullet$} ;

    \draw[legende fleche] [<-] (4, -10.0) -- (8, -10.0) node[legende text] {%
      Après \SI{1}{\hour} d'incubation, les plasmides résiduels sont dégradés par
      l'ajout de DNAse à \SI{20}{\ug\per\mL}, et incubation \SI{15}{\minute} à
      \SI{37}{\celsius}. La culture est ensuite étalée en spirale sur milieu
      sélectif LB additionné de kanamycine à \SI{50}{\ug\per\mL}, et incubés
      pendant \SI{24}{\hour} à \SI{30}{\celsius}.
      %
    } node {$\bullet$} ;

    \draw[legende fleche] [<-] (4, -11.5) -- (8, -11.5) node[legende text]
    {%
      \(96\) colonies transformantes, résistantes à la kanamycines sont isolées
      par étalement sur le même milieu sélectif LB additionné de kanamycine.
      %
    } node {$\bullet$} ;


    \draw[legende fleche] [<-] (4, -13.0) -- (8, -13.0) node[legende text] {%
      Pour chaque transformant isolé, 1 colonie est prélevée à l'œse et
      mise en suspension dans \SI{50}{\uL} d'eau ultrapure.
      %
    } node {$\bullet$} ;

    \draw[legende fleche] [<-] (4, -14.5) -- (8, -14.5) node[legende text] {%
      Des PCRs cibles de la région d'intérêt sont réalisées en utilisant
      \SI{2}{\uL} des suspensions précédemment réalisées pour matrice.
      %
    } node {$\bullet$} ;

    \draw[legende fleche] [<-] (4, -16.2) -- (8, -16.2) node[legende text] {%
      Les PCRs sont contrôlées par électrophorèse sur gel d'agarose à
      \(1\%\). Les amplicons sont ensuite séquencés par la technique de
      Sanger.
      %
    } node {$\bullet$} ;

  \end{tikzpicture}

\end{center}
  \caption[Protocoles de transformation]{\textbf{Protocole de transformation et
      d'obtention des amplicons des zones de recombinaison.} \rmfamily%
    \setstretch{1.1} %
  }
  \label{img:manip}
}

\subsection{Constructions des plasmides}
\label{subsec:constructions}


Nous avons utilisé deux approches. La première consiste à introduire
exclusivement des bases GC ou AT. (voir figure \label{img:strategies}). Sous
l'hypothèse d'une conversion biaisée vers GC, les régions converties devraient
être plus longues lorsqu'on introduit exclusivement des bases GC que lorsqu'on
introduit des bases AT. Sur un locus du génome d'\emph{Acinetobacter
  baylyi}, 23 dinucléotides séparés par trente bases ont été choisis. Ce sont
tous des dinucléotides alternant AT et GC. Un gène de synthèse a été conçu pour
introduire un mésappariemment entre la base A ou T du génome et une base G ou C
: cette construction sera appelée GC dans la suite de ce rapport. La
construction AT a été conçue de façon à introduire un mésappariemment entre les
bases G ou C du génome et une base A ou T.

  \begin{center}
    \rmfamily \footnotesize
    \setstretch{1.0}
    \begin{tikzpicture}
      \draw[gs_don_col] [->] (0, 0) -- ++(0.5, +0.5) -- ++(0.5, 0)
      node[above = 3pt, midway, draw, circle, fill, opacity = 0.3, font=\tiny] {1}
      node[right, align = left] {Régions\\converties ?};
      \draw[kanr_col]   [->] (0, 0) -- ++(0.5, -0.5) -- ++(0.5, 0)
      node[above = 3pt, midway, draw, circle, fill, opacity = 0.3, font=\tiny] {2}
      node[right, align = left] {Rapport\\en base ?};

      \draw[gs_don_col] [->] (2.7, +0.5) -- ++(0.5, +0.25) -- ++(0.5, 0)
      node[right] {\texttt{--C--G--C--G--} : Construction GC};
      \draw[gs_don_col] [->] (2.7, +0.5) -- ++(0.5, -0.25) -- ++(0.5, 0)
      node[right] {\texttt{-A--T--A--T---} : Construction AT};
      \draw[kanr_col]   [->] (2.7, -0.5) -- ++(0.5, +0.25) -- ++(0.5, 0)
      node[right] {\texttt{--C-T---C-T---} : Construction CG/AT};
      \draw[kanr_col]   [->] (2.7, -0.5) -- ++(0.5, -0.25) -- ++(0.5, 0)
      node[right] {\texttt{-A---G-A---G--} : Construction AT/CG};

      \node[right,gs_rec_col] at (3.7, 1.2) {\texttt{-CA-CA-CA-CA--} : Séquence sauvage};
    \end{tikzpicture}
    \caption[Schéma des stratégies]{\label{img:strategies} \textbf{Schéma des deux approches et des constructions utilisées} \\
      \rmfamily%
      \setstretch{1.1} %
      Des dinucléotides ont été choisis sur la séquence sauvage pour alterner AT
      et CG. La construction CG introduit uniquement des bases C et G, la
      construction AT des bases A et T, la construction AT/CG alterne AT et CG,
      et la construction CG/AT alterne CG et AT. }
  \end{center}

La seconde approche consiste à introduire à la fois des bases AT et CG. En
l'absence de biais, le rapport en base ne devrait pas être affecté : on
introduit autant de bases GC que de bases AT. Deux constructions ont donc été
conçues sur la base des constructions exclusivement GC ou AT. La construction
GC/AT alterne l'introduction d'une base GC avec celle d'une base AT. La
construction AT/GC alterne l'introduction d'une base AT avec celle d'une base
GC.

Une première partie du travail a consisté à construire le vecteur permettant
d'introduire le gène synthétique porteur des mésappariemments chez
\emph{Acinetobacter} par transformation naturelle. Les gènes synthétiques ont
été synthétisés par quelqu'un.
%% TODO qui a synthétisé les gènes ?
Les PCRs spécifiques du gène d'intérêt ont été réalisées. Les amplicons ainsi
obtenus ont été insérés par ligature dans le plasmide pGEM-T. Le plasmide a été
inséré dans la souche d'\textit{E.coli} One
Shot\textsuperscript{{\textregistered}} \texttt{TOP-10} (Invitrogen, Carlsbad,
États-Unis), suivant le protocole du fabriquant. Les cellules transformantes
possédant l'insert ont été discriminées par un crible bleu / blanc sur gélose
solide Luria-Bertani additionnée d'Ampicilline à \SI{75}{\ug\per\mL}, de XGal et
d' IPTG. Le sens d'insertion du fragment a été vérifié par PCR M13R-1392 (voir
annexe la fameuse annexe).
%% TODO schéma des amorces
Les inserts des clones positifs ont été séquencés par séquençage Sanger (GATC
Biotech, Constance, Allemagne). Les plasmides des clones validés par séquençage
ont été extraits par le kit Nucleospin-Plasmid (Macherey-Nagel, Düren,
Allemagne), et ouverts par digestion \emph{Spe}I.

La cassette de résistance à la kanamycine \emph{aphA}3 est amplifiée par PCR
avec les amorces 1408 et 1409. La région \emph{ancre} facilitant la
recombinaison est amplifiée avec les amorces 1410 et 1411. Les deux
amplicons obtenus ont été ligaturés dans le plasmide pGEM-T porteur des
constructions ouvert par \emph{Spe}I. La ligature a été réalisée simultanément
par le kit InFusion (Takara Clontech, Saint Germain en Layes, France). Le
produit de ligature obtenu a été inséré dans la souche optimisée pour la
chimio-compétence \emph{E.coli} Stellar (Takara Clontech). Les transformants ont
été séléctionnés sur milieu LB solide additionné d'Ampicilline à
\SI{75}{\ug\per\mL} de kanamycine à \SI{50}{\ug\per\mL}. Les transformants ont
été confirmés par PCR spécifique de l'insert avec les amorces XXXX et YYYY.

\addfig{%

  \begin{center}
  \setstretch{1.0}
    \rmfamily

    \tikzset{legende fleche/.style={Gray, dotted, thick, opacity = 0.6}}
    \tikzset{legende text/.style={black, text width = 5cm, font = \scriptsize, above}}
    \tikzset{align fleche/.style={midway, right, align = left, darkgray,
        font=\scriptsize}}

    \begin{tikzpicture}[join=round]

      \draw [->]
       (0, 0) node[above] {\includegraphics[width = 0.4\textwidth]{img/electropherogram.jpg}}
       -- ++(0, -1)
       node[align fleche, text width = 6cm ] {%
         \textsf{{\color{Black} a.}} %
         Analyses des électrophérogrammes obtenus par le programme
         \texttt{phred} et attribution du score de qualité aux bases.
         %
       }
       node[font = \footnotesize, below] {\texttt{\textcolor{kanr_col}{ATCG.......................ACCG}}}
       ;

       \node[font = \footnotesize] at (0, -2.5) {\texttt{\textcolor{gs_rec_col}{ATCG.......................ATCG}}}; % receveur
       \node[font = \footnotesize] at (0, -3.0) {\texttt{\textcolor{gs_don_col}{ACCG.......................ACCG}}}; % donneur
       \node[font = \footnotesize, opacity = 0.7] at (0, -3.5) {\texttt{\textcolor{kanr_col}{ATCG.......................ACCG}}}; % recombinant
       \node[gs_rec_col, font=\scriptsize] at (-4, -2.5) {Receveur :};
       \node[gs_don_col, font=\scriptsize] at (-4, -3.0) {Donneur :};
       \node[kanr_col, font=\scriptsize] at (-4, -3.5) {Recombinant :};

       \draw[bend left=90,  dotted, thick] [->]
       (+3.2, -1.2) to
       node[midway, right, align fleche, align = left, text width = 6cm] {%
         \textsf{{\color{Black} b.}} %
         Alignement avec la référence par \texttt{muscle}. %
         La référence est composée de l'alignement de la séquence receveuse
         sauvage et de la séquence donneuse synthétique.
         %
       } ++(0, -2.2) ;

       \draw[align fleche] [->] (0, -4.0) -- ++(0, -1.5) %
       node[align fleche, midway, right, align = left, text width = 7cm] {%
         \textsf{{\color{Black} c.}} %
         Détermination du sens des événèments de conversion et attribution du
         score de qualité aux bases. L'information de la position sur la
         séquence de référence permet de comparer les lectures de séquençage
         entre elles.} %
       node[font=\tiny, below] {%
         %
         \fontspec{Gill Sans}
         \begin{tabular}{>{\color{gs_rec_col}}c>{\color{gs_don_col}}c>{\color{kanr_col}}cccc}
           \toprule
           Receveur & Donneur & Recombinant & Position & Qualité & Conversion \\
           \midrule
           A        & A       & A           & 31      & 30      &          \\
           \rowcolor{LightGray}
           T        & C       & C           & 32      & 40      & oui      \\
           C        & C       & C           & 33      & 42      &          \\
           G        & G       & G           & 34      & 42      &          \\
           .        & .       & .           & .       & .       &          \\
           .        & .       & .           & .       & .       &          \\
           .        & .       & .           & .       & .       &          \\
           A        & A       & A           & 61      & 42      &          \\
           \rowcolor{LightGray}
           T        & C       & T           & 62      & 30      & non      \\
           C        & C       & C           & 63      & 40      &          \\
           G        & G       & G           & 64      & 28      &          \\
           \bottomrule
           %
         \end{tabular}
       };

       \draw[legende fleche] [<-] (3.8, -6.5) -- (4.4, -6.5) -- (5, -9.0);
       \draw[legende fleche] [<-] (3.8, -8.3) -- (4.4, -8.3) -- (5, -9.0) -- (8, -9.0) %
       node[opacity = 1, Gray, text width = 5.5cm, font=\scriptsize, above] {%
         Ces positions sont les positions d'intérêt. Elles correspondent à
         l'introduction d'une base G ou C par le donneur, alors que le receveur
         présente une base A ou T. La base présente chez le clone recombinant
         permet de déterminer la polarité de la conversion. Le score de qualité
         permet d'accorder plus ou moins de confiance à la base appelée par le
         programme \texttt{phred}.
         %
       } node {$\bullet$} ;


       %%
       %% Flèches montrant le basculement de ligne à colonne.
       %%
       \draw[dotted, kanr_col, thin, bend left ] [->] (+2.9, -3.7) to (-0.5, -8.8);
       \draw[dotted, kanr_col, thin, bend right] [->] (-2.9, -3.7) to (-0.7, -6.3);

  \end{tikzpicture}

\end{center}

  \caption[Analyses des zones de recombinaison]{\textbf{Exemple d'analyse de la
      zone de recombinaison pour un clone
      transformant} \\
    \rmfamily%
    \setstretch{1.1} %
    Les électrophérogrammes de séquençage obtenus ont été analysés en utilisant
    un programme permettant d'attribuer à chaque position un score de qualité.
    Les séquences obtenues ont été alignées à la référence, ce qui a permis
    d'inférer pour chaque site polymorphe le sens de la conversion du
    recombinant. }
  \label{img:align}
}

\subsection{Transformations d'\emph{Acinetobacter baylyi}}
\label{subsec:transfo}


\SI{1}{\ug} de plasmide a été extrait et linéarisé par digestion \emph{Sca}I.
L'enzyme a été inactivée par incubation \SI{10}{\minute} à \SI{70}{\celsius}.
\SI{390}{\uL} d'une culture pure d'\emph{Acinetobacter baylyi} ADP1,
avec une absorbance de \SI{0,8} à \SI{600}{\nm} ont été incubés pendant
\SI{1}{\hour} à \SI{28}{\celsius} en présence de \SI{200}{\ng} de plasmide linéarisé.
Les suspensions ont été incubées \SI{15}{\minute} à \SI{37}{\celsius} en présence de
DNAse à \SI{20}{\umol\per\L} pour éliminer les plasmides résiduels. Les cellules
recombinantes ont été sélectionnées par étalement en spirales (InterScience, St
Nom la Bretêche, France) sur milieu LB solide additionné de Kanamycine à
\SI{50}{\ug\per\mL} et incubés \SI{24}{\hour} à \SI{30}{\celsius}. Les
dénombrements ont été effectués via un compteur automatique
Scan\textsuperscript{\textregistered}1200 (InterScience).

96 clones isolés ont été purifiés par isolement sur milieu LB solide additionné
de kanamycine à \SI{50}{\umol\per\mL} et incubation \SI{24}{\hour} à
\SI{30}{\celsius}. Une colonie isolée par clone a été suspendue dans
\SI{50}{\uL} d'\ce{H_2O} ultra-pure. \SI{2}{\uL} de ces suspensions ont servi de
matrice pour amplifier les régions recombinantes en PCR par la Taq polymérase
haute fidélité Phusion (ThermoFischer, Waltham, États-Unis). L'amplification a
été confirmée par migration sur gel d'agarose à 1\%. Les amplicons ont été
séquencés par la technique de Sanger\cite{sanger_dna_1977} (GATC Biotech).

\subsection{Alignements}
\label{subsec:align}


Les spectrogrammes de séquençage reçus au format propriétaire \texttt{abi}
(Applied Biosystem, Foster City, États-Unis) ont été analysés par le programme
\texttt{phred} \cite{ewing_base-calling_1998} et convertis en format universel
\texttt{FASTA} (voir figure \ref{img:align}). Les séquences obtenues ont été
alignées à la référence par \texttt{ muscle v3.8.31} \cite{edgar_muscle:_2004}.
La référence en question correspond à l'alignement de la séquence sauvage et de
la séquence du gène synthétique, respectivement \emph{receveur} et
\emph{donneur} de l'évènement de recombinaison. Un script Python
\cite{cock_biopython:_2009} a été développé pour analyser les alignements
obtenus. Il infère les positions des SNPs d'intérêt dans l'alignement de
référence, et détermine le génotype du clone séquencé. Les alignements par paire
en colonne obtenus ont été analysés par \textrm{R} \texttt{3.2.3}
\cite{r_core_team_r:_2015}. Les programmes développés sont accessibles à
l'adresse \url{https://github.com/sam217pa/gbc-seq_mars}. Les données formattées
et les fonctions d'analyse ont été assemblées dans le package \textrm{R}
\texttt{gcbiasr} disponible à l'adresse
\url{https://github.com/sam217pa/gbc-gcbiasr}.
% \todo[inline]{inclure données dans le package}
