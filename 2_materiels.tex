%==============================================================================
% début de la page de figure.
% - stratégies envisagées
% - protocole de constructions moléculaires
%
\null
\vfill

% ------------------------------------------------------------
% TODO placer la figure des stratégies
% ------------------------------------------------------------
% ------------------------------------------------------------

% ------------------------------------------------------------
% TODO placer la figure des constructions
\begin{center}
  \setstretch{1.0}

  \begin{tikzpicture}[scale=0.7, line width = 2pt, join=round]

    \begin{scope}[shift={(-0.3, 10)}]
      % \draw[Green] (0, 0) arc (0:30:3);
      \draw[Green] (0, 0) arc (150:390:1);
      \draw[Green] (1, -0.5) node {\scriptsize pGEM-T};
      \draw[Green] (0.1, 0.3) node {T};
      \draw[Green] (1.7, 0.3) node {T};
    \end{scope}

    \begin{scope}[shift={(0, 11)}, line width = 0.5 pt]
      \foreach \y in {0,0.2,...,2} {
        \draw[Cerulean] (0, \y) -- (1, \y);
        \draw[Gold1   ] (0.9, \y) -- (1, \y);
        \draw[Cerulean] (-0.1, \y) node {\tiny A};
        \draw[Cerulean] (1.1, \y) node {\tiny A};
      }
    \end{scope}

    \begin{scope}[shift={(-0.3, 7)}]
      \draw[Green] (0, 0) arc (150:390:1);
      \draw[Gold1] (0, 0) arc (150:30:1) node[auto, above, sloped] {\tiny \texttt{SpeI\\}};
      \draw[Cerulean] (0, 0) arc (150:40:1) ;
    \end{scope}

    \begin{scope}[shift={(-5, 4)}]
      \draw[Green] (0, 0) -- (4, 0);
      \draw[Cerulean] (4, 0) -- (6, 0);
      \draw[Gold1] (6, 0) -- (6.2, 0);

      \draw[Gold1] (6, 0.5) -- (6.2, 0.5);
      \draw[Orchid3] (6.2, 0.5) -- (7.8, 0.5);
      \draw[SteelBlue1] (7.8, 0.5) -- (8.0, 0.5);

      \draw[SteelBlue1] (7.8, 1.0) -- (8.0, 1.0);
      \draw[Orange1] (8.0, 1.0) -- (9.8, 1.0);
      \draw[Brown1] (9.8, 1.0) -- (10.0, 1.0);

      \draw[Brown1] (0, 0) -- (0.2, 0);
    \end{scope}

    \begin{scope}[shift={(-7, 2.5)}]
      \draw[Green, rounded corners=2ex] (3, -1) rectangle (11, 0);
      \draw[Cerulean] (4, 0) -- (6, 0);
      \draw[Gold1] (6, 0) -- (6.2, 0);
      \draw[Gold1] (6, 0) -- (6.2, 0);
      \draw[Orchid3] (6.2, 0) -- (7.8, 0);
      \draw[SteelBlue1] (7.8, 0) -- (8.0, 0);
      \draw[SteelBlue1] (7.8, 0) -- (8.0, 0);
      \draw[Orange1] (8.0, 0) -- (9.8, 0);
      \draw[Brown1] (9.8, 0) -- (10.0, 0);
    \end{scope}

    \begin{scope}[shift={(2, 0)}]
      \begin{scope}[shift={(-5, 0)}, scale = 0.2]
        \draw[Green, rounded corners=0.2ex] (3, -1) rectangle (11, 0);
        \draw[Cerulean] (4, 0) -- (6, 0);
        \draw[Gold1] (6, 0) -- (6.2, 0);
        \draw[Gold1] (6, 0) -- (6.2, 0);
        \draw[Orchid3] (6.2, 0) -- (7.8, 0);
        \draw[SteelBlue1] (7.8, 0) -- (8.0, 0);
        \draw[SteelBlue1] (7.8, 0) -- (8.0, 0);
        \draw[Orange1] (8.0, 0) -- (9.8, 0);
        \draw[Brown1] (9.8, 0) -- (10.0, 0);
      \end{scope}

      \begin{scope}[shift={(-3, 0)}, scale = 0.2]
        \draw[Green, rounded corners=0.2ex] (3, -1) rectangle (11, 0);
        \draw[Cerulean] (4, 0) -- (6, 0);
        \draw[Gold1] (6, 0) -- (6.2, 0);
        \draw[Gold1] (6, 0) -- (6.2, 0);
        \draw[Orchid3] (6.2, 0) -- (7.8, 0);
        \draw[SteelBlue1] (7.8, 0) -- (8.0, 0);
        \draw[SteelBlue1] (7.8, 0) -- (8.0, 0);
        \draw[Orange1] (8.0, 0) -- (9.8, 0);
        \draw[Brown1] (9.8, 0) -- (10.0, 0);
      \end{scope}

      \begin{scope}[shift={(-1, 0)}, scale = 0.2]
        \draw[Green, rounded corners=0.2ex] (3, -1) rectangle (11, 0);
        \draw[Cerulean] (4, 0) -- (6, 0);
        \draw[Gold1] (6, 0) -- (6.2, 0);
        \draw[Gold1] (6, 0) -- (6.2, 0);
        \draw[Orchid3] (6.2, 0) -- (7.8, 0);
        \draw[SteelBlue1] (7.8, 0) -- (8.0, 0);
        \draw[SteelBlue1] (7.8, 0) -- (8.0, 0);
        \draw[Orange1] (8.0, 0) -- (9.8, 0);
        \draw[Brown1] (9.8, 0) -- (10.0, 0);
      \end{scope}

      \begin{scope}[shift={(-5, -0.5)}, scale = 0.2]
        \draw[Green, rounded corners=0.2ex] (3, -1) rectangle (11, 0);
        \draw[Cerulean] (4, 0) -- (6, 0);
        \draw[Gold1] (6, 0) -- (6.2, 0);
        \draw[Gold1] (6, 0) -- (6.2, 0);
        \draw[Orchid3] (6.2, 0) -- (7.8, 0);
        \draw[SteelBlue1] (7.8, 0) -- (8.0, 0);
        \draw[SteelBlue1] (7.8, 0) -- (8.0, 0);
        \draw[Orange1] (8.0, 0) -- (9.8, 0);
        \draw[Brown1] (9.8, 0) -- (10.0, 0);
      \end{scope}

      \begin{scope}[shift={(-3, -0.5)}, scale = 0.2]
        \draw[Green, rounded corners=0.2ex] (3, -1) rectangle (11, 0);
        \draw[Cerulean] (4, 0) -- (6, 0);
        \draw[Gold1] (6, 0) -- (6.2, 0);
        \draw[Gold1] (6, 0) -- (6.2, 0);
        \draw[Orchid3] (6.2, 0) -- (7.8, 0);
        \draw[SteelBlue1] (7.8, 0) -- (8.0, 0);
        \draw[SteelBlue1] (7.8, 0) -- (8.0, 0);
        \draw[Orange1] (8.0, 0) -- (9.8, 0);
        \draw[Brown1] (9.8, 0) -- (10.0, 0);
      \end{scope}

      \begin{scope}[shift={(-1, -0.5)}, scale = 0.2]
        \draw[Green, rounded corners=0.2ex] (3, -1) rectangle (11, 0);
        \draw[Cerulean] (4, 0) -- (6, 0);
        \draw[Gold1] (6, 0) -- (6.2, 0);
        \draw[Gold1] (6, 0) -- (6.2, 0);
        \draw[Orchid3] (6.2, 0) -- (7.8, 0);
        \draw[SteelBlue1] (7.8, 0) -- (8.0, 0);
        \draw[SteelBlue1] (7.8, 0) -- (8.0, 0);
        \draw[Orange1] (8.0, 0) -- (9.8, 0);
        \draw[Brown1] (9.8, 0) -- (10.0, 0);
      \end{scope}
    \end{scope}

    \begin{scope}[shift={(-2.5, -1)}]
      \begin{scope}[shift={(0, -1)}, scale = 0.4]
        \draw[Cerulean] (4, 0) -- (6, 0);
        \draw[Gold1] (6, 0) -- (6.2, 0);
        \draw[Gold1] (6, 0) -- (6.2, 0);
        \draw[Orchid3] (6.2, 0) -- (7.8, 0);
        \draw[SteelBlue1] (7.8, 0) -- (8.0, 0);
        \draw[SteelBlue1] (7.8, 0) -- (8.0, 0);
        \draw[Orange1] (8.0, 0) -- (9.8, 0);
        \draw[Brown1] (9.8, 0) -- (10.0, 0);
        \draw[Green] (2, 0) -- (4, 0);
        \draw[Green] (10, 0) -- (12, 0);
      \end{scope}
      \begin{scope}[shift={(0, -1.2)}, scale = 0.4]
        \draw[Cerulean] (4, 0) -- (6, 0);
        \draw[Gold1] (6, 0) -- (6.2, 0);
        \draw[Gold1] (6, 0) -- (6.2, 0);
        \draw[Orchid3] (6.2, 0) -- (7.8, 0);
        \draw[SteelBlue1] (7.8, 0) -- (8.0, 0);
        \draw[SteelBlue1] (7.8, 0) -- (8.0, 0);
        \draw[Orange1] (8.0, 0) -- (9.8, 0);
        \draw[Brown1] (9.8, 0) -- (10.0, 0);
        \draw[Green] (2, 0) -- (4, 0);
        \draw[Green] (10, 0) -- (12, 0);
      \end{scope}
      \begin{scope}[shift={(0, -1.4)}, scale = 0.4]
        \draw[Cerulean] (4, 0) -- (6, 0);
        \draw[Gold1] (6, 0) -- (6.2, 0);
        \draw[Gold1] (6, 0) -- (6.2, 0);
        \draw[Orchid3] (6.2, 0) -- (7.8, 0);
        \draw[SteelBlue1] (7.8, 0) -- (8.0, 0);
        \draw[SteelBlue1] (7.8, 0) -- (8.0, 0);
        \draw[Orange1] (8.0, 0) -- (9.8, 0);
        \draw[Brown1] (9.8, 0) -- (10.0, 0);
        \draw[Green] (2, 0) -- (4, 0);
        \draw[Green] (10, 0) -- (12, 0);
      \end{scope}
      \begin{scope}[shift={(0, -1.6)}, scale = 0.4]
        \draw[Cerulean] (4, 0) -- (6, 0);
        \draw[Gold1] (6, 0) -- (6.2, 0);
        \draw[Gold1] (6, 0) -- (6.2, 0);
        \draw[Orchid3] (6.2, 0) -- (7.8, 0);
        \draw[SteelBlue1] (7.8, 0) -- (8.0, 0);
        \draw[SteelBlue1] (7.8, 0) -- (8.0, 0);
        \draw[Orange1] (8.0, 0) -- (9.8, 0);
        \draw[Brown1] (9.8, 0) -- (10.0, 0);
        \draw[Green] (2, 0) -- (4, 0);
        \draw[Green] (10, 0) -- (12, 0);
      \end{scope}
    \end{scope}

  \end{tikzpicture}
\end{center}
\caption[Constructions moléculaires]{%
  \textbf{Constructions moléculaires} \\
  \rmfamily %
  Les gènes de synthèse sont amplifié par PCR spécifique. Des bases adénines
  sont ajoutées naturellement sur les fragments au cours des derniers cycles
  de PCR. Ces fragments sont ensuite ligaturés dans le plasmide pGEM-T.
  Celui-ci est disponible ouvert avec des bases thymines sortantes. La
  ligature est catalysée par la T4DNA Ligase. Le plasmide obtenu est
  linéarisé par digestion enzymatique au site d'insertion des fragments.
  
  La cassette de résistance à la kanamycine KanR ainsi que la région
  facilitant l'homologie Ancre sont préalablement amplifiées par PCR en
  utilisant des amorces porteuses de régions 3' flottantes, complémentaires
  des extrémités des fragments voisins. La ligature entre les trois
  fragments obtenus est réalisée à l'aide du kit Takara InFusion. Les
  constructions plasmidiques obtenues sont amplifiées clonalement par
  culture en milieu sélectif liquide, et linéarisées par digestion
  enzymatique. 
}
% ------------------------------------------------------------
\vfill

\thispagestyle{empty}
\addtocounter{page}{-1}
\newpage

% fin de la page de figure
% ==============================================================================

% ==============================================================================
\section{Matériels \& Méthodes}
\label{sec:materiel}
% ==============================================================================


\afterpage{%
  \null
  \vfill
  \begin{center}
  \rmfamily
  \setstretch{1.0}

  \tikzset{
    culture/.pic={
      \node at (0, 0) {\includegraphics[width = 0.015\textwidth]{img/lb_liquide.png}};
    }
  }

  \tikzset{
    transfo/.pic={
      \node at (0, 0) {\includegraphics[width = 0.03\textwidth]{img/eppendorf_full.png}};
    }
  }

  \tikzset{
    etalement/.pic={
      \node at (0, 0) {\includegraphics[width = 0.05\textwidth]{img/etalement.png}};
    }
  }

  \tikzset{
    petri/.pic={
      \node at (0, 0) {\includegraphics[width = 0.05\textwidth]{img/petri_open.png}};
    }
  }

  \tikzset{
    boite/.pic={
      \node at (0, 0) {\includegraphics[width = 0.06\textwidth]{img/boite_96.png}};
    }
  }

  \tikzset{
    pcr plate/.pic={
      \node at (0, 0) {\includegraphics[width = 0.05\textwidth]{img/pcr_plate.jpeg}};
    }
  }

  \tikzset{legende fleche/.style={Gray, dotted, thick, opacity = 0.6}}
  \tikzset{legende text/.style={black, text width = 7cm, font = \scriptsize, above}}

  \begin{tikzpicture}

    % tube eppendorf seul
    \draw [->]
    (0, 0) node[above]
    {\includegraphics[width=0.05\textwidth]{img/eppendorf.png}}
    % boite de petri
    -- (0, -1) node[below]
    {\includegraphics[width=0.2\textwidth]{img/petri_open.png}}
    ;

    %% tubes de culture
    \draw[out = 210, in = 90] [->] (0, -3) to ++(-3.5, -1) pic[below] {culture};
    \draw[out = 210, in = 90] [->] (0, -3) to ++(-2.5, -1) pic[below] {culture};
    \draw[out = -90, in = 90] [->] (0, -3) to ++(-0.5, -1) pic[below] {culture};
    \draw[out = -90, in = 90] [->] (0, -3) to ++(+0.5, -1) pic[below] {culture};
    \draw[out = -30, in = 90] [->] (0, -3) to ++(+2.5, -1) pic[below] {culture};
    \draw[out = -30, in = 90] [->] (0, -3) to ++(+3.5, -1) pic[below] {culture};

    %% tubes de transfo
    \draw [->] (-3.5, -7) node[above, font = \scriptsize] {GC} -- ++(0, -0.7) pic[below] {transfo};
    \draw [->] (-2.5, -7) node[above, font = \scriptsize] {GC} -- ++(0, -0.7) pic[below] {transfo};
    \draw [->] (-0.5, -7) node[above, font = \scriptsize] {AT} -- ++(0, -0.7) pic[below] {transfo};
    \draw [->] (+0.5, -7) node[above, font = \scriptsize] {AT} -- ++(0, -0.7) pic[below] {transfo};
    \draw [->] (+2.5, -7) node[above, font = \scriptsize] {\(\nicefrac{GC}{AT}\)} -- ++(0, -0.7) pic[below] {transfo};
    \draw [->] (+3.5, -7) node[above, font = \scriptsize] {\(\nicefrac{AT}{GC}\)} -- ++(0, -0.7) pic[below] {transfo};

    %% étalement sur boiîtes de pétri
    \draw [->] (-3.5, -9) -- ++(0, -0.5) pic[below] {petri};
    \draw [->] (-2.5, -9) -- ++(0, -0.5) pic[below] {petri};
    \draw [->] (-0.5, -9) -- ++(0, -0.5) pic[below] {petri};
    \draw [->] (+0.5, -9) -- ++(0, -0.5) pic[below] {petri};
    \draw [->] (+2.5, -9) -- ++(0, -0.5) pic[below] {petri};
    \draw [->] (+3.5, -9) -- ++(0, -0.5) pic[below] {petri};

    %% purification sur boîtes
    \draw [->] (-3.5, -10.5) -- ++(0, -0.5) pic[below] {etalement};
    \draw [->] (-2.5, -10.5) -- ++(0, -0.5) pic[below] {etalement};
    \draw [->] (-0.5, -10.5) -- ++(0, -0.5) pic[below] {etalement};
    \draw [->] (+0.5, -10.5) -- ++(0, -0.5) pic[below] {etalement};
    \draw [->] (+2.5, -10.5) -- ++(0, -0.5) pic[below] {etalement};
    \draw [->] (+3.5, -10.5) -- ++(0, -0.5) pic[below] {etalement};

    %% suspension dans 50µL d'eau
    \draw [->] (-3.5, -12.2) -- ++(0, -0.5) pic[below] {boite};
    \draw [->] (-2.5, -12.2) -- ++(0, -0.5) pic[below] {boite};
    \draw [->] (-0.5, -12.2) -- ++(0, -0.5) pic[below] {boite};
    \draw [->] (+0.5, -12.2) -- ++(0, -0.5) pic[below] {boite};
    \draw [->] (+2.5, -12.2) -- ++(0, -0.5) pic[below] {boite};
    \draw [->] (+3.5, -12.2) -- ++(0, -0.5) pic[below] {boite};

    %% PCR
    \draw [->] (-3.5, -13.6) -- ++(0, -0.5) pic[below] {pcr plate};
    \draw [->] (-2.5, -13.6) -- ++(0, -0.5) pic[below] {pcr plate};
    \draw [->] (-0.5, -13.6) -- ++(0, -0.5) pic[below] {pcr plate};
    \draw [->] (+0.5, -13.6) -- ++(0, -0.5) pic[below] {pcr plate};
    \draw [->] (+2.5, -13.6) -- ++(0, -0.5) pic[below] {pcr plate};
    \draw [->] (+3.5, -13.6) -- ++(0, -0.5) pic[below] {pcr plate};

    %% gels électrophorèse
    \draw [->] (-3.5, -15) -- ++(0, -0.5) node[below] {\includegraphics[width = 0.06\textwidth]{img/s1_plate.jpg}};
    \draw [->] (-2.5, -15) -- ++(0, -0.5) node[below] {\includegraphics[width = 0.06\textwidth]{img/s3_plate.jpg}};
    \draw [->] (-0.5, -15) -- ++(0, -0.5) node[below] {\includegraphics[width = 0.06\textwidth]{img/w1_plate.jpg}};
    \draw [->] (+0.5, -15) -- ++(0, -0.5) node[below] {\includegraphics[width = 0.06\textwidth]{img/w2_plate.jpg}};
    \draw [->] (+2.5, -15) -- ++(0, -0.5) node[below] {\includegraphics[width = 0.06\textwidth]{img/sw_plate.jpg}};
    \draw [->] (+3.5, -15) -- ++(0, -0.5) node[below] {\includegraphics[width = 0.06\textwidth]{img/ws_plate.jpg}};

    %%
    %% ANNOTATIONS
    %%

    % %% trait de légende
    \draw[legende fleche] [<-] (1.5, -1.5) -- (2, -1.5) -- (3, 0) -- (8, 0)
    node[legende text] {%
      Une culture cryogénisée d'\emph{Acinetobacter baylyi ADP1} est purifiée
      par étalement sur milieu non sélectif de Luria Bertani.
      %
    } node {$\bullet$} ;

    \draw[legende fleche] [<-] (4, -4.5) -- (8, -4.5) node[legende text] {%
      Une colonie isolée est pré-cultivée pendant \SI{24}{\hour} à \SI{30}{\celsius}
      en milieu LB. \SI{50}{\uL} de cette pré-culture sont resuspensdus dans
      \SI{5}{\mL} de LB.
      %
    } node {$\bullet$} ;

    \draw[legende fleche] [<-] (4, -8.0) -- (8, -8.0) node[legende text] {%
      Lorsque la culture a atteint une absorbance à \SI{600}{\nm} de \(0.8\),
      \SI{400}{\ng} de construction plasmidique linéarises sont ajoutées dans
      \SI{390}{\uL} de culture, et incubés pendant \SI{1}{\hour} à \SI{30}{\celsius}.

      Les transformations ont été réalisées avec trois constructions différentes
      : les constructions introduisant des G et des C, celles introduisant des A
      et des T, et celles introduisant les deux à la fois en alternance.
      %
    } node {$\bullet$} ;

    \draw[legende fleche] [<-] (4, -10.0) -- (8, -10.0) node[legende text] {%
      Après \SI{1}{\hour} d'incubation, les plasmides résiduels sont dégradés par
      l'ajout de DNAse à \SI{20}{\ug\per\mL}, et incubation \SI{15}{\min} à
      \SI{37}{\celsius}. La culture est ensuite étalée en spirale sur milieu
      sélectif LB additionné de kanamycine à \SI{50}{\ug\per\mL}, et incubés
      pendant \SI{24}{\hour} à \SI{37}{\celsius}.
      %
    } node {$\bullet$} ;

    \draw[legende fleche] [<-] (4, -11.5) -- (8, -11.5) node[legende text]
    {%
      \(96\) colonies transformantes, résistantes à la kanamycines sont isolées
      par étalement sur le même milieu sélectif LB+Kan.
      %
    } node {$\bullet$} ;


    \draw[legende fleche] [<-] (4, -13.0) -- (8, -13.0) node[legende text] {%
      Pour chaque transformant isolé, 1 colonie est prélevée à l'œse et
      suspendue dans \SI{50}{\uL} d'eau ultrapure.
      %
    } node {$\bullet$} ;

    \draw[legende fleche] [<-] (4, -14.5) -- (8, -14.5) node[legende text] {%
      Des PCRs cibles de la région d'intérêt sont réalisées en utilisant
      \SI{2}{\uL} des suspensions précédemment réalisées.
      %
    } node {$\bullet$} ;

    \draw[legende fleche] [<-] (4, -16.2) -- (8, -16.2) node[legende text] {%
      Les PCRs ont été contrôlées par électrophorèse sur gel d'agarose à
      \(1\%\). Les amplicons ont été ensuite séquencés par la technique de
      Sanger.
      %
    } node {$\bullet$} ;

  \end{tikzpicture}

\end{center}
  \caption[Protocoles de transformation]{\textbf{Protocole de transformation et
      d'obtention des amplicons des zones de recombinaison.} \rmfamily%
    \setstretch{1.1} %
  }
  \label{img:manip}
  \vfill
  \thispagestyle{empty}
  \addtocounter{page}{-1}
  \newpage
}





\subsection{Constructions des plasmides}
\label{subsec:constructions}

Nous avons utilisé deux approches (voir figure \ref{img:strategies}). La
première consiste à introduire exclusivement des bases GC ou AT. Sous
l'hypothèse d'une conversion biaisée vers GC, les régions converties devraient
être plus longues lorsqu'on introduit exclusivement des bases GC que lorsqu'on
introduit des bases AT. Sur un locus du génome d'\emph{Acinetobacter baylyi},
$23$ dinucléotides séparés par trente bases ont été choisis\footnote{Ils
  introduisent 3\% de divergence avec la séquence sauvage. Un plus grand nombre
  de marqueurs diminue les fréquences de recombinaison ; un nombre inférieur
  diminue la qualité du signal.}. Ce sont tous des dinucléotides alternant AT et
GC. Un gène de synthèse a été conçu pour introduire des mésappariemments entre
les bases A ou T des dinucléotides choisis et une base G ou C~: cette
construction sera appelée GC dans la suite de ce rapport. La construction AT a
été conçue de façon à introduire un mésappariemment entre les bases G ou C des
dinucléotides choisis et une base A ou T.


La seconde approche consiste à introduire à la fois des bases AT et CG. En
l'absence de biais, le rapport en base ne devrait pas être affecté : on
introduit autant de bases GC que de bases AT. Deux constructions ont donc été
conçues sur la base des constructions exclusivement GC ou AT. La construction
GC/AT alterne l'introduction d'une base GC avec celle d'une base AT. La
construction AT/GC alterne l'introduction d'une base AT avec celle d'une base
GC.

Une première partie du travail a consisté à construire le vecteur permettant
d'introduire le gène synthétique porteur des mésappariemments chez
\emph{Acinetobacter} par transformation naturelle. Les gènes synthétiques ont
été synthétisés par quelqu'un.
%% TODO Franck qui a synthétisé les gènes ?
Les PCRs spécifiques du gène d'intérêt ont été réalisées. Les amplicons ainsi
obtenus ont été insérés par ligature dans le plasmide pGEM-T. Le plasmide a été
inséré dans la souche d'\textit{E.coli} One
Shot\textsuperscript{{\textregistered}} \texttt{TOP-10} (Invitrogen, Carlsbad,
États-Unis), suivant le protocole du fabriquant. Les cellules transformantes
possédant l'insert ont été discriminées par un crible bleu / blanc sur gélose
solide Luria-Bertani \ac{lb} additionnée d'Ampicilline à \SI{75}{\ug\per\mL}, de
XGal et d' IPTG. Le sens d'insertion du fragment a été vérifié par PCR M13R-1392
(voir annexe \ref{img:amorces}). Les inserts des clones positifs ont été
séquencés par séquençage Sanger (GATC Biotech, Constance, Allemagne). Les
plasmides des clones validés par séquençage ont été extraits par le kit
Nucleospin-Plasmid (Macherey-Nagel, Düren, Allemagne), et ouverts par digestion
\emph{Spe}I.

La cassette de résistance à la kanamycine \emph{aphA}3 est amplifiée par PCR
avec les amorces 1408 et 1409. La région \emph{ancre} facilitant la
recombinaison est amplifiée avec les amorces 1410 et 1411. Les deux
amplicons obtenus ont été ligaturés dans le plasmide pGEM-T porteur des
constructions ouvert par \emph{Spe}I. La ligature a été réalisée simultanément
par le kit InFusion (Takara Clontech, Saint Germain en Layes, France). Le
produit de ligature obtenu a été inséré dans la souche optimisée pour la
chimio-compétence \emph{E.coli} Stellar (Takara Clontech). Les transformants ont
été séléctionnés sur milieu LB solide additionné d'Ampicilline à
\SI{75}{\ug\per\mL} de kanamycine à \SI{50}{\ug\per\mL}. Les transformants ont
été confirmés par PCR spécifique de l'insert avec les amorces 1393 et 1411
(voir annexe \ref{img:amorces}).

% -----------------------------------------------------------------------------
\subsection{Transformations d'\emph{Acinetobacter baylyi}}
\label{subsec:transfo}
% -----------------------------------------------------------------------------


\SI{1}{\ug} de plasmide a été extrait et linéarisé par digestion \emph{Sca}I.
L'enzyme a été inactivée par incubation \SI{10}{\minute} à \SI{70}{\celsius}.
\SI{390}{\uL} d'une culture pure d'\emph{Acinetobacter baylyi} ADP1, avec une
absorbance de \SI{0,8} à \SI{600}{\nm} ont été incubés pendant \SI{1}{\hour} à
\SI{28}{\celsius} en présence de \SI{200}{\ng} de plasmide linéarisé. Les
suspensions ont été ensuite incubées \SI{15}{\minute} à \SI{37}{\celsius} en
présence de DNAse à \SI{20}{\umol\per\L} pour éliminer les plasmides résiduels.
Les cellules recombinantes ont été sélectionnées par étalement en spirales
(InterScience, St Nom la Bretêche, France) sur milieu LB solide additionné de
kanamycine à \SI{50}{\ug\per\mL} et incubées \SI{24}{\hour} à \SI{30}{\celsius}.
Les dénombrements ont été effectués via un compteur automatique
Scan\textsuperscript{\textregistered}1200 (InterScience).

96 clones isolés ont été purifiés par isolement sur milieu LB solide additionné
de kanamycine à \SI{50}{\umol\per\mL} et incubés \SI{24}{\hour} à
\SI{30}{\celsius}. Une colonie isolée par clone a été mise en suspension dans
\SI{50}{\uL} d'\ce{H_2O} ultra-pure. \SI{2}{\uL} de ces suspensions ont servi de
matrice pour amplifier les régions recombinantes en PCR par la Taq polymérase
haute fidélité Phusion (ThermoFischer, Waltham, États-Unis). L'amplification a
été confirmée par migration sur gel d'agarose à 1\%. Les amplicons ont été
séquencés par la technique de Sanger\cite{sanger_dna_1977} (GATC Biotech).

\afterpage{%
 \null
 \vfill
  \begin{center}
  \setstretch{1.0}
    \rmfamily

    \tikzset{legende fleche/.style={Gray, dotted, thick, opacity = 0.6}}
    \tikzset{legende text/.style={black, text width = 5cm, font = \scriptsize, above}}
    \tikzset{align fleche/.style={midway, right, align = left, darkgray,
        font=\scriptsize}}

    \begin{tikzpicture}[join=round]

      \draw [->]
       (0, 0) node[above] {\includegraphics[width = 0.4\textwidth]{img/electropherogram.jpg}}
       -- ++(0, -1)
       node[align fleche] {Conversion en \\ FASTA par Phred}
       node[below] {\texttt{\textcolor{am1_col}{ATCG.......................ACCG}}}
       ;

       \node at (0, -2.5) {\texttt{\textcolor{gs_rec_col}{ATCG.......................ATCG}}}; % receveur
       \node at (0, -3.0) {\texttt{\textcolor{gs_don_col}{ACCG.......................ACCG}}}; % donneur
       \node[opacity = 0.7] at (0, -3.5) {\texttt{\textcolor{am1_col}{ATCG.......................ACCG}}}; % recombinant
       \node[gs_rec_col, font=\scriptsize] at (-4, -2.5) {Receveur :};
       \node[gs_don_col, font=\scriptsize] at (-4, -3.0) {Donneur :};

       \draw[bend left=90,  dotted, thick] [->]
       (+3.2, -1.2) to
       node[midway, right, align fleche, align = left, text width = 6cm] {%
         %
         \textbf{Alignement avec la référence par \texttt{muscle}.} La référence
         est composée de l'alignement de la séquence receveuse sauvage et de la
         séquence donneuse synthétique.
         %
       } ++(0, -2.2) ;

       \draw[align fleche] [->] (0, -4.0) -- ++(0, -1.5) %
       node[align fleche, midway, right, align = left, text width = 7cm] {%
         Détermination de la polarité des événèments de conversion \\
         et attribution du score de qualité aux bases. L'information de la
         position sur la séquence de référence permet de comparer les lectures
         de séquençage entre elles.} %
       node[font=\tiny, below] {%
         %
         \fontspec{Gill Sans}
         \begin{tabular}{>{\color{gs_rec_col}}c>{\color{gs_don_col}}c>{\color{am1_col}}cccc}
           \toprule
           \head{Receveur} & \head{Donneur} & \head{Recombinant} & \head{Position} & \head{Qualité} & \head{Conversion} \\
           \midrule
           A        & A       & A           & 31      & 30      &          \\
           \rowcolor{LightGray}
           T        & C       & C           & 32      & 40      & oui      \\
           C        & C       & C           & 33      & 42      &          \\
           G        & G       & G           & 34      & 42      &          \\
           .        & .       & .           & .       & .       &          \\
           .        & .       & .           & .       & .       &          \\
           .        & .       & .           & .       & .       &          \\
           A        & A       & A           & 61      & 42      &          \\
           \rowcolor{LightGray}
           T        & C       & T           & 62      & 30      & non      \\
           C        & C       & C           & 63      & 40      &          \\
           G        & G       & G           & 64      & 28      &          \\
           \bottomrule
           %
         \end{tabular}
       };

       \draw[legende fleche] [<-] (3.8, -6.5) -- (4.4, -6.5) -- (5, -8.5);
       \draw[legende fleche] [<-] (3.8, -8.3) -- (4.4, -8.3) -- (5, -8.5) -- (8,
       -8.5) %
       node[opacity = 1, DarkGray, text width = 5cm, font=\scriptsize, above] {%
         Ces positions sont les positions d'intérêt. Elles correspondent à
         l'introduction d'une base G ou C par le donneur, alors que le receveur
         présente une base A ou T. La base présente chez le clone recombinant
         permet de déterminer la polarité de la conversion. Le score de qualité
         permet d'accorder plus ou moins de confiance à la base appelée par le
         programme \texttt{phred}.
         %
       } node {$\bullet$} ;


       %%
       %% Flèches montrant le basculement de ligne à colonne.
       %%
       \draw[dotted, am1_col, thin, bend left ] [->] (+2.9, -3.7) to (-0.5, -8.8);
       \draw[dotted, am1_col, thin, bend right] [->] (-2.9, -3.7) to (-0.7, -6.3);

  \end{tikzpicture}

\end{center}

  \caption[Analyses des zones de recombinaison]{\textbf{Exemple d'analyse de la
      zone de recombinaison pour un clone
      transformant} \\
    \rmfamily%
    \setstretch{1.1} %
    Les électrophérogrammes de séquençage obtenus ont été analysés en utilisant
    un programme permettant d'attribuer à chaque position un score de qualité.
    Les séquences obtenues ont été alignées à la référence, ce qui a permis
    d'inférer pour chaque site polymorphe le sens de la conversion du
    recombinant. }
  \label{img:align}
  \vfill
  \thispagestyle{empty}
  \addtocounter{page}{-1}
  \newpage
}

\newpage

% ------------------------------------------------------------------------------
\subsection{Alignements}
\label{subsec:align}
% ------------------------------------------------------------------------------

Les spectrogrammes de séquençage reçus au format propriétaire \texttt{abi}
(Applied Biosystem, Foster City, États-Unis) ont été analysés par le programme
\texttt{phred} \cite{ewing_base-calling_1998} et convertis en format universel
\texttt{FASTA} (voir figure \ref{img:align}). Les séquences obtenues ont été
alignées à la référence par \texttt{ muscle v3.8.31} \cite{edgar_muscle:_2004}.
La référence en question correspond à l'alignement de la séquence sauvage et de
la séquence du gène synthétique, respectivement \emph{receveur} et
\emph{donneur} de l'évènement de recombinaison. Un programme Python
\cite{cock_biopython:_2009} a été développé pour analyser les alignements
obtenus. Il détermine les positions des SNPs d'intérêt dans l'alignement de
référence, et infère le génotype du clone séquencé. Les alignements par paire
en colonne obtenus ont été analysés par \textrm{R} \texttt{3.2.3}
\cite{r_core_team_r:_2015}. Les programmes développés sont accessibles à
l'adresse \url{https://github.com/sam217pa/gbc-seq_mars}. Les données formattées
et les fonctions d'analyse ont été assemblées dans le package \textrm{R}
\texttt{gcbiasr} disponible à l'adresse
\url{https://github.com/sam217pa/gbc-gcbiasr}.
% \todo[inline]{inclure données dans le package}
